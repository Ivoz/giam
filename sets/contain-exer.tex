\begin{enumerate}
\item Insert either $\in$ or $\subseteq$ in the blanks in the following 
sentences (in order to produce true sentences).


\begin{tabular}{lcl}
\rule{0pt}{16pt}i) $1$ \underline{\rule{36pt}{0pt}} $\{3, 2, 1, \{a, b\}\}$ & \rule{36pt}{0pt} & iii) $\{a, b\}$  \underline{\rule{36pt}{0pt}} $\{3, 2, 1, \{a, b\}\}$ \\
\rule{0pt}{16pt}ii) $\{a\}$ \underline{\rule{36pt}{0pt}} $\{a, \{a, b\}\}$ & &
iv) $\{\{a, b\}\}$  \underline{\rule{36pt}{0pt}} $\{a, \{a, b\}\}$ \\
\end{tabular}

\hint{$\in$, $\subseteq$, $\in$, $\subseteq$}

\item  Suppose that $p$ is a prime, for each $n$ in $\Integers^+$, 
define the set $P_n = \{ x \in \Integers^+ \suchthat \, p^n \divides x \}$.  
Conjecture and prove a statement about the containments between these sets.

\hint{When $p=2$ we have seen these sets.  $P_1$ is the even numbers, $P_2$ is the doubly-even numbers,
etc.}

\item  Provide a counterexample to dispel the notion that a subset must
have fewer elements than its superset.

\hint{A subset is called {\em proper} if it is neither empty nor equal to the superset.   If
we are talking about finite sets then the proper subsets do indeed have fewer elements
than the supersets.  Among infinite sets it is possible to have proper subsets having the same 
number of elements as their superset, for example there are just as many even natural numbers
as there are natural numbers all told.}

\item  We have seen that $A \subseteq B$ corresponds to $M_A \implies M_B$.
What corresponds to the contrapositive statement?

\hint{Turn ``logical negation'' into ``set complement'' and reverse the direction of the inclusion.}
 

\item Determine two sets $A$ and $B$ such that both of the sentences
$A \in B$ and $A \subseteq B$ are true.

\hint{The smallest example I can think of would be $A=\emptyset$ and $B=\{\emptyset\}$.  You should come up with a different example.}

\item Prove that the set of perfect fourth powers is contained in the
set of perfect squares.

\hint{It would probably be helpful to have precise definitions of the sets described in the problem.

The fourth powers are
\[ F = \{x \suchthat \exists y \in \Integers, x=y^4 \}. \]

The squares are 
\[ S = \{x \suchthat \exists z \in \Integers, x=z^2 \}. \]

To show that one set is contained in another, we need to show that the first set's membership
criterion implies that of the second set.}

\end{enumerate}



%% Emacs customization
%% 
%% Local Variables: ***
%% TeX-master: "GIAM-hw.tex" ***
%% comment-column:0 ***
%% comment-start: "%% "  ***
%% comment-end:"***" ***
%% End: ***

