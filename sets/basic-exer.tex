\begin{enumerate}
\item What is the power set of $\emptyset$?  Hint: if you got the last exercise
in the chapter you'd know that this power set has $2^0 = 1$ element.

\hint{The power set of a set always includes the empty set as well as the whole set that we
are forming the power set of.  In this case they happen to coincide so ${\mathcal P}(\emptyset) = \{ \emptyset \}$.  Notice that $2^0 =1$.}

\item Try iterating the power set operator.  What is ${\mathcal P}({\mathcal P}(\emptyset))$?  What is ${\mathcal P}({\mathcal P}({\mathcal P}(\emptyset)))$?

\hint{I won't spoil you're fun, but as a check ${\mathcal P}({\mathcal P}(\emptyset))$ should have $2$ elements, and ${\mathcal P}({\mathcal P}({\mathcal P}(\emptyset)))$ should have $4$.}

\item Determine the following cardinalities.
  \begin{enumerate}
    \item $A = \{ 1, 2, \{3, 4, 5\}\} \quad |A| = $\rule{36pt}{1pt}
    \item $B = \{ \{1, 2, 3, 4, 5\} \} \quad |B| = $\rule{36pt}{1pt}  
  \end{enumerate}

\hint{Three and one}

\item What, in Logic, corresponds the notion $\emptyset$ in Set theory?

\hint{A contradiction.}

\item What, in Set theory, corresponds to the notion $t$ (a tautology) in Logic?

\hint{The universe of discourse.}

\item What is the truth set of the proposition $P(x) = $ ``3 divides $x$ and 2 divides $x$''?

\hint{ The set of all multiples of $6$.}

\item Find a logical open sentence such that $\{0, 1, 4, 9, \ldots \}$ is
its truth set.

\hint{Many answers are possible.  Perhaps the easiest is $\exists y \in \Integers, x = y^2$.}

\item How many singleton sets are there in the power set of 
$\{a,b,c,d,e\}$?  ``Doubleton'' sets?

\hint{5, 10}

\item How many 8 element subsets are there in
\[ {\mathcal P}(\{a,b,c,d,e,f,g,h,i,j,k,l,m,n,o,p\})? \]

\hint{ $\binom{16}{8} = 12870$}

\item How many singleton sets are there in the power set of 
$\{1,2,3, \ldots n\}$?

\hint{$n$}

\end{enumerate}



%% Emacs customization
%% 
%% Local Variables: ***
%% TeX-master: "GIAM-hw.tex" ***
%% comment-column:0 ***
%% comment-start: "%% "  ***
%% comment-end:"***" ***
%% End: ***

