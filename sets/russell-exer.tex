\begin{enumerate}
\item Verify that $(A \implies {\lnot}A) \land ({\lnot}A \implies A)$
is a logical contradiction in two ways:  by filling out a truth table and 
using the laws of logical equivalence.

\hint{In order to get started on this you'll need to convert the conditionals into equivalent
disjunctions.  Recall that $X \implies Y \; \equiv \; {\lnot}X \lor Y$.}

\item One way out of Russell's paradox is to declare that the collection
of sets that don't contain themselves as elements is not a set itself.
Explain how this circumvents the paradox. 

\hint{If it's not a set then it doesn't necessarily have to have the property that we
can be {\em sure} whether an element is in it or not.}

\end{enumerate}


%% Emacs customization
%% 
%% Local Variables: ***
%% TeX-master: "GIAM-hw.tex" ***
%% comment-column:0 ***
%% comment-start: "%% "  ***
%% comment-end:"***" ***
%% End: ***

