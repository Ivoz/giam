\begin{enumerate}
\item  Prove that positive numbers of the form $3k +1$ are equinumerous with
positive numbers of the form $4k + 2$.

\wbvfill

\item Prove that $\displaystyle f(x) =  c + \frac{(x-a)(d-c)}{(b-a)}$ 
provides a bijection from the interval $[a, b]$ to the interval $[c, d]$.

\wbvfill

\workbookpagebreak

\item Prove that any two circles are equinumerous (as sets of points).

\wbvfill

\item Determine a formula for the bijection from $(-1, 1)$ to the line $y = 1$
determined by vertical projection onto the upper half of the unit circle,
followed by projection from the point $(0, 0)$.

\wbvfill

\workbookpagebreak

\item  It is possible to generalize the argument that shows a line segment is
equivalent to a line to higher dimensions.   In two dimensions we would
show that the unit disk (the interior of the unit circle) is equinumerous
with the entire plane $\Reals \times \Reals$.   In three dimensions we would show that
the unit ball (the interior of the unit sphere) is equinumerous with the
entire space $\Reals^3 = \Reals \times \Reals \times \Reals$.  Here we 
would like you to prove the two-dimensional case.

Gnomonic projection is a style of map rendering in which a portion of a
sphere is projected onto a plane that is tangent to the sphere.  The 
sphere's center is used as the point to project from.  Combine 
vertical projection from the unit disk
in the x--y plane to the upper half of the unit sphere $x^2 + y^2 + z^2 = 1$,
with gnomonic projection from the unit sphere to the plane z = 1, to
deduce a bijection between the unit disk and the (infinite) plane.

\wbvfill

\end{enumerate}
 
%% Emacs customization
%% 
%% Local Variables: ***
%% TeX-master: "GIAM-hw.tex" ***
%% comment-column:0 ***
%% comment-start: "%% "  ***
%% comment-end:"***" ***
%% End: ***

