\documentclass{amsart}
%\usepackage{}
\renewcommand{\baselinestretch}{1.5}
\addtolength{\textwidth}{.2in}
\newcommand{\versionNum}{$3.2$\ }

\newboolean{InTextBook}
\setboolean{InTextBook}{false}
\newboolean{InWorkBook}
\setboolean{InWorkBook}{false}
\newboolean{InHints}
\setboolean{InHints}{false}

%When this boolean is true (beginning in Section 5.1) we will use the convention
% that $0 \in \Naturals$.  If it is false we will continue to count $1$ as the smallest
%natural number (thus making Giuseppe Peano spin in his grave...)
 
\newboolean{ZeroInNaturals}

%This boolean is used to distinguish the version where we use $\sim$ rather than $\lnot$

\newboolean{LNotIsSim}

%The values of the last two booleans are set in ``switches.tex''

\setboolean{ZeroInNaturals}{true}
\setboolean{LNotIsSim}{false}


\let\savedlnot\lnot
\ifthenelse{\boolean{LNotIsSim}}{\renewcommand{\lnot}{\sim} }{}

%This command puts different amounts of space depending on whether we are
% in the text, the workbook or the hints & solutions manual. 
\newcommand{\twsvspace}[3]{%
 \ifthenelse{\boolean{InTextBook} }{\vspace{#1}}{%
  \ifthenelse{\boolean{InWorkBook} }{\vspace{#2}}{%
   \ifthenelse{\boolean{InHints} }{\vspace{#3}}{} %
   }%
  }%
 }


\newcommand{\wbvfill}{\ifthenelse{\boolean{InWorkBook}}{\vfill}{}}
\newcommand{\wbitemsep}{\ifthenelse{\boolean{InWorkBook} }{\rule[-24pt]{0pt}{60pt}}{}}
\newcommand{\textbookpagebreak}{\ifthenelse{\boolean{InTextBook}}{\newpage}{}}
\newcommand{\workbookpagebreak}{\ifthenelse{\boolean{InWorkBook}}{\newpage}{}}
\newcommand{\hintspagebreak}{\ifthenelse{\boolean{InHints}}{\newpage}{}}

\newcommand{\hint}[1]{\ifthenelse{\boolean{InHints}}{ {\par \hspace{12pt} \color[rgb]{0,0,1} #1 } }{}}
\newcommand{\inlinehint}[1]{\ifthenelse{\boolean{InHints}}{ { \color[rgb]{0,0,1} #1 } }{}}

\newlength{\cwidth}
\newcommand{\cents}{\settowidth{\cwidth}{c}%
\divide\cwidth by2
\advance\cwidth by-.1pt
c\kern-\cwidth
\vrule width .1pt depth.2ex height1.2ex
\kern\cwidth}

\newcommand{\sageprompt}{ {\tt sage$>$} }
\newcommand{\tab}{\rule{20pt}{0pt}}
\newcommand{\blnk}{\rule{1.5pt}{0pt}\rule{.4pt}{1.2pt}\rule{9pt}{.4pt}\rule{.4pt}{1.2pt}\rule{1.5pt}{0pt}}
\newcommand{\suchthat}{\; \rule[-3pt]{.5pt}{13pt} \;}
\newcommand{\divides}{\!\mid\!}
\newcommand{\tdiv}{\; \mbox{div} \;}
\newcommand{\restrict}[2]{#1 \,\rule[-4pt]{.25pt}{14pt}_{\,#2}}
\newcommand{\lcm}[2]{\mbox{lcm} (#1, #2)}
\renewcommand{\gcd}[2]{\mbox{gcd} (#1, #2)}
\newcommand{\Naturals}{{\mathbb N}}
\newcommand{\Integers}{{\mathbb Z}}
\newcommand{\Znoneg}{{\mathbb Z}^{\mbox{\tiny noneg}}}
\ifthenelse{\boolean{ZeroInNaturals}}{%
  \newcommand{\Zplus}{{\mathbb Z}^+} }{%
  \newcommand{\Zplus}{{\mathbb N}} }
\newcommand{\Enoneg}{{\mathbb E}^{\mbox{\tiny noneg}}}
\newcommand{\Qnoneg}{{\mathbb Q}^{\mbox{\tiny noneg}}}
\newcommand{\Rnoneg}{{\mathbb R}^{\mbox{\tiny noneg}}}
\newcommand{\Rationals}{{\mathbb Q}}
\newcommand{\Reals}{{\mathbb R}}
\newcommand{\Complexes}{{\mathbb C}}
%\newcommand{\F2}{{\mathbb F}_{2}}
\newcommand{\relQ}{\mbox{\textsf Q}}
\newcommand{\relR}{\mbox{\textsf R}}
\newcommand{\nrelR}{\mbox{\raisebox{1pt}{$\not$}\rule{1pt}{0pt}{\textsf R}}}
\newcommand{\relS}{\mbox{\textsf S}}
\newcommand{\relA}{\mbox{\textsf A}}
\newcommand{\Dom}[1]{\mbox{Dom}(#1)}
\newcommand{\Cod}[1]{\mbox{Cod}(#1)}
\newcommand{\Rng}[1]{\mbox{Rng}(#1)}

\DeclareMathOperator\caret{\raisebox{1ex}{$\scriptstyle\wedge$}}

\newtheorem*{defi}{Definition}
\newtheorem*{exer}{Exercise}
\newtheorem{thm}{Theorem}[section]
\newtheorem*{thm*}{Theorem}
\newtheorem{lem}[thm]{Lemma}
\newtheorem*{lem*}{Lemma}
\newtheorem{cor}{Corollary}
\newtheorem{conj}{Conjecture}

\renewenvironment{proof}%
{\begin{quote} \emph{Proof:} }%
{\rule{0pt}{0pt} \newline \rule{0pt}{15pt} \hfill Q.E.D. \end{quote}}


\begin{document}
\thispagestyle{empty}

\centerline{\Large Activity 9 -- Introduction to Proof}
\centerline{\large conditionals}

\bigskip
\Large


\begin{enumerate}

\item Let $P = $ ``the number is even,'' and let $Q =$ ``the number is divisible by 4.''  Which conditional is true?  $P \implies Q$ or $Q \implies P$ ?  

\vfill

\item Express ``If $p$ is a prime, then $p$ is odd'' as a disjunction.

\vfill

\item Express the negation of ``If $x$ is $0 \mod 12$, then $x$ is odd'' as a conjunction.

\vfill

\item What conditional sentence is equivalent to $A \lor B$?

\vfill

\item What conditional sentence is the negation of $A \land B$?

\vfill

\newpage

\item You will sometimes hear the terms {\em necessary} and {\em sufficient} used when discussing the statements that are part of a conditional.  

A necessary condition is one that must be true in order for another statement to be true.  For example, it is necessary that I have detergent so that I can do the laundry.

A sufficient condition for another is one that guarantees the other's truth. For example, seeing the clean, folded clothes is sufficient to know that I've done the laundry.

Consider the lawnmower example.  Which is a sufficient condition to know that the lawnmower's engine is running, and which is a neccesary condition?

\begin{itemize}
\item There is gas in the tank
\item I can hear the motor.
\end{itemize}

\vspace{.5in}

\item Suppose $P \implies Q$ is true.  Which of the following are correct?

\begin{itemize}
\item $P$ is a necessary condition for $Q$.
\item $Q$ is a necessary condition for $P$
\item $P$ is a sufficient condition for $Q$.
\item $Q$ is a sufficient condition for $P$
\end{itemize}

\vfill

\newpage

\item Suppose $X$ is the statement ``Major Tom is alive.''  (And let's further suppose that we're sure who is meant by ``Major Tom'' so it's not ambiguous.)  Here are two additional statements about Major Tom:

\begin{align*}
 & A = \mbox{``The atmosphere in Major Tom's capsule contains oxygen''} \\
 & B = \mbox{``I can hear Major Tom's heartbeat on the monitor''} \\
\end{align*}

Which is the necessary condition and which is the sufficient condition to know that $X$ is true?

\vfill

\item If $S$ is a sufficient condition for $X$, and $N$ is a necessary condition for $X$, which is correct?
\[ S \implies X \implies N \]
\[ N \implies X \implies S \]

If $P$ is necessary {\em and} sufficient for $X$ what symbol can we place between them?

\vfill

\newpage

\item If a natural number $n$ is even, by definition we know that $n = 2k$ for some integer $k$.  Then, by simple algebra it follows that $n^2 = (2k)^2 = 4k^2 = 2(2k^2)$ and since $(2k^2)$ is clearly an integer, we get that $n^2$ is even.

That was a more-or-less accurate proof of the conditional 

\[ (n \; \mbox{is even}) \implies (n^2 \; \mbox{is even}). \]

What are the converse, inverse and contrapositive of this conditional?  (Hint: Be sure to write, e.g., ``$n$ is odd'' rather than ``it's not the case that $n$ is even'' when negating the parts of this conditional.)

\vfill



\end{enumerate}


\end{document}
