\documentclass{amsart}

\usepackage{hyperref}
\usepackage{color}
\usepackage{graphicx}

\hypersetup{colorlinks=true}

\addtolength{\topmargin}{-.5 in}
\addtolength{\textheight}{.5 in}
\addtolength{\oddsidemargin}{-.5 in}
\addtolength{\evensidemargin}{-.5 in}
\addtolength{\textwidth}{1 in}

\newcommand{\arcsec}{ {\rm arcsec}}
\newcommand{\arccsc}{ {\rm arccsc}}
\newcommand{\arccot}{ {\rm arccot}}
\newcommand{\diff}{\frac{\mbox{d}}{\mbox{d}x}\,}
\newcommand{\dx}{\;\mbox{d}x}
\newcommand{\dy}{\;\mbox{d}y}
\newcommand{\dz}{\;\mbox{d}z}

\newcommand{\dr}{\;\mbox{d}r}
\newcommand{\ds}{\;\mbox{d}s}
\newcommand{\dt}{\;\mbox{d}t}

\newcommand{\dtheta}{\;\mbox{d}\theta}
\newcommand{\dphi}{\;\mbox{d}\phi}
\newcommand{\drho}{\;\mbox{d}\rho}
\newcommand{\dA}{\;\mbox{d}A}
\newcommand{\dV}{\;\mbox{d}V}

\newcommand{\Integers}{ {\mathbb Z} }
\newcommand{\Rationals}{ {\mathbb Q} }
\newcommand{\Reals}{ {\mathbb R} }

\newcommand{\vs}{\rule[-24pt]{0pt}{60pt}}

\pagestyle{empty}

\begin{document}
\thispagestyle{empty}

\centerline{\Large \bf Activity -- MAT 252 -- Spring 2020}
\bigskip
\centerline{\large \bf November 23, 2020}

\Large




\begin{enumerate}

\item Suppose that the vector field $\vec{F} \, = \, \langle f, g \rangle$ satisfies $g_x \, = \, f_y$.  What value does this imply for all line integrals $\int \vec{F} \mbox{d}\vec{r}$ where the integral is taken over a simple closed curve?  Explain why this fact tells us that such line integrals are path independent.

\vfill

\newpage

\item A {\bf stream function} for a vector field $\vec{F} \, = \, \langle f, g \rangle$ is a function of two variables whose level curves are so-called {\bf streamlines} -- curves that are aligned with the flow of the vector field.

In general a stream function $\psi(x,y)$ satisfies $\psi_x = -g$ and $\psi_y = f$.  

Find a stream function for the vector field $\vec{F} \, = \, \langle x^2, -2xy \rangle$.

\vfill

\item When a vector field is both conservative and source free, both the potential functions and stream functions satisfy an important partial differential equation known as Laplace's equation:

\[ \phi_{xx} + \phi_{yy} = 0 \quad \mbox{and} \quad  \psi_{xx} + \psi_{yy} = 0.  \]

Verify Laplace's equation for the potential and stream functions of the vector field $\vec{F} \, = \, \langle y, x \rangle$.

\vfill

\newpage

\item Use the 3-dimensional $\nabla$ operator and a dot product to find the divergence of $\vec{F} \, = \, \langle y, -x, z^2 \rangle$.

\vfill

\item Use the 3-dimensional $\nabla$ operator and a cross product to find the curl of $\vec{F} \, = \, \langle xy, -x^2, 1-z \rangle$.

\vfill

\newpage 

\item Verify that the divergence of the curl of $\vec{F} \, = \, \langle x^2-y^2, xy, z  \rangle$ is 0.

\vfill

\item Choose a random function of 3 variables and calculate it's gradient.  We can regard your chosen function as the scalar potential of the vector field you just computed.  Exchange vector fields with a team mate and each of you calculate the curl of the other's field.   Explain what happened.

\vfill

\newpage

\item The Laplacian operator in 3-d is $ \nabla^2(f) \; = \; f_{xx} + f_{yy} + f_{zz}$.  Functions that have a zero Laplacian are special.  What is the Laplacian of 

\[ f(x,y,z) \; = \cos{xy} + e^z. \]

\vfill

\item Find a non-constant function of 3 variables whose Laplacian is zero. 

\vfill

\end{enumerate}

\end{document}

