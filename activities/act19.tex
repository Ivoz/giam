\documentclass{amsart}
\usepackage{amssymb}
\renewcommand{\baselinestretch}{1.5}
\addtolength{\textwidth}{.2in}
\addtolength{\topmargin}{-.5in}
\addtolength{\textheight}{1in}

\usepackage{ifthen}

\newcommand{\versionNum}{$3.2$\ }

\newboolean{InTextBook}
\setboolean{InTextBook}{false}
\newboolean{InWorkBook}
\setboolean{InWorkBook}{false}
\newboolean{InHints}
\setboolean{InHints}{false}

%When this boolean is true (beginning in Section 5.1) we will use the convention
% that $0 \in \Naturals$.  If it is false we will continue to count $1$ as the smallest
%natural number (thus making Giuseppe Peano spin in his grave...)
 
\newboolean{ZeroInNaturals}

%This boolean is used to distinguish the version where we use $\sim$ rather than $\lnot$

\newboolean{LNotIsSim}

%The values of the last two booleans are set in ``switches.tex''

\setboolean{ZeroInNaturals}{true}
\setboolean{LNotIsSim}{false}


\let\savedlnot\lnot
\ifthenelse{\boolean{LNotIsSim}}{\renewcommand{\lnot}{\sim} }{}

%This command puts different amounts of space depending on whether we are
% in the text, the workbook or the hints & solutions manual. 
\newcommand{\twsvspace}[3]{%
 \ifthenelse{\boolean{InTextBook} }{\vspace{#1}}{%
  \ifthenelse{\boolean{InWorkBook} }{\vspace{#2}}{%
   \ifthenelse{\boolean{InHints} }{\vspace{#3}}{} %
   }%
  }%
 }


\newcommand{\wbvfill}{\ifthenelse{\boolean{InWorkBook}}{\vfill}{}}
\newcommand{\wbitemsep}{\ifthenelse{\boolean{InWorkBook} }{\rule[-24pt]{0pt}{60pt}}{}}
\newcommand{\textbookpagebreak}{\ifthenelse{\boolean{InTextBook}}{\newpage}{}}
\newcommand{\workbookpagebreak}{\ifthenelse{\boolean{InWorkBook}}{\newpage}{}}
\newcommand{\hintspagebreak}{\ifthenelse{\boolean{InHints}}{\newpage}{}}

\newcommand{\hint}[1]{\ifthenelse{\boolean{InHints}}{ {\par \hspace{12pt} \color[rgb]{0,0,1} #1 } }{}}
\newcommand{\inlinehint}[1]{\ifthenelse{\boolean{InHints}}{ { \color[rgb]{0,0,1} #1 } }{}}

\newlength{\cwidth}
\newcommand{\cents}{\settowidth{\cwidth}{c}%
\divide\cwidth by2
\advance\cwidth by-.1pt
c\kern-\cwidth
\vrule width .1pt depth.2ex height1.2ex
\kern\cwidth}

\newcommand{\sageprompt}{ {\tt sage$>$} }
\newcommand{\tab}{\rule{20pt}{0pt}}
\newcommand{\blnk}{\rule{1.5pt}{0pt}\rule{.4pt}{1.2pt}\rule{9pt}{.4pt}\rule{.4pt}{1.2pt}\rule{1.5pt}{0pt}}
\newcommand{\suchthat}{\; \rule[-3pt]{.5pt}{13pt} \;}
\newcommand{\divides}{\!\mid\!}
\newcommand{\tdiv}{\; \mbox{div} \;}
\newcommand{\restrict}[2]{#1 \,\rule[-4pt]{.25pt}{14pt}_{\,#2}}
\newcommand{\lcm}[2]{\mbox{lcm} (#1, #2)}
\renewcommand{\gcd}[2]{\mbox{gcd} (#1, #2)}
\newcommand{\Naturals}{{\mathbb N}}
\newcommand{\Integers}{{\mathbb Z}}
\newcommand{\Znoneg}{{\mathbb Z}^{\mbox{\tiny noneg}}}
\ifthenelse{\boolean{ZeroInNaturals}}{%
  \newcommand{\Zplus}{{\mathbb Z}^+} }{%
  \newcommand{\Zplus}{{\mathbb N}} }
\newcommand{\Enoneg}{{\mathbb E}^{\mbox{\tiny noneg}}}
\newcommand{\Qnoneg}{{\mathbb Q}^{\mbox{\tiny noneg}}}
\newcommand{\Rnoneg}{{\mathbb R}^{\mbox{\tiny noneg}}}
\newcommand{\Rationals}{{\mathbb Q}}
\newcommand{\Reals}{{\mathbb R}}
\newcommand{\Complexes}{{\mathbb C}}
%\newcommand{\F2}{{\mathbb F}_{2}}
\newcommand{\relQ}{\mbox{\textsf Q}}
\newcommand{\relR}{\mbox{\textsf R}}
\newcommand{\nrelR}{\mbox{\raisebox{1pt}{$\not$}\rule{1pt}{0pt}{\textsf R}}}
\newcommand{\relS}{\mbox{\textsf S}}
\newcommand{\relA}{\mbox{\textsf A}}
\newcommand{\Dom}[1]{\mbox{Dom}(#1)}
\newcommand{\Cod}[1]{\mbox{Cod}(#1)}
\newcommand{\Rng}[1]{\mbox{Rng}(#1)}

\DeclareMathOperator\caret{\raisebox{1ex}{$\scriptstyle\wedge$}}

\newtheorem*{defi}{Definition}
\newtheorem*{exer}{Exercise}
\newtheorem{thm}{Theorem}[section]
\newtheorem*{thm*}{Theorem}
\newtheorem{lem}[thm]{Lemma}
\newtheorem*{lem*}{Lemma}
\newtheorem{cor}{Corollary}
\newtheorem{conj}{Conjecture}

\renewenvironment{proof}%
{\begin{quote} \emph{Proof:} }%
{\rule{0pt}{0pt} \newline \rule{0pt}{15pt} \hfill Q.E.D. \end{quote}}


\addtolength{\abovedisplayskip}{0pt}
\addtolength{\belowdisplayskip}{24pt}
\addtolength{\abovedisplayshortskip}{0pt}
\addtolength{\belowdisplayshortskip}{44pt}


\begin{document}
\thispagestyle{empty}

\centerline{\Large Activity 19 -- Introduction to Proof}
\centerline{\large proofs and disproofs of existentials}

\bigskip
\Large


\begin{enumerate}

\item Is there a Fibonacci number that is a perfect cube?

\vfill

\item Partition numbers count the number of ways we can write a number $n$ as a sum (we don't count different orderings of the same sum).   We denote the partition numbers by $p(n)$.  By convention, $p(0)$ is defined to be 1 ($p(1)$ is also 1).  As a more interesting example, $p(4) = 5$ since there are 5 distinct ways to write 4 as a sum: $4$, $3+1$, $2+2$, $2+1+1$ and $1+1+1+1$.  The partition numbers are A000041 in the OEIS.

After the first two 1's the values of $p(n)$ are all prime (at first).  Prove that there is a non-prime partition number, $p(n)$ with $n \geq 2$.

\vfill

\item Show that there is a 4 digit vampire number.

\vfill

\newpage

\centerline{\bf Review problems for chapter 3.}

\item Use proof by exhaustion to show that there are no 2 digit vampire numbers.

\vfill

\item Prove
\[ \forall a, b, c \in \Integers, \; ( a \mid b \, \land \, b \mid c ) \implies a \mid c. \]

\vfill

\newpage

\item Prove 
\[ \forall a, b \in \Integers, \; ( a \mid b \, \land \, b \mid a ) \implies a=b. \]

\vfill 

\item Prove 
\[ \forall x \in \Reals, \; x^5 < x^4 \; \implies 8 > 5x+3. \]

\vfill

\newpage

\item Prove that for all integers $a, b,$ and $c$, if $a \mid b$ and $a \mid (b+c)$, then
$a|c$.

\vfill

\item Show that $\binom{n}{k} \cdot \binom{k}{r} \; = \; \binom{n}{r} \cdot \binom{n-r}{k-r}$ (for all integers $r$, $k$ and $n$ with $r \leq k \leq n$). 

\vfill

\newpage

\item Referring to the partition numbers defined in problem 2, we say a partition has ``odd parts'' if all the numbers in the sum are odd.  We say a partition has ``distinct parts'' if all the numbers in the sum are different from one another.  It is a general fact that the number of partitions of $n$ into odd parts is equal to the number of partitions of $n$ into distinct parts. 
(Frankly, that is really weird\textellipsis)

Show exhaustively that the number of partitions of 7 into odd parts is equal to the number of partitions of 7 into distinct parts.

\vfill

\end{enumerate}

\end{document}
