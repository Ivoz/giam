\begin{enumerate}

\item Trace through the division algorithm with inputs $n=27$ and
  $d=5$, each time an assignment statement is encountered write it
  out.  How many assignments are involved in this particular
  computation?
\hint{\par
r=27 \newline
q=0  \newline
r=27-5=22  \newline
q=0+1=1  \newline
r=22-5=17  \newline
q=1+1=2  \newline
r=17-5=12  \newline
q=2+1=3  \newline
r=12-5=7  \newline
q=3+1=4  \newline
r=7-5=2  \newline
q=4+1=5  \newline
return r is 2 and q is 5.
}

\wbvfill

\item Find the gcd's and lcm's of the following pairs of numbers.
\medskip

\centerline{
\begin{tabular}{|c|c|c|c|} \hline
\rule[-3pt]{0pt}{18pt} $a$ & $b$ & $\gcd{a}{b}$ & $\lcm{a}{b}$ \\ \hline
\rule[-3pt]{0pt}{18pt} 110 & 273 & & \\ \hline
\rule[-3pt]{0pt}{18pt}105 & 42 & & \\ \hline
\rule[-3pt]{0pt}{18pt}168 & 189 & & \\ \hline
\end{tabular}
}

\hint{For such small numbers you can just find their prime factorizations and use that, although it might be useful to practice your understanding of the Euclidean algorithm by tracing through it to find the gcd's and then using the formula
\[ \lcm (a,b) = \frac{ab}{\gcd (a,b).} \]
}

\workbookpagebreak

\item Formulate a description of the gcd of two numbers in terms of
  their prime factorizations in the general case (when the
  factorizations may include powers of the primes involved).

\wbvfill

\hint{Suppose that one number's prime factorization contains $p^e$ and the other
contains $p^f$, where $e < f$. What power of $p$ will divide both, $p^e$ or $p^f$ ?}

\item Trace through the Euclidean algorithm with inputs $a=3731$ and
  $b=2730$, each time the assignment statement that calls the division
  algorithm is encountered write out the expression $a=qb+r$.   (With the
  actual values involved !) 

\wbvfill

\hint{The quotients you obtain should alternate between 1 and 2.}

\end{enumerate}
