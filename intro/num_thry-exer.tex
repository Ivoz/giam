\begin{enumerate}

\item An integer $n$ is \index{doubly-even} \emph{doubly-even} 
if it is even, and the integer $m$ guaranteed to exist because 
$n$ is even is itself even.  Is 0 doubly-even?  What are the 
first 3 positive, doubly-even integers?

\wbvfill

\hint{Answers: yes, 0,4 and 8.}

\item Dividing an integer by two has an interesting interpretation
when using binary notation: simply shift the digits to the right.
Thus, $22 = 10110_2$ when divided by two gives $1011_2$ which is
$8+2+1=11$.  How can you recognize a doubly-even integer from
its binary representation?

\wbvfill

\hint{Even numbers have a zero in their units place. What digit must also be zero in a doubly-even number's binary representation?}

\item The \index{octal representation} \emph{octal} representation 
of an integer uses powers of 8 in place notation.  The digits of an 
octal number run from 0 to 7, one never sees 8's or 9's.  How would 
you represent 8 and 9 as octal numbers?  What octal number comes 
immediately after $777_8$?  What (decimal) number is $777_8$?

\wbvfill

\workbookpagebreak

\hint{Eight is $10_8$, nine is $11_8$. The point of asking questions about $777$, is that (in octal) $7$ is the digit that is analogous to $9$ in base-$10$. Thus $777_8$ is something like $999_{10}$ in that the number following both of them is written $1000$ (although $1000_8$ and $1000_{10}$ are certainly not equal!)}

\hintspagebreak

\item One method of converting from decimal to some other base is
called \index{repeated division algorithm} \emph{repeated division}.  
One divides the number by the base
and records the remainder -- one then divides the quotient obtained
by the base and records the remainder.  Continue dividing the 
successive quotients by the base until the quotient is smaller than
the base.  Convert 3267 to base-7 using repeated division.  Check 
your answer by using the meaning of base-7 place notation.  (For
example $54321_7$ means $5\cdot7^4 + 4\cdot7^3 + 3 \cdot7^2 +
2\cdot7^1 + 1\cdot7^0$.)

\wbvfill

\hint{It is helpful to write something of the form $n = qd+r$ at each stage. The first two stages should look like

\[ 3267 \; = \; 466 \cdot 7 + 5 \]

\[ 466 \; = \; 66 \cdot 7 + 4 \]

you do the rest\ldots
}

\item State a theorem about the octal representation of even numbers.

\wbvfill

\hint{One possibility is to mimic the result for base-10 that an even number always ends in 0,2,4,6 or 8.}

\item In hexadecimal (base-16) notation one needs 16 ``digits,'' the
  ordinary digits are used for 0 through 9, and the letters A through
  F are used to give single symbols for 10 through 15.  The first  32
  natural number in hexadecimal are:
  1,2,3,4,5,6,7,8,9,A,B,C,D,E,F,10,11,12,13,14,15,16,\newline 17,18,19,1A,
  1B,1C,1D,1E,1F,20. 

  Write the next 10 hexadecimal numbers after $AB$.

  Write the next 10 hexadecimal numbers after $FA$.

\hint{As a check, the tenth number after AB is B5.\newline
The tenth hexadecimal number after FA is 104.}

\wbvfill

\workbookpagebreak

\item For conversion between the three bases used most often in 
Computer Science we can take binary as the ``standard'' base and 
convert using a table look-up.  Each octal digit will correspond 
to a binary triple, and each hexadecimal digit will correspond to 
a 4-tuple of binary numbers.  Complete the following tables.  
(As a check, the 4-tuple next to $A$ in the table for
hexadecimal should be 1010 -- which is nice since $A$ 
is really 10 so if you read that as ``ten-ten'' it is a good 
aid to memory.)

\begin{center}
\begin{tabular}{ccc}
\begin{tabular}{|c|c|} \hline
octal & binary \\ \hline \hline
\rule{0pt}{14pt} 0 & 000 \\ \hline
\rule{0pt}{14pt} 1 & 001 \\ \hline
\rule{0pt}{14pt} 2 & \\ \hline
\rule{0pt}{14pt} 3 & \\ \hline
\rule{0pt}{14pt} 4 & \\ \hline
\rule{0pt}{14pt} 5 & \\ \hline
\rule{0pt}{14pt} 6 & \\ \hline
\rule{0pt}{14pt} 7 & \\ \hline
\end{tabular}
 & \rule{72pt}{0pt} &
\begin{tabular}{|c|c|} \hline
hexadecimal & binary \\ \hline \hline
\rule{0pt}{14pt} 0 & 0000 \\ \hline
\rule{0pt}{14pt} 1 & 0001 \\ \hline
\rule{0pt}{14pt} 2 & 0010 \\ \hline
\rule{0pt}{14pt} 3 & \\ \hline
\rule{0pt}{14pt} 4 & \\ \hline
\rule{0pt}{14pt} 5 & \\ \hline
\rule{0pt}{14pt} 6 & \\ \hline
\rule{0pt}{14pt} 7 & \\ \hline
\rule{0pt}{14pt} 8 & \\ \hline
\rule{0pt}{14pt} 9 & \\ \hline
\rule{0pt}{14pt} A & \\ \hline
\rule{0pt}{14pt} B & \\ \hline
\rule{0pt}{14pt} C & \\ \hline
\rule{0pt}{14pt} D & \\ \hline
\rule{0pt}{14pt} E & \\ \hline
\rule{0pt}{14pt} F & \\ \hline
\end{tabular}
\end{tabular}
\end{center}
 
\hint{

\vfill

This is just counting in binary. Remember the sanity check that the hexadecimal digit A is represented by 1010 in binary.  ($10_{10} \; = \; A_{16} \; = \; 1010_{2}$)

\vfill

}

\hintspagebreak
\workbookpagebreak
\textbookpagebreak

\item Use the tables from the previous problem to make the following conversions.

\begin{enumerate}
\item Convert $757_8$ to binary.
\item Convert $1007_8$ to hexadecimal.
\item Convert $100101010110_2$ to octal.
\item Convert $1111101000110101_2$ to hexadecimal.
\item Convert $FEED_{16}$ to binary.
\item Convert $FFFFFF_{16}$ to octal.
\end{enumerate}

\hint{Answers for the first three:
\[  757_8 = 111 101 111_2 \]
\[ 1007_8 = 001 000 000 111_2 = 0010 0000 0111_2 = 207_{16} \]
\[ 100 101 010 110_2 = 4526_8 \]
}

\item Try the following conversions between various number systems:

\begin{enumerate}
\item Convert $30$ (base 10) to binary.
\item Convert $69$ (base 10) to base 5.
\item Convert $1222_3$ to binary.
\item Convert $1234_7$ to base 10.
\item Convert $EEED_{15}$ to base 12. (Use $\{1, 2, 3 \ldots 9, d, e\}$ as the digits in base 12.)
\item Convert $678_{9}$ to hexadecimal.
\end{enumerate}

\item It is a well known fact that if a number is divisible by 3, then 3
  divides the sum of the (decimal) digits of that number.  Is this
  result true in base 7?  Do you think this result is true in {\em
  any} base? 
 
 \wbvfill
 
\hint{Might this effect have something to do with 10 being just one bigger than 9 (a multiple of 3)?}

\item Suppose that 340 pounds of sand must be placed into bags having
  a 50 pound capacity.  Write an expression using either floor or
  ceiling notation for the number of bags required.

\wbvfill

\hint{Seven 50 pound bags would hold 350 pounds of sand. They'd also be able to handle 340 pounds!}

\item True or false? 

\[ \left\lfloor \frac{n}{d}\right\rfloor < \left\lceil \frac{n}{d}\right\rceil \]
 
\noindent for all integers $n$ and $d>0$. Support your claim.

\wbvfill

\hint{You have to try a bunch of examples.  You should try to make sure the examples
you try cover all the possibilities.  The pairs that provide counterexamples (i.e. show the statement is false in general) are relatively sparse, so be systematic.}

\workbookpagebreak

\item What is the value of $\lceil\pi\rceil^{2}-\lceil\pi^{2}\rceil$?

\wbvfill

\hint{ $\pi^2 = 9.8696$ }



\item Assuming the symbols $n$,$d$,$q$ and $r$ have meanings as in the
  quotient-remainder theorem (\ifthenelse{\boolean{InTextBook}}{Theorem~\ref{quo-rem} on page \pageref{quo-rem}}{see page 29 of GIAM}).  Write
  expressions for $q$ and $r$, in terms of $n$ and $d$ using floor
  and/or ceiling notation.

\wbvfill

\hint{I just can't bring myself to spoil this one for you, you really need to work this out on your own. }

\textbookpagebreak

\item Calculate the following quantities:

\begin{enumerate}
\item \wbitemsep $3 \mod 5$
\item \wbitemsep $37 \mod 7$
\item \wbitemsep $1000001 \mod 100000$
\item \wbitemsep $6 \tdiv 6$
\item \wbitemsep $7 \tdiv 6$
\item \wbitemsep $1000001 \tdiv 2$
\end{enumerate}

\hint{The even numbered ones are 2, 1, 500000.}

\hintspagebreak
\workbookpagebreak

\item Calculate the following binomial coefficients:

\begin{enumerate}
\item \wbitemsep $\binom{3}{0}$
\item \wbitemsep $\binom{7}{7}$
\item \wbitemsep $\binom{13}{5}$
\item \wbitemsep $\binom{13}{8}$
\item \wbitemsep $\binom{52}{7}$
\end{enumerate}

\hint{The even numbered ones are 1 and 1287. The TI-84 calculates binomial coefficients. The symbol used is {\tt nCr} (which is placed between the numbers -- i.e. it is an infix operator). You get {\tt nCr} as the 3rd item in the {\tt PRB} menu under {\tt MATH}. In sage the command is {\tt binomial(n,k)}.}

\item An ice cream shop sells the following flavors: chocolate, vanilla, 
strawberry, coffee, butter pecan, mint chocolate chip and raspberry.
How many different bowls of ice cream -- with three scoops -- can they make?  

\wbvfill

\hint{You're choosing three things out of a set of size seven\ldots}

\end{enumerate}
