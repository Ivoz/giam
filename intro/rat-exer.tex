\begin{enumerate}

\item \index{Rational approximation} Rational Approximation is 
a field of mathematics that has received much study.  The main idea 
is to find rational numbers that are very good approximations to
given irrationals.  For example, $22/7$ is a well-known rational 
approximation to $\pi$.  Find good rational approximations to 
$\sqrt{2}, \sqrt{3}, \sqrt{5}$ and $e$.

\vfill

\wbvfill

\hint{One approach is to truncate a decimal approximation and then rationalize. E.g. $\sqrt{2}$ is approximately 1.4142, so 14142/10000 isn't a bad approximator (although naturally 7071/5000 is better since it involves smaller numbers).}

\vfill

\item The theory of base-$n$ notation that we looked at in 
\ifthenelse{\boolean{InTextBook}}{sub-section~\ref{base-n}}{the sub-section on base-$n$} can be extended to deal with real and 
rational numbers by introducing a decimal point (which should 
probably be re-named in accordance with the base) and adding 
digits to the right of it.  For instance $1.1011$ is binary notation
for $1 \cdot 2^0 + 1 \cdot 2^{-1} + 0 \cdot 2^{-2} + 
1\cdot 2^{-3} + 1\cdot 2^{-4}$ or $\displaystyle 1 + \frac{1}{2} + 
\frac{1}{8} + \frac{1}{16} = 1 \frac{11}{16}$.

Consider the binary number $.1010010001000010000010000001\ldots$, 
is this number rational or irrational?  Why?

\vfill

\hint{Does the rule about rational numbers having terminating or repeating decimal representations carry over to other bases?

\vfill

}

\workbookpagebreak
\hintspagebreak

\item If a number $x$ is even, it's easy to show that its square $x^2$
is even.  The lemma that went unproved in this section asks us to
start with a square ($x^2$) that is even and deduce that the unsquared
number ($x$) is even.  Perform some numerical experimentation to
check whether this assertion is reasonable.  Can you give an argument
that would prove it?

\vfill

\hint{What if the lemma wasn't true? Can you work out what it would mean if we had a number x such that x2 was even but x itself was odd?}

\vfill

\item The proof that $\sqrt{2}$ is irrational can be generalized 
to show that $\sqrt{p}$ is irrational for every prime number $p$.
What statement would be equivalent to the lemma about the parity
of $x$ and $x^2$ in such a generalization?

\vfill

\hint{Hint: Saying ``x is even'' is the same thing as saying ``x is evenly divisible by 2.''  Replace the $2$ by $p$ and you're halfway there\ldots}

\vfill

\item Write a proof that $\sqrt{3}$ is irrational.

\vfill

\hint{You can mostly just copy the argument for $\sqrt{2}$.}

\vfill


\end{enumerate}