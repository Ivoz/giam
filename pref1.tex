\chapter*{To the student}
\chaptermark{To the student}

You are at the right place in your mathematical career to be
reading this book if you liked Trigonometry and Calculus,
were able to solve all the problems, but felt mildly annoyed
with the text when it put in these verbose, incomprehensible 
things called ``proofs.''   Those things probably bugged you
because a whole lot of verbiage (not to mention a sprinkling 
of epsilons and deltas) was wasted on showing that a thing 
was true, which was {\em obviously true!}  Your physical intuition
is sufficient to convince you that a statement like the 
Intermediate Value Theorem just {\em has} to be true -- how 
can a function move from one value at $a$ to a different value
at $b$ without passing through all the values in between?

Mathematicians discovered something fundamental hundreds of
years before other scientists -- physical intuition is worthless
in certain extreme situations.  Probably you've heard of some
of the odd behavior of particles in Quantum Mechanics or General
Relativity.  Physicists have learned, the hard way, not to trust
their intuitions.  At least, not until those intuitions have
been retrained to fit reality!  Go back to your Calculus textbook
and look up the Intermediate Value Theorem.  You'll probably
be surprised to find that it doesn't say anything about {\em all}
functions, only those that are continuous.  So what, you say,
aren't most functions continuous?  Actually, the number of functions
that aren't continuous represents an infinity so huge that it
outweighs the infinity of the real numbers! 

The point of this book is to help you with the transition from
doing math at an elementary level (which is concerned mostly with 
solving problems) to doing math at an advanced level (which is 
much more concerned with axiomatic systems and proving statements
within those systems).  

As you begin your study of advanced mathematics, we hope you will
keep the following themes in mind:

\begin{enumerate}
\item Mathematics is alive!  Math is not just something to be studied from
ancient tomes.  A mathematician must have a sense of playfulness.  One
needs to ``monkey around'' with numbers and other mathematical
structures, make discoveries and conjectures and uncover truths.
\item Math is not scary!  There is an incredibly terse and compact
language that is used in mathematics -- on first sight it looks
like hieroglyphics.  That language is actually easy to master, and once
mastered, the power that one gains by expressing ideas rigorously with
those symbols is truly astonishing.
\item Good proofs are everything!  No matter how important a fact one
discovers, if others don't become convinced of the truth of the
statement it does not become a part of the edifice of human knowledge.
It's been said that a proof is simply an argument that convinces.  In
mathematics, one ``convinces'' by using one of a handful of argument
forms and developing one's argument in a clear, step-by-step fashion.
Within those constraints there is actually quite a lot of room for 
individual style -- there is no one right way to write a proof. 
\item You have two cerebral hemispheres -- use them both!  In perhaps
no other field is the left/right-brain dichotomy more evident than in
math.  Some believe that mathematical thought, deductive reasoning, is
synonymous with left-brain function.  In truth, doing mathematics is
often a creative, organic, visual, right-brain sort of process --
however, in communicating one's results one must find that linear,
deductive, step-by-step, left-brain argument.  You must use your whole
mind to master advanced mathematics.
\end{enumerate}

Also, there are amusing quotations at the start of every chapter.

%% Emacs customization
%% 
%% Local Variables: ***
%% TeX-master: "GIAM.tex" ***
%% comment-column:0 ***
%% comment-start: "%% "  ***
%% comment-end:"***" ***
%% End: ***
 
