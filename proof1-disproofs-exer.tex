\begin{enumerate}
\item Find a polynomial that assumes only prime values for
a reasonably large range of inputs.
\item Find a counterexample to Conjecture~\ref{conj:prim} using 
only powers of 2.
\item The alternating sum of factorials provides an interesting
example of a sequence of integers.
\begin{center}
\[ 1! = 1 \]
\[ 2! - 1! = 1\]
\[ 3! - 2! + 1! = 5 \]
\[ 4! - 3! + 2! - 1! = 19 \]
et cetera
\end{center}

\noindent Are they all prime?  (After the first two 1's.)
\item It has been conjectured that whenever $p$ is prime, $2^p - 1$ is
also prime.  Find a minimal counterexample.

\item True or false:  The sum of any two irrational numbers is irrational.
Prove your answer.

\item True of false:  There are two irrational numbers whose sum is rational.
Prove your answer.

\item True or false: The product of any two irrational numbers is irrational.
Prove your answer.

\item True or false: There are two irrational numbers whose product is rational.
Prove your answer.

\item True or false:  Whenever an integer $n$ is a divisor of the square of an integer, $m^2$, it follows that $n$ is a divisor of $m$ as well.
(In symbols, $\forall n \in \Integers, \forall m \in \Integers, n \mid m^2 \; \implies \; n \mid m$.)
Prove your answer.

\item In an exercise in Section~\ref{sec:more} we proved that the quadratic 
equation $ax^2 + bx + c = 0$ has two solutions if $ac < 0$.  Find a counterexample which shows that this implication cannot be replaced with a biconditional.  

\end{enumerate}
