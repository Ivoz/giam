\begin{enumerate}
\item Determine a substitution rule -- a consistent way of replacing one digit
with another along the diagonal so that a diagonalization proof showing
that the interval (0, 1) is uncountable will work in decimal. Write up
the proof.

\item Can a diagonalization proof showing that the interval (0, 1) is uncountable
be made workable in base-3 (ternary) notation?

\item In the proof of Cantor's theorem we construct a set $S$ that cannot
be in the image of a presumed bijection from $A$ to ${\mathcal P}(A)$.  
Suppose $A = \{1, 2, 3\}$ and f determines the following correspondences: 
$1 \longleftrightarrow \emptyset$,
$2 \longleftrightarrow \{1, 3\}$ and $3 \longleftrightarrow \{1, 2, 3\}$. 
What is $S$?

\item An argument very similar to the one embodied in the proof of Cantor's
theorem is found in the Barber's paradox. This paradox was
originally introduced in the popular press in order to give laypeople an
understanding of Cantor's theorem and Russell's paradox. It sounds
somewhat sexist to modern ears. (For example, it is presumed without
comment that the Barber is male.)

\begin{quote}
In a small town there is a Barber who shaves those men (and
only those men) who do not shave themselves. Who shaves
the Barber?
\end{quote}

Explain the similarity to the proof of Cantor's theorem.

\item Cantor's theorem, applied to the set of all sets leads to an interesting
paradox. The power set of the set of all sets is a collection of sets, so
it must be contained in the set of all sets. Discuss the paradox and
determine a way of resolving it.

\item Verify that the final deduction in the proof of Cantor's theorem, 
``$(y \in S  \implies  y \notin S) \land  (y \notin S \implies y \in S)$,'' 
is truly a contradiction.

\end{enumerate}

%% Emacs customization
%% 
%% Local Variables: ***
%% TeX-master: "GIAM-hw.tex" ***
%% comment-column:0 ***
%% comment-start: "%% "  ***
%% comment-end:"***" ***
%% End: ***

