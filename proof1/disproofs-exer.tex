\begin{enumerate}
\item Find a polynomial that assumes only prime values for
a reasonably large range of inputs.

\hint{It sort of depends on what is meant by ``a reasonably large range of inputs.''  For example the polynomial $p(x) = 2x+1$ gives primes three times in a row (at $x=1,2$ and $3$).  See if you can do better than that.
}

\wbvfill

\item Find a counterexample to \ifthenelse{\boolean{InTextBook}}{Conjecture~\ref{conj:prim}}{the conjecture that $\forall a,b,c \in \Integers, a \divides bc \; \implies \; a \divides b \, \lor \, a \divides c$} using only powers of 2.

\hint{The intent of the problem is that you find three numbers, $a$, $b$ and $c$, that are all powers 
of $2$ and such that $a$ divides the product $bc$, but neither of the factors separately. For instance, 
if you pick $a=16$, then you would need to choose $b$ and $c$ so that $16$ doesn't divide evenly 
into them (they would need to be less than $16$\ldots) but so that their product {\em is} divisible by $16$.
}

\wbvfill

\workbookpagebreak

\item The alternating sum of factorials provides an interesting
example of a sequence of integers.
\begin{center}
\[ 1! = 1 \]
\[ 2! - 1! = 1\]
\[ 3! - 2! + 1! = 5 \]
\[ 4! - 3! + 2! - 1! = 19 \]
et cetera
\end{center}

\noindent Are they all prime?  (After the first two 1's.)

\hint{

Here's some Sage code that would test this conjecture:

{\tt 
n=1\newline
for i in [2..8]:\newline
\rule{18pt}{0pt}n = factorial(i) - n\newline
\rule{18pt}{0pt}show(factor(n))\newline
}

Of course it turns out that going out to $8$ isn't quite far enough\ldots

}

\wbvfill

\item It has been conjectured that whenever $p$ is prime, $2^p - 1$ is
also prime.  Find a minimal counterexample.

\hint{I would definitely seek help at your friendly neighborhood CAS.  In Sage 
you can loop over the first several prime numbers using the following syntax.

{\tt for p in [2,3,5,7,11,13]:}

\noindent If you want to automate that somewhat, there is a Sage function that returns a list
of all the primes in some range.  So the following does the same thing.

{\tt for p in primes(2,13):}
}

\wbvfill

\workbookpagebreak

\item True or false:  The sum of any two irrational numbers is irrational.
Prove your answer.

\hint{This statement and the next are negations of one another.  Your answers should reflect that.}

\wbvfill

\hintspagebreak

\item True or false:  There are two irrational numbers whose sum is rational.
Prove your answer.

\hint{If a number is irrational, isn't its negative also irrational?  That's actually a pretty huge hint.}

\wbvfill

\item True or false: The product of any two irrational numbers is irrational.
Prove your answer.

\hint{This one and the next are negations too. Aren't they?}

\wbvfill

\item True or false: There are two irrational numbers whose product is rational.
Prove your answer.

\hint{The two numbers {\em could} be equal couldn't they?}

\wbvfill

\workbookpagebreak

\item True or false:  Whenever an integer $n$ is a divisor of the square of an integer, $m^2$, it follows that $n$ is a divisor of $m$ as well.
(In symbols, $\forall n \in \Integers, \forall m \in \Integers, n \mid m^2 \; \implies \; n \mid m$.)
Prove your answer.

\hint{Hint: List all of the divisors of $36 = (2\cdot 3)^2$.  See if any of them are bigger than $6$.}

\wbvfill



\item In an exercise in Section~\ref{sec:more} we proved that the quadratic 
equation $ax^2 + bx + c = 0$ has two solutions if $ac < 0$.  Find a counterexample which shows that this implication cannot be replaced with a biconditional.  

\hint{We'd want $ac$ to be positive (so $a$ and $c$ have the same sign) but nevertheless have $b^2-4ac > 0$.  It seems that if we make $b$ sufficiently large that could happen.}

\wbvfill

\end{enumerate}
