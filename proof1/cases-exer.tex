\begin{enumerate}
\item Prove that if $n$ is an odd number then $n^4 \pmod{16} = 1$.

\hint{

While one could perform fairly complicated arithmetic, expanding expression like
$(16k+13)^4$ and then regrouping to put it in the form $16q+1$ (and one would need 
to do that work for each of the odd remainders modulo $16$),  that would be missing out
on the true power of modular notation.  In a ``$\pmod{16}$'' calculation one can simply ignore
summands like $16k$ because they are $0 \pmod{16}$.  Thus, for example,

  \[ (16k+7)^4 \pmod{16} \; = \; 7^4 \pmod{16} \; = \; 2401 \pmod{16}  \; = \; 1. \]
  
So, essentially one just needs to compute the $4$th powers of $1, 3, 5, 7, 9, 11, 13$  and $15$, and
then reduce them modulo 16.  An even greater economy is possible if one notes that (modulo 16) many
of those cases are negatives of one another -- so their $4$th powers are equal.
}

\wbvfill
     
\item Prove that every prime number other than 2 and 3 has the form
$6q+1$ or $6q+5$ for some integer $q$.  (Hint: this problem involves
thinking about cases as well as contrapositives.)

\hint{It is probably obvious that the "cases" will be the possible remainders mod 6. Numbers of the form 6q+0 will be multiples of 6, so clearly not prime. The other forms that need to be eliminated are 6q+2, 6q+3, and 6q+4.
}

\wbvfill

\workbookpagebreak

\item Show that the sum of any three consecutive integers is divisible
by 3.

\hint{Write the sum as $n + (n+1) + (n+2)$.}

\wbvfill

\item There is a graph known as $K_4$ that has $4$ nodes and there is an edge between every pair of nodes.
The pebbling number of $K_4$ has to be at least $4$ since it would be possible to put one pebble on each of
$3$ nodes and not be able to reach the remaining node using pebbling moves.  Show that the pebbling number of $K_4$ is actually $4$.

\hint{If there are two pebbles on any node we will be able to reach all the other nodes using pebbling moves
(since every pair of nodes is connected).}

\wbvfill

\workbookpagebreak

\item Find the pebbling number of a graph whose nodes are the corners and 
whose edges are the, uhmm, edges of a cube.

\hint{It should be clear that the pebbling number is at least $8$ -- $7$ pebbles could be distributed, 
one to a node, and the $8$th node would be unreachable.  It will be easier to play around with this if
you figure out how to draw the cube graph ``flattened-out'' in the plane.}

\wbvfill

\item A \index{vampire number}\emph{vampire number} is a $2n$ digit number $v$ that factors as $v=xy$
where $x$ and $y$ are $n$ digit numbers and the digits of $v$ are the 
union of the digits in $x$ and $y$ in some order.  The numbers $x$ and $y$
are known as the ``fangs'' of $v$.  To eliminate trivial
cases, pairs of trailing zeros are disallowed.  

Show that there are no 2-digit vampire numbers.

Show that there are seven 4-digit vampire numbers.

\hint{The 2-digit challenge is do-able by hand (just barely).  The $4$ digit question certainly requires 
some computer assistance!}

\wbvfill

\workbookpagebreak

\item Lagrange's theorem on representation of integers as sums of squares
says that every positive integer can be expressed as the sum of at most 
$4$ squares.  For example, $79 = 7^2 + 5^2 + 2^2 + 1^2$.  Show (exhaustively) 
that $15$ can not be represented using fewer than $4$ squares.

\hint{Note that $15 = 3^2 + 2^2 + 1^2 + 1^2$.  Also, if $15$ were expressible as a sum of fewer than $4$ squares, the squares involved would be $1$, $4$ and $9$.  It's really not that hard to try all the possibilities.}

\wbvfill

\item Show that there are exactly $15$ numbers $x$ in the range $1 \leq x \leq 100$ that can't be represented using fewer than $4$ squares.



\hint{The following Sage code generates all the numbers up to $100$ that {\em can} be written
as the sum of at most $3$ squares.

{\tt
var('x y z') \newline
a=[s$\caret$2 for s in [1..10]]  \newline
b=[s$\caret$2 for s in [0..10]]  \newline
s = []  \newline
for x in a:  \newline
\tab for y in b:  \newline
\tab \tab for z in b:  \newline
\tab \tab \tab s = union(s,[x+y+z])  \newline
s = Set(s)  \newline
H=Set([1..100]) \newline
show(H.intersection(s))  \newline
}
}

\wbvfill

\workbookpagebreak

\item The \index{trichotomy property}\emph{trichotomy property} of the real 
numbers simply states that every real number is either positive or negative 
or zero.  Trichotomy can be used to prove many statements by looking at the
three cases that it guarantees.  Develop a proof (by cases) that the square of
any real number is non-negative.

\hint{By trichotomy, x is either zero, negative, or positive. If x is zero, its square is zero. If x is negative, its square is positive. If x is positive, its square is also positive.}

\wbvfill

\item Consider the game called ``binary determinant tic-tac-toe''\footnote{ %
This question was problem A4 in the 63rd annual %
\index{William Lowell Putnam Mathematics Competition} %
William Lowell Putnam Mathematics Competition (2002).  %
There are three collections of questions %
and answers  from previous Putnam exams available from the MAA % 
\cite{putnam1,putnam2,putnam3}% 
}
which is played by two players who alternately fill in the entries of a 
$3 \times 3$ array.  Player One goes first, placing 1's in the array and 
player Zero goes second, placing 0's.  Player One's goal is that the 
final array have determinant 1, and player Zero's goal is that the 
determinant be 0.  The determinant calculations are carried out mod 2.

Show that player Zero can always win a game of binary determinant tic-tac-toe
by the method of exhaustion.

\hint{If you know something about determinants it would help here.  The determinant will be
0 if there are two identical rows (or columns) in the finished array.  Also, if there is a row or column
that is all zeros, player Zero wins too.  Also, cyclically permuting either rows or columns has no effect
on the determinant of a binary array.  This means we lose no generality in assuming player One's
first move goes (say) in the upper-left corner.}

\wbvfill

\workbookpagebreak

\rule{0pt}{0pt}

\workbookpagebreak

\end{enumerate}
