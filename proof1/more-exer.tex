\begin{enumerate}
\item Suppose you have a savings account which bears interest 
compounded monthly.  The July statement shows a balance of 
\$ 2104.87 and the September statement shows a balance \$ 2125.97.
What would be the balance on the (missing) August statement?

\hint{A savings account where we are not depositing or withdrawing funds has a balance that is growing geometrically.}

\item \label{quad} Recall that a quadratic equation $ax^2+bx+c=0$ has two real solutions
if and only if the discriminant $b^2-4ac$ is positive.  Prove that if 
$a$ and $c$ have different signs then the quadratic equation has two 
real solutions.

\hint{You don't need all the hypotheses. If $a$ and $c$ have different signs, then $ac$ is a negative quantity}

\item Prove that if $x^3-x^2$ is negative then $3x+4 < 7$.

\hint{This follows very easily by the method of working backwards from the conclusion. Remember that when multiplying or dividing both sides of an inequality by some number, the direction of the inequality may reverse (unless we know the number involved is positive).  Also, remember that we can't divide by zero, so if we are (just for example, don't know why I'm mentioning it really\ldots) dividing both sides of an inequality by $x^2$ then we must treat the case where $x=0$ separately.}

\item Prove that for all integers $a,b,$ and $c$, if $a|b$ and $a|(b+c)$, then
$a|c$.

\item Show that if $x$ is a positive real number, then $x+\frac{1}{x} \geq 2$. 

\hint{If you work backwards from the conclusion on this one, you should eventually come to the inequality $(x-1)^2 \geq 0$.  Notice that this inequality is always true -- all squares are non-negative. When you go to write-up your proof (writing things in the forward direction), you'll want to acknowledge this truth. Start with something like ``Regardless of the value of $x$, the quantity $(x-1)^2$ is greater than or equal to zero as it is a perfect square.''}

\item Prove that for all real numbers $a,b,$ and $c$, if $ac<0$, then the quadratic
equation $ax^{2}+bx+c=0$ has two real solutions.\\
\textbf{Hint:} The quadratic equation $ax^{2}+bx+c=0$ has two
real solutions if and only if $b^{2}-4ac>0$ and $a\neq0$.

\hint{This is very similar to problem \ref{quad}.}

\item Show that $\binom{n}{k} \cdot \binom{k}{r} \; = \; \binom{n}{r} \cdot \binom{n-r}{k-r}$ (for all integers $r$, $k$ and $n$ with $r \leq k \leq n$). 

\hint{Use the definition of the binomial coefficients as fractions involving factorials:

E.g. $\displaystyle\binom{n}{k} \; = \; \frac{n!}{k! (n-k)!}$

Write down the definitions, both of the left hand side and the right hand side and consider how you can
convert one into the other.}

\item In proving the \index{product rule} \emph{product rule} in Calculus using the definition of the derivative, we might start our proof with:

\[
\frac{\mbox{d}}{\mbox{d}x} \left( f(x) \cdot g(x) \right)
\]

\[ = \lim_{h \longrightarrow 0} \frac{f(x+h) \cdot g(x+h) - f(x) \cdot g(x)}{h} \]

\noindent The last two lines of our proof should be:
\[
= \lim_{h \longrightarrow 0} \frac{f(x+h) - f(x)}{h} \cdot g(x) \; + \; f(x) \cdot \lim_{h \longrightarrow 0} \frac{g(x+h) - g(x)}{h}
\]

\[
= \frac{\mbox{d}}{\mbox{d}x}\left( f(x) \right) \cdot g(x) \; + \; f(x) \cdot \frac{\mbox{d}}{\mbox{d}x}\left( g(x) \right) 
\]

Fill in the rest of the proof.

\hint{The critical step is to subtract and add the same thing: $f(x)g(x+h)$ in the numerator of the fraction
in the limit which gives the definition of $\frac{\mbox{d}}{\mbox{d}x} \left( f(x) \cdot g(x) \right)$.  Also, you'll need to recall the laws of limits (like ``the limit of a product is the product of the limits -- provided both exist'') }

\end{enumerate}