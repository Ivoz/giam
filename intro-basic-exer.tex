\begin{enumerate}

\item Each of the quantities indexing the rows of the following table
is in one or more of the sets which index the columns.  Place a 
check mark in a table entry if the quantity is in the set.

\begin{tabular}{|c||c|c|c|c|c|} \hline
 & $\Naturals$ & $\Integers$ & $\Rationals$ & $\Reals$ & $\Complexes$
 \\ \hline\hline
\rule{0pt}{15pt} $17$ & & & & & \\ \hline
\rule{0pt}{15pt} $\pi$ & & & & & \\ \hline
\rule{0pt}{15pt} $22/7$ & & & & & \\ \hline
\rule{0pt}{15pt} $-6$ & & & & & \\ \hline
\rule{0pt}{15pt} $e^0$ & & & & & \\ \hline
\rule{0pt}{15pt} $1+i$ & & & & & \\ \hline
\rule{0pt}{15pt} $\sqrt{3}$ & & & & & \\ \hline
\rule{0pt}{15pt} $i^2$ & & & & & \\  \hline
\end{tabular}

\vfill

\workbookpagebreak

\item Write the set $\Integers$ of integers using a singly infinite
listing.

\vfill


\item Identify each as rational or irrational.
\begin{enumerate}
\item $5021.2121212121\ldots$
\item $0.2340000000\ldots$
\item $12.31331133311133331111\ldots$
\item $\pi$
\item $2.987654321987654321987654321\ldots$
\end{enumerate}

\vfill

\textbookpagebreak

\item The ``see and say'' sequence is produced by first writing a 1, 
then iterating the following procedure:  look at the previous entry 
and say how many entries there are of each integer and write down what 
you just said.  The first several terms of the ``see and say'' sequence 
are $1, 11, 21, 1112, 3112, 211213, 312213, 212223, \ldots$.  Comment on the
rationality (or irrationality) of the number whose decimal digits are obtained 
by concatenating the ``see and say'' sequence.

\vfill

\item Give a description of the set of rational numbers whose decimal
expansions terminate.  (Alternatively, you may think of their decimal
expansions ending in an infinitely-long string of zeros.)

\vfill

\item Find the first 20 decimal places of $\pi$, $3/7$, $\sqrt{2}$, 
  $2/5$, $16/17$, $\sqrt{3}$, $1/2$ and $42/100$.  Classify each of
these quantity's decimal expansion as: terminating, having a repeating
pattern, or showing no discernible pattern.

\vfill

\workbookpagebreak
 
\item Consider the process of long division.  Does this algorithm give
any insight as to why rational numbers have terminating or repeating
decimal expansions?  Explain.

\vfill

\item Give an argument as to why the product of two rational numbers
is again a rational.

\vfill

\item Perform the following computations with complex numbers

  \begin{enumerate}
  \item \rule{0pt}{20pt}$ (4 + 3i) - (3 + 2i) $
  \item \rule{0pt}{20pt}$ (1 + i) + (1 - i) $
  \item \rule{0pt}{20pt}$ (1 + i) \cdot (1 - i) $
  \item \rule{0pt}{20pt}$ (2 - 3i) \cdot (3 - 2i) $
  \end{enumerate}

\vfill

\textbookpagebreak

\item The {\em conjugate} of a complex number is denoted with a
  superscript star, and is formed by negating the imaginary part.
  Thus if $z = 3+ 4i$ then the conjugate of $z$ is  $z^\ast = 3-4i$.
  Give an argument as to why the product of a complex number and its
  conjugate is a real quantity.  (I.e. the imaginary part of
  $z\cdot z^\ast$ is necessarily 0, no matter what complex number is
  used for $z$.) 

\workbookpagebreak

\end{enumerate}



%% Emacs customization
%% 
%% Local Variables: ***
%% TeX-master: "GIAM.tex" ***
%% comment-column:0 ***
%% comment-start: "%% "  ***
%% comment-end:"***" ***
%% End: ***

