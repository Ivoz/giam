\begin{enumerate}
\item Determine the number of entries in the following sequences.

  \begin{enumerate}
  \item \wbitemsep $(999, 1000, 1001, \ldots  2006)$
  \item \wbitemsep $(13, 15, 17, \ldots 199)$
  \item \wbitemsep $(13, 19, 25, \ldots 601)$
  \item \wbitemsep $(5, 10, 17, 26, 37, \ldots 122)$
  \item \wbitemsep $(27, 64, 125, 216, \ldots 8000)$
  \item \wbitemsep $(7, 11, 19, 35, 67, \ldots 131075)$
  \end{enumerate}

\workbookpagebreak

\item How many ``full houses'' are there in Yahtzee?  (A full house is a pair
together with a three-of-a-kind.)

\wbvfill

\item In how many ways can you get ``two pairs'' in Yahtzee?

\wbvfill

\item Prove that the binomial coefficients $\displaystyle \binom{n+k-1}{k}$
and $\displaystyle \binom{n+k-1}{n-1}$ are equal.

\wbvfill

\workbookpagebreak

\item The ``Cryptographer's alphabet'' is used to supply small examples
in coding and cryptography.  It consists of the first 6 letters, $\{a, b, c, d, e, f\}$.  How many ``words'' of length up to 6 can be made with this 
alphabet?  (A word need not actually be a word in English, for example 
both ``fed'' and ``dfe'' would be words in the sense we are using the term.)

\wbvfill

\item How many ``words'' are there of length 4, with distinct letters from the 
Cryptographer's alphabet, in which the letters appear in increasing order 
alphabetically?  (``Acef'' would be one such word, but ``cafe'' would not.)

\wbvfill

\item How many ``words'' are there of length 4 from the 
Cryptographer's alphabet, with repeated letters allowed,
 in which the letters appear in non-decreasing order alphabetically?

\wbvfill

\workbookpagebreak

\item How many subsets does a finite set have?

\wbvfill

\item How many handshakes will transpire when $n$ people first meet?

\wbvfill

\item How many functions are there from a set of size $n$ to a set of size $m$?

\wbvfill

\item How many relations are there from a set of size $n$ to a set of size $m$?

\wbvfill

\workbookpagebreak

\end{enumerate}

%% Emacs customization
%% 
%% Local Variables: ***
%% TeX-master: "GIAM-hw.tex" ***
%% comment-column:0 ***
%% comment-start: "%% "  ***
%% comment-end:"***" ***
%% End: ***

