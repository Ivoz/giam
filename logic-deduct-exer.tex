\begin{enumerate}

\item In the movie ``Monty Python and the Holy Grail'' we encounter
a medieval villager who (with a bit of prompting) makes the 
following argument.

\begin{quote}
If she weighs the same as a duck, then she's made of wood. \newline
If she's made of wood then she's a witch. \newline
Therefore, if she weighs the same as a duck, she's a witch.
\end{quote} 

Which rule of inference is he using?

\item In constructive dilemma, the antecedent of the conditional 
sentences are usually chosen to represent opposite alternatives. 
This allows us to introduce their disjunction as a tautology. 
Consider the following proof that there is never any reason to worry
(found on the walls of an Irish pub).

\begin{quote}
Either you are sick or you are well. \newline
If you are well there's nothing to worry about. \newline
If you are sick there are just two possibilities: \newline
Either you will get better or you will die. \newline
If you are going to get better there's nothing to worry about. \newline
If you are going to die there are just two possibilities:\newline
Either you will go to Heaven or to Hell. \newline
If you go to Heaven there is nothing to worry about.
If you go to Hell, you'll be so busy shaking hands with all your friends there won't be time to worry \ldots
\end{quote}

Identify the three tautologies that are introduced in this ``proof.''

\newpage

\item For each of the following arguments, write it in symbolic form and determine 
which rules of inference are used.

\begin{enumerate}
\item \rule{0pt}{24pt} You are either with us, or you're against us.  And you don't appear to be with us.
So, that means you're against us!
\item \rule{0pt}{24pt} All those who had cars escaped the flooding.  Sandra had a car -- therefore, Sandra
escaped the flooding.
\item \rule{0pt}{24pt}  When Johnny goes to the casino, he always gambles 'til he goes broke.  Today, Johnny
has money, so Johnny hasn't been to the casino recently.
\item \rule{0pt}{24pt} (A non-constructive proof that there are 
irrational numbers $a$ and $b$ such that $a^b$ is rational.)  
Either $\sqrt{2}^{\sqrt{2}}$ is rational or it is irrational.
If $\sqrt{2}^{\sqrt{2}}$ is rational, we let $a=b=\sqrt{2}$.
Otherwise, we let $a=\sqrt{2}^{\sqrt{2}}$ and $b=\sqrt{2}$.
(Since $\sqrt{2}^{\sqrt{2}^{\sqrt{2}}} = 2$, which is rational.) It follows that in either case, there
are irrational numbers $a$ and $b$ such that $a^b$ is rational.
\end{enumerate}

\end{enumerate}