
\begin{table}[hbt] 
\begin{center}
\begin{tabular}{c|c|c} 
 & \begin{minipage}{.35\textwidth} \centerline{Intersection}
\centerline{\rule[-10pt]{0pt}{10pt}version} \end{minipage} & 
\begin{minipage}{.35\textwidth} \centerline{Union}
\centerline{\rule[-10pt]{0pt}{10pt}version} \end{minipage} \\ \hline
\begin{minipage}{.25\textwidth} \rule{0pt}{22pt}Commutative \\ \rule{12pt}{0pt} laws\rule[-10pt]{0pt}{10pt} \end{minipage} & 
\begin{minipage}{.35\textwidth} \centerline{$A \cap B = B \cap A$} \end{minipage} & 
\begin{minipage}{.35\textwidth} \centerline{$A \cup B = B \cup A$} \end{minipage} \\ \hline
\begin{minipage}{.25\textwidth} \rule{0pt}{22pt}Associative \\ \rule{12pt}{0pt} laws\rule[-10pt]{0pt}{10pt} \end{minipage} & 
\begin{minipage}{.35\textwidth} \centerline{$A \cap (B \cap C)$\rule{26pt}{0pt}} 
\centerline{\rule{26pt}{0pt} $= (A \cap B) \cap C $}\end{minipage} &
\begin{minipage}{.35\textwidth} \centerline{$A \cup (B \cup C)$ \rule{26pt}{0pt}}
\centerline{\rule{26pt}{0pt} $= (A \cup B) \cup C $} \end{minipage} \\ \hline 
\begin{minipage}{.25\textwidth} \rule{0pt}{22pt}Distributive \\ \rule{12pt}{0pt} laws\rule[-10pt]{0pt}{10pt} \end{minipage} &  
\begin{minipage}{.35\textwidth} 
\centerline{$A \cap (B \cup C) = $ \rule{26pt}{0pt}} 
\centerline{\rule{16pt}{0pt}$(A \cap B) \cup (A \cap C)$} \end{minipage} & 
\begin{minipage}{.35\textwidth} \centerline{$A \cup (B \cap C) = $ \rule{26pt}{0pt}} 
\centerline{\rule{16pt}{0pt}$(A \cup B) \cap (A \cup C)$} \end{minipage} \\ \hline 
\begin{minipage}{.25\textwidth} \rule{0pt}{22pt}DeMorgan's \\ \rule{12pt}{0pt} laws\rule[-10pt]{0pt}{10pt} \end{minipage} & 
\begin{minipage}{.35\textwidth} \centerline{$\overline{A \cap B}$ \rule{25pt}{0pt}}
\centerline{ \rule{16pt}{0pt} $ = \; \overline{A} \cup \overline{B}$} \end{minipage} & 
\begin{minipage}{.35\textwidth} \centerline{$\overline{A \cup B}$\rule{25pt}{0pt}}
\centerline{ \rule{16pt}{0pt} $= \; \overline{A} \cap \overline{B}$} \end{minipage} \\ \hline 
\begin{minipage}{.25\textwidth} \rule{0pt}{22pt}Complementarity\rule[-10pt]{0pt}{10pt} \end{minipage} & 
\begin{minipage}{.35\textwidth} \centerline{$A \cap \overline{A} \; = \; \emptyset$} \end{minipage} & 
\begin{minipage}{.35\textwidth} \centerline{$A \cup \overline{A} \; = \; U$} \end{minipage} \\ \hline 
\begin{minipage}{.25\textwidth} \rule{0pt}{22pt}Identity \\ \rule{12pt}{0pt} laws\rule[-10pt]{0pt}{10pt} \end{minipage} & 
\begin{minipage}{.35\textwidth} \centerline{$A \cap U = A$} \end{minipage} & 
\begin{minipage}{.35\textwidth} \centerline{$A \cup \emptyset = A$} \end{minipage} \\ \hline 
\begin{minipage}{.25\textwidth} \rule{0pt}{22pt}Domination\rule[-10pt]{0pt}{10pt} \end{minipage} & 
\begin{minipage}{.35\textwidth}  \centerline{$A \cap \emptyset = \emptyset$} \end{minipage} & 
\begin{minipage}{.35\textwidth} \centerline{$A \cup U = U$} \end{minipage} \\ \hline
\begin{minipage}{.25\textwidth} \rule{0pt}{22pt}Idempotence\rule[-10pt]{0pt}{10pt} \end{minipage} & 
\begin{minipage}{.35\textwidth} \centerline{$A \cap A = A$} \end{minipage} & 
\begin{minipage}{.35\textwidth} \centerline{$A \cup A = A$} \end{minipage} \\ \hline
\begin{minipage}{.25\textwidth} \rule{0pt}{22pt}Absorption\rule[-10pt]{0pt}{10pt} \end{minipage} & 
\begin{minipage}{.35\textwidth} \centerline{$A \cap (A \cup B) = A$} \end{minipage} & 
\begin{minipage}{.35\textwidth} \centerline{$A \cup (A \cap B) = A$} \end{minipage} \\
\end{tabular} 
\end{center} 
\caption{Basic set theoretic equalities.}
\index{set theoretic equalities}
\label{tab:set_equiv}
\end{table}

\clearpage


\noindent{\large \bf Exercises --- \thesection\ }

\begin{enumerate}
\item Let $A = \{1,2,\{1,2\},b\}$ and let $B=\{a, b, \{1,2\} \}$.
Find the following:
  \begin{enumerate}
  \item $A \cap B$
  \item $A \cup B$
  \item $A \setminus B$
  \item $B \setminus A$
  \item $A \triangle B$
  \end{enumerate}

\item In a standard deck of playing cards one can distinguish sets
based on face-value and/or suit.  Let $A, 2, \ldots 9, 10, J, Q$ and $K$
represent the sets of cards having the various face-values.  Also, let
$\heartsuit$, $\spadesuit$, $\clubsuit$ and $\diamondsuit$ be the 
sets of cards having the possible suits.  Find the following
  \begin{enumerate}
  \item $A \cap \heartsuit$
  \item $A \cup \heartsuit$
  \item $J \cap (\spadesuit \cap \heartsuit)$ %one-eyed jacks
  \item $K \cap \heartsuit$ %suicide king
  \item $A \cap K$
  \item $A \cup K$
  \end{enumerate}

\item Do element-chasing proofs (show that an element is in the left-hand side if and only if it is in the right-hand side) to prove each of the following set equalities.  

  \begin{enumerate}
  \item $\overline{A\cap B} \; = \; \overline{A}\cup\overline{B}$

  \item $A\cup B \; = \; A\cup(\overline{A}\cap B)$

  \item $A\triangle B \; = \; (A\cup B)\setminus(A\cap B)$

  \item $(A\cup B)\setminus C \; = \; (A\setminus C)\cup(B\setminus C)$

  \end{enumerate}

\item For each positive integer $n$, we'll define an interval $I_n$
by

\[ I_n = [-n, 1/n). \]

Find the union and intersection of all the intervals in this infinite family.

\[ \bigcup_{n \in \Naturals} I_n \quad = \]

\[ \bigcap_{n \in \Naturals} I_n \quad = \]

\item There is a set $X$ such that, for all sets $A$, we have 
$X \triangle A = A$.  What is $X$?

\item There is a set $Y$ such that, for all sets $A$, we have 
$Y \triangle A = \overline{A}$.  What is $Y$?

\item In proving a set-theoretic identity, we are basically showing that
two sets are equal.  One reasonable way to proceed is to show that
each is contained in the other.  Prove that 
$A \cap (B \cup C) = (A \cap B) \cup (A \cap C)$ by showing that 
$A \cap (B \cup C) \subseteq (A \cap B) \cup (A \cap C)$ and 
$(A \cap B) \cup (A \cap C) \subseteq A \cap (B \cup C)$.

\item Prove the set-theoretic versions of DeMorgan's laws using the technique
discussed in the previous problem.

\end{enumerate}


%% Emacs customization
%% 
%% Local Variables: ***
%% TeX-master: "GIAM-hw.tex" ***
%% comment-column:0 ***
%% comment-start: "%% "  ***
%% comment-end:"***" ***
%% End: ***

