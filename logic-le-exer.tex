\begin{enumerate}

\item There are 3 operations used in basic algebra (addition, 
multiplication and exponentiation) and thus
there are potentially 6 different distributive laws.  State
all 6 ``laws'' and determine which 2 are actually valid.
(As an example, the distributive law of addition over multiplication
would look like $x + (y \cdot z) = (x + y) \cdot (x + z)$, this isn't 
one of the true ones.) 

\item Use truth tables to verify or disprove the following 
logical equivalences.

\begin{enumerate}
\item $(A \land B) \lor B \; \cong \; (A \lor B) \land B$
\item $A \land (B \lor {\lnot}A) \; \cong \; A \land B $
\item $(A \land {\lnot}B) \lor ({\lnot}A \land {\lnot}B) \cong
(A \lor {\lnot}B) \land ({\lnot}A \lor {\lnot}B)$ 
\item The absorption laws.
\end{enumerate}

\item Draw pairs of related digital logic circuits that illustrate
DeMorgan's laws.

\item Find the negation of each of the following and simplify as much as possible.
\medskip

  \begin{enumerate}
  \item $(A \lor B) \; \iff \; C$
\medskip

  \item $(A \lor B) \; \implies \; (A \land B)$

  \end{enumerate}

\item Because a conditional sentence is equivalent to a certain disjunction, and 
because DeMorgan's law tells us that the negation of a disjunction is a conjunction,
it follows that the negation of a conditional is a conjunction.  Find denials (the negation
of a sentence is often called its ``denial'') for each of the following conditionals.

\begin{enumerate}
\item ``If you smoke, you'll get lung cancer.''
\item ``If a substance glitters, it is not necessarily gold.''
\item ``If there is smoke, there must also be fire.''
\item ``If a number is squared, the result is positive.''
\item ``If a matrix is square, it is invertible.''
\end{enumerate}

\newpage

\item The so-called ``ethic of reciprocity'' is an idea that has come 
up in many of the world's religions and philosophies.  
Below are statements of the ethic
from several sources.  Discuss their logical meanings and determine which (if 
any) are logically equivalent.

\begin{enumerate}
\item ``One should not behave towards others in a way which is disagreeable to oneself.'' Mencius Vii.A.4 (Hinduism)
\item ``None of you [truly] believes until he wishes for his brother what he wishes for himself.'' Number 13 of Imam ``Al-Nawawi's Forty Hadiths.'' (Islam)
\item ``And as ye would that men should do to you, do ye also to them likewise.'' Luke 6:31, King James Version. (Christianity)
\item ``What is hateful to you, do not to your fellow man. This is the law: all the rest is commentary.'' Talmud, Shabbat 31a. (Judaism)
\item ``An it harm no one, do what thou wilt'' (Wicca)
\item ``What you would avoid suffering yourself, seek not to impose on others.'' (the Greek philosopher Epictetus -- first century A.D.)
\item ``Do not do unto others as you expect they should do unto you. Their tastes may not be the same.'' (the Irish playwright George Bernard Shaw -- 20th century A.D.)
\end{enumerate}

\item You encounter two natives of the land of knights and knaves. Fill
in an explanation for each line of the proofs of their identities. 

\begin{enumerate}
\item Natasha says, ``Boris is a knave.'' \\
Boris says, ``Natasha and I are knights.''\\

\textbf{Claim:} Natasha is a knight, and Boris is a knave.\\

\begin{proof} If Natasha is a knave, then Boris is a knight.\\
If Boris is a knight, then Natasha is a knight.\\
Therefore, if Natasha is a knave, then Natasha is a knight.\\
Hence Natasha is a knight.\\
Therefore, Boris is a knave.
\end{proof}

\item Bonaparte says ``I am a knight and Wellington is a knave.''\\
Wellington says ``I would tell you that B is a knight.''

\textbf{Claim:} Bonaparte is a knight and Wellington is a knave.

\begin{proof}
    Either Wellington is a knave or Wellington is a knight.\\
    If Wellington is a knight it follows that Bonaparte is a knight.\\
    If Wellington is a knave, then his statement "I would tell you that Bonaparte is a knight" is false. \\
    So Wellington would tell us that Bonaparte is a knave. \\
    Since Wellington is a knave we conclude that Bonaparte is a knight.\\
    Therefore Bonaparte is a knight.\\
    Finally, since Bonaparte is a knight, Wellington is a knave. \\
\end{proof}

\end{enumerate}

\end{enumerate}
