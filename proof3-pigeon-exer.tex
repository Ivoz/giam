\begin{enumerate}

\item The statement that there are two non-bald New Yorkers with
the same number of hairs on their heads requires some careful 
estimates to justify it.  Please justify it.

\item A mathematician, who always rises earlier than her spouse, has
developed a scheme -- using the pigeonhole principle -- to ensure that
she always has a matching pair of socks.  She keeps only blue socks, green 
socks and
black socks in her sock drawer -- 10 of each.  So as not to wake her 
husband she must
select some number of socks from her drawer in the early morning dark
and take them with her to the adjacent bathroom where she dresses.
What number of socks does she choose?

\item If we select $1001$ numbers from the set $\{1, 2, 3, \ldots, 2000\}$
it is certain that there will be two numbers selected such that one divides
the other.  We can prove this fact by noting that every number in the given
set can be expressed in the form $2^k \cdot m$ where $m$ is an odd number
and using the pigeonhole principle.  Write-up this proof.

\item Given any set of $53$ integers, show that there are two of them
having the property 
that either their sum or their difference is evenly divisible by $103$.

\item Prove that if $10$ points are placed inside a square of side length 3,
there will be 2 points within $\sqrt{2}$ of one another.

\item Prove that if $10$ points are placed inside an equilateral triangle
of side length 3, there will be 2 points within $1$ of one another.

\item Prove that in a simple graph (an undirected graph with no 
loops or parallel edges) having $n$ nodes, there must be two nodes 
having the same degree. 

\end{enumerate}


%% Emacs customization
%% 
%% Local Variables: ***
%% TeX-master: "GIAM-hw.tex" ***
%% comment-column:0 ***
%% comment-start: "%% "  ***
%% comment-end:"***" ***
%% End: ***

