%\documentclass[handout,landscape]{beamer}
\documentclass[landscape]{beamer}
%\hypersetup{pdfpagemode=FullScreen}
\mode<handout>
{
  \usepackage{pgf}
  \usepackage{pgfpages}

\pgfpagesdeclarelayout{6 on 1 boxed}
{
  \edef\pgfpageoptionheight{\the\paperheight} 
  \edef\pgfpageoptionwidth{\the\paperwidth}
  \edef\pgfpageoptionborder{0pt}
}
{
  \pgfpagesphysicalpageoptions
  {%
    logical pages=6,%
    physical height=\pgfpageoptionheight,%
    physical width=\pgfpageoptionwidth%
  }
  \pgfpageslogicalpageoptions{1}
  {%
    border code=\pgfsetlinewidth{2pt}\pgfstroke,%
    border shrink=\pgfpageoptionborder,%
    resized width=.5\pgfphysicalwidth,%
    resized height=.5\pgfphysicalheight,%
    center=\pgfpoint{.25\pgfphysicalwidth}{.833\pgfphysicalheight}%
  }%
  \pgfpageslogicalpageoptions{2}
  {%
    border code=\pgfsetlinewidth{2pt}\pgfstroke,%
    border shrink=\pgfpageoptionborder,%
    resized width=.5\pgfphysicalwidth,%
    resized height=.5\pgfphysicalheight,%
    center=\pgfpoint{.75\pgfphysicalwidth}{.833\pgfphysicalheight}%
  }%
  \pgfpageslogicalpageoptions{3}
  {%
    border code=\pgfsetlinewidth{2pt}\pgfstroke,%
    border shrink=\pgfpageoptionborder,%
    resized width=.5\pgfphysicalwidth,%
    resized height=.5\pgfphysicalheight,%
    center=\pgfpoint{.25\pgfphysicalwidth}{.5\pgfphysicalheight}%
  }%
  \pgfpageslogicalpageoptions{4}
  {%
    border code=\pgfsetlinewidth{2pt}\pgfstroke,%
    border shrink=\pgfpageoptionborder,%
    resized width=.5\pgfphysicalwidth,%
    resized height=.5\pgfphysicalheight,%
    center=\pgfpoint{.75\pgfphysicalwidth}{.5\pgfphysicalheight}%
  }%
  \pgfpageslogicalpageoptions{5}
  {%
    border code=\pgfsetlinewidth{2pt}\pgfstroke,%
    border shrink=\pgfpageoptionborder,%
    resized width=.5\pgfphysicalwidth,%
    resized height=.5\pgfphysicalheight,%
    center=\pgfpoint{.25\pgfphysicalwidth}{.167\pgfphysicalheight}%
  }%
  \pgfpageslogicalpageoptions{6}
  {%
    border code=\pgfsetlinewidth{2pt}\pgfstroke,%
    border shrink=\pgfpageoptionborder,%
    resized width=.5\pgfphysicalwidth,%
    resized height=.5\pgfphysicalheight,%
    center=\pgfpoint{.75\pgfphysicalwidth}{.167\pgfphysicalheight}%
  }%
}


  \pgfpagesuselayout{6 on 1 boxed}[letterpaper, border shrink=5mm]
  \nofiles
}

\usepackage{listings}
%\lstset{language=TeX}
\usepackage{multimedia}
\usepackage[normalem]{ulem}
\usepackage{amssymb}

%\usecolortheme[named=Purple]{structure} 
%\usetheme{Copenhagen}
\usetheme{Warsaw} 
\usecolortheme{seahorse}
\useoutertheme{infolines} 
%\usetheme[height=7mm]{Rochester} 
%\setbeamertemplate{items}[ball] 
\setbeamertemplate{blocks}[rounded][shadow=true] 
%\setbeamertemplate{navigation symbols}{} 
\author{Joe Fields}
\title{Introduction to Proof} 
%\subtitle{}
\date{Lecture 12 (GIAM \S 2.5)}
\institute[SCSU]{ {\tt fieldsj1@southernct.edu} }

\newcommand{\versionNum}{$3.2$\ }

\newboolean{InTextBook}
\setboolean{InTextBook}{false}
\newboolean{InWorkBook}
\setboolean{InWorkBook}{false}
\newboolean{InHints}
\setboolean{InHints}{false}

%When this boolean is true (beginning in Section 5.1) we will use the convention
% that $0 \in \Naturals$.  If it is false we will continue to count $1$ as the smallest
%natural number (thus making Giuseppe Peano spin in his grave...)
 
\newboolean{ZeroInNaturals}

%This boolean is used to distinguish the version where we use $\sim$ rather than $\lnot$

\newboolean{LNotIsSim}

%The values of the last two booleans are set in ``switches.tex''

\setboolean{ZeroInNaturals}{true}
\setboolean{LNotIsSim}{false}


\let\savedlnot\lnot
\ifthenelse{\boolean{LNotIsSim}}{\renewcommand{\lnot}{\sim} }{}

%This command puts different amounts of space depending on whether we are
% in the text, the workbook or the hints & solutions manual. 
\newcommand{\twsvspace}[3]{%
 \ifthenelse{\boolean{InTextBook} }{\vspace{#1}}{%
  \ifthenelse{\boolean{InWorkBook} }{\vspace{#2}}{%
   \ifthenelse{\boolean{InHints} }{\vspace{#3}}{} %
   }%
  }%
 }


\newcommand{\wbvfill}{\ifthenelse{\boolean{InWorkBook}}{\vfill}{}}
\newcommand{\wbitemsep}{\ifthenelse{\boolean{InWorkBook} }{\rule[-24pt]{0pt}{60pt}}{}}
\newcommand{\textbookpagebreak}{\ifthenelse{\boolean{InTextBook}}{\newpage}{}}
\newcommand{\workbookpagebreak}{\ifthenelse{\boolean{InWorkBook}}{\newpage}{}}
\newcommand{\hintspagebreak}{\ifthenelse{\boolean{InHints}}{\newpage}{}}

\newcommand{\hint}[1]{\ifthenelse{\boolean{InHints}}{ {\par \hspace{12pt} \color[rgb]{0,0,1} #1 } }{}}
\newcommand{\inlinehint}[1]{\ifthenelse{\boolean{InHints}}{ { \color[rgb]{0,0,1} #1 } }{}}

\newlength{\cwidth}
\newcommand{\cents}{\settowidth{\cwidth}{c}%
\divide\cwidth by2
\advance\cwidth by-.1pt
c\kern-\cwidth
\vrule width .1pt depth.2ex height1.2ex
\kern\cwidth}

\newcommand{\sageprompt}{ {\tt sage$>$} }
\newcommand{\tab}{\rule{20pt}{0pt}}
\newcommand{\blnk}{\rule{1.5pt}{0pt}\rule{.4pt}{1.2pt}\rule{9pt}{.4pt}\rule{.4pt}{1.2pt}\rule{1.5pt}{0pt}}
\newcommand{\suchthat}{\; \rule[-3pt]{.5pt}{13pt} \;}
\newcommand{\divides}{\!\mid\!}
\newcommand{\tdiv}{\; \mbox{div} \;}
\newcommand{\restrict}[2]{#1 \,\rule[-4pt]{.25pt}{14pt}_{\,#2}}
\newcommand{\lcm}[2]{\mbox{lcm} (#1, #2)}
\renewcommand{\gcd}[2]{\mbox{gcd} (#1, #2)}
\newcommand{\Naturals}{{\mathbb N}}
\newcommand{\Integers}{{\mathbb Z}}
\newcommand{\Znoneg}{{\mathbb Z}^{\mbox{\tiny noneg}}}
\ifthenelse{\boolean{ZeroInNaturals}}{%
  \newcommand{\Zplus}{{\mathbb Z}^+} }{%
  \newcommand{\Zplus}{{\mathbb N}} }
\newcommand{\Enoneg}{{\mathbb E}^{\mbox{\tiny noneg}}}
\newcommand{\Qnoneg}{{\mathbb Q}^{\mbox{\tiny noneg}}}
\newcommand{\Rnoneg}{{\mathbb R}^{\mbox{\tiny noneg}}}
\newcommand{\Rationals}{{\mathbb Q}}
\newcommand{\Reals}{{\mathbb R}}
\newcommand{\Complexes}{{\mathbb C}}
%\newcommand{\F2}{{\mathbb F}_{2}}
\newcommand{\relQ}{\mbox{\textsf Q}}
\newcommand{\relR}{\mbox{\textsf R}}
\newcommand{\nrelR}{\mbox{\raisebox{1pt}{$\not$}\rule{1pt}{0pt}{\textsf R}}}
\newcommand{\relS}{\mbox{\textsf S}}
\newcommand{\relA}{\mbox{\textsf A}}
\newcommand{\Dom}[1]{\mbox{Dom}(#1)}
\newcommand{\Cod}[1]{\mbox{Cod}(#1)}
\newcommand{\Rng}[1]{\mbox{Rng}(#1)}

\DeclareMathOperator\caret{\raisebox{1ex}{$\scriptstyle\wedge$}}

\newtheorem*{defi}{Definition}
\newtheorem*{exer}{Exercise}
\newtheorem{thm}{Theorem}[section]
\newtheorem*{thm*}{Theorem}
\newtheorem{lem}[thm]{Lemma}
\newtheorem*{lem*}{Lemma}
\newtheorem{cor}{Corollary}
\newtheorem{conj}{Conjecture}

\renewenvironment{proof}%
{\begin{quote} \emph{Proof:} }%
{\rule{0pt}{0pt} \newline \rule{0pt}{15pt} \hfill Q.E.D. \end{quote}}


\newcommand{\vs}{\rule{0pt}{12pt}}

\AtBeginSection[]
{
 \begin{frame}{Table of Contents} 
  \tableofcontents[currentsection]
 \end{frame}
}

%%%% SAVE %%%%
%{ %magic to get a full screen image...
%\setbeamertemplate{navigation symbols}{}  % hide navigation buttons 
%\setbeamertemplate{background canvas}{\centerline{\includegraphics 
%	[height=\paperheight]{Cantor_4.jpeg}}}
%\begin{frame}[plain]
%\rule{0pt}{0pt}
%\end{frame} 
%} %end of magic


\begin{document}

\begin{frame}[plain]
  \titlepage
\end{frame}


\section{Quantified sentences}

\begin{frame}{a motivating example}
\begin{itemize}
\item Consider the difference between the following\pause
\begin{itemize}
\item $x$ is a prime number less than 100.\pause
\item There are precisely 25 prime numbers less than 100.\pause
\end{itemize}
\item The first is ambiguous.\pause
\item The second is a statement.\pause
\item You can change ambiguous things into statements by asking ``How many?''
\end{itemize}
\end{frame}

\begin{frame}{quantifiers revisited}
\begin{itemize}
\item If the answer to ``How many?'' is ``All of them!'' \pause
\item \hspace{.2in} You've got something that should be universally quantified.\pause
\item If the answer to ``How many?'' is ``Not sure, but at least some of them\textellipsis'' \pause
\item \hspace{.2in} You've got something that should be existentially quantified.\pause
\item Takeaway: Quantifiers are one way to turn ambiguous sentences in logical statements.
\end{itemize}
\end{frame}

\begin{frame}{open sentences}
\begin{itemize}
\item An {\em open sentence} is a sentence that involves variables.\pause
\item Example: Let $P(x,y) \; = \; x < y$\pause
\item As it stands, $P(x,y)$ is just ambiguous -- we can't tell which is bigger unless you tell us what $x$ and $y$ are!\pause
\item But add some quantifiers and you get a statement: \pause
\item $\forall x \in \Reals, \exists y \in \Reals, P(x,y)$ \pause
\item For every real number $x$, there is a real number $y$ that's bigger than it. \pause 
\item $\forall x \in \Reals, \forall y \in \Reals, P(x,y)$ \pause is also a statement (a false one!) \pause
\item $\exists x \in \Reals, \exists y \in \Reals, P(x,y)$ \pause ???
\end{itemize}
\end{frame}

\begin{frame}{bound variables}
\begin{itemize}
\item When a variable is in the scope of a quantifier we call it {\em bound}.\pause
\item If all of the variables in a sentence are bound, the sentence is a statement.\pause
\item Sentences with some variables that remain unbound are often seen in definitions\pause
\end{itemize}
\end{frame}

\begin{frame}{examples}
\begin{itemize}
\item The sentence $\exists y, x<y$ has one bound variable ($y$) and one unbound variable ($x$). \pause
\item Notice that this sentence defines some concept?\pause
\item It tells us that $x$ is finite!\pause
\item Here are a few examples taken from the book:\pause
\begin{itemize}
\item[i)] $P(x)$ = ``$2^{2^x}+1$ is a prime.'' \pause

\item[ii)] $Q(x,y)$ = ``$x$ is prime or $y$ is a divisor of $x$.'' \pause

\item[iii)] $L(f,c,l)$ = ``The function $f$ has limit $l$ at $c$, if 
and only if, 
for every positive number $\epsilon$, there is a positive number $\delta$ 
such that whenever $|x-c| < \delta$ it follows that $|f(x)-l| < \epsilon$.''  
\end{itemize}
\end{itemize}
\end{frame}

\section{examples}

\begin{frame}{Fermat numbers}
\begin{itemize}
\item $P(x)$ = ``$2^{2^x}+1$ is a prime.'' \pause
\item There are no quantifiers, often when that happens we implicitly assume universal quantification.\pause
\item So, ``$\forall x \in \Naturals, \; 2^{2^x}+1$ is a prime'' would be the likely interpretation.\pause
\item Sadly that's not true \pause
\item If we let $U \; = \; \{0,1,2,3,4\}$, then ``$\forall x \in U, \; 2^{2^x}+1$ is a prime'' \pause
\item Notice that we could re-express that as $P(0) \land P(1) \land P(2) \land P(3) \land P(4)$.
\end{itemize}
\end{frame}

\begin{frame}{Euler}
\begin{itemize}
\item Fermat probably computed that $F_5=4294967297$, but \ldots \pause
\item Fermat died in 1665. \pause
\item Euler looked at this issue around 1732 and found that $641 \divides F_5$. \pause
\item In the meantime, John Napier had invented his ``bones'' which were a mechanical aid to calculation.  Perhaps Euler was able to enlist a squad of minions to do all the tedious trial division using the bones. \pause
\item But, Euler did have a reputation for being a lightning-fast mental calculator\textellipsis
\end{itemize}
\end{frame}

\begin{frame}{Definition of a limit}
\begin{itemize}
\item $L(f,c,l)$ = ``The function $f$ has limit $l$ at $c$, if 
and only if, 
for every positive number $\epsilon$, there is a positive number $\delta$ 
such that whenever $|x-c| < \delta$ it follows that $|f(x)-l| < \epsilon$.''  \pause
\item Unbound variables are $f$, $c$ and $l$. \pause ($x$, $\delta$ and $\epsilon$ are bound.)\pause
\item Let's do the translation to symbols:\pause
\item $ \lim_{x\rightarrow c} f(x) \, = \, l \quad \iff $ \newline
\rule{72pt}{0pt} $\forall \epsilon>0 \, \exists \delta>0 \, \forall x \, (|x-c| < \delta) \implies (|f(x)-l| < \epsilon). $
\end{itemize}
\end{frame}

\section{Quantifiers as logical connectives}

\begin{frame}{universals and conjunctions}
\begin{itemize}
\item When we're dealing with a finite universe, (like $U \; = \; \{0,1,2,3,4\}$ in the Fermat numbers example.) \pause
\item A universally quantified sentence means the same thing as a conjunction involving all the members of the universe.\pause
\item $\forall x \in U, \; P(x)$ just means $P(0) \land P(1) \land P(2) \land P(3) \land P(4)$.\pause
\item There's even a way to denote the conjunction of infinitely many statements\pause 
\item Let $F(n)$ be ``n has a factorization into prime power factors.'' \pause
\item $\displaystyle \forall n \in \Naturals, \; F(n)$ means the same thing as $\displaystyle \bigwedge_{n \in \Naturals} F(n)$.
\end{itemize}
\end{frame}

\begin{frame}{existentials and disjunctions}
\begin{itemize}
\item The disjunction of a bunch of statements will be true when any one of the statements is true.\pause
\item An existentially quantified sentence means the same thing as a disjunction involving all the members of the universe.\pause
\item An example from GIAM:
\begin{quote}
Find a four digit number that
is an integer multiple of its reversal. The sentence that states that this question has a solution is

\[
\exists abcd \in {\mathbb Z},  \exists k \in {\mathbb Z}, abcd = k\cdot dcba
\]

This could be expressed instead as the disjunction of 9000 statements, or more 
compactly as

\[
\bigvee_{1000\leq abcd \leq 9999}  \exists k \in {\mathbb Z}, abcd = k\cdot dcba.
\]

\end{quote}
\end{itemize}
\end{frame}

\section{A recreational problem}

\begin{frame}{check {\em how} many cases!?!}
\begin{itemize}
\item Let's just go straight over to CoCalc shall we?
\end{itemize}
\end{frame}

\section{Negating quantified sentences}

\begin{frame}{generalising DeMorgan}
\begin{itemize}
\item A generalized version of DeMorgan's laws\pause
\item \rule[-6pt]{0pt}{24pt} $ \lnot ( \forall x \in U, P(x)) \; \cong \; \exists x \in U, \lnot P(x) $\pause \newline
 \rule[-6pt]{0pt}{24pt} $ \lnot ( \exists x \in U, P(x)) \; \cong \; \forall x \in U, \lnot P(x) $\pause
\item Watch what happens when there are multiple quantifiers!\pause
\item \rule[-6pt]{0pt}{24pt} $ \lnot ( \forall x \in U, \, \exists y \in U, \; P(x,y) )$ \pause \newline
\rule[-6pt]{0pt}{24pt} $ \exists x \in U, \; \lnot ( \exists y \in U, \; P(x,y) )$ \pause \newline
\rule[-6pt]{0pt}{24pt} $ \exists x \in U, \; \forall y \in U, \; \lnot P(x,y) $ \pause
\item In brief: switch every quantifier and negate the statement that comes after.
\end{itemize}
\end{frame}

\begin{frame}{generalising DeMorgan}
\begin{itemize}
\item Recall the statement that we use to define the concept of the limit? \pause
\item \rule[-6pt]{0pt}{24pt} $ \forall \epsilon>0, \, \exists \delta>0, \, \forall x, \, (|x-c| < \delta) \implies (|f(x)-l| < \epsilon). $ \pause
\item How do we show that the limit of $f$ as $x$ approaches $c$ is not $l$ ? \pause
\item \rule[-6pt]{0pt}{24pt} $ \lim_{x\rightarrow c} f(x) \, \neq \, l \quad \iff $ \pause
\rule[-6pt]{0pt}{24pt} $ \lnot ( \forall \epsilon>0, \, \exists \delta>0, \, \forall x, \, (|x-c| < \delta) \implies (|f(x)-l| < \epsilon) )$ \pause
\item This is equivalent to \newline
\rule[-6pt]{0pt}{24pt} $ \exists \epsilon>0, \, \forall \delta>0, \, \exists x, \,  \lnot ( (|x-c| < \delta) \implies (|f(x)-l| < \epsilon) )$ \pause
\item We're almost there, we just need to remember how to negate a conditional\textellipsis \pause \newline
\rule[-6pt]{0pt}{24pt} $ \exists \epsilon>0, \, \forall \delta>0, \, \exists x, \,   (|x-c| < \delta ) \land (|f(x)-l| \geq \epsilon) )$
\end{itemize}
\end{frame}

\end{document}

\end{document}
