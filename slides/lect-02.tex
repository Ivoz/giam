%\documentclass[handout,landscape]{beamer}
\documentclass[landscape]{beamer}
\hypersetup{pdfpagemode=FullScreen}
\mode<handout>
{
  \usepackage{pgf}
  \usepackage{pgfpages}

\pgfpagesdeclarelayout{6 on 1 boxed}
{
  \edef\pgfpageoptionheight{\the\paperheight} 
  \edef\pgfpageoptionwidth{\the\paperwidth}
  \edef\pgfpageoptionborder{0pt}
}
{
  \pgfpagesphysicalpageoptions
  {%
    logical pages=6,%
    physical height=\pgfpageoptionheight,%
    physical width=\pgfpageoptionwidth%
  }
  \pgfpageslogicalpageoptions{1}
  {%
    border code=\pgfsetlinewidth{2pt}\pgfstroke,%
    border shrink=\pgfpageoptionborder,%
    resized width=.5\pgfphysicalwidth,%
    resized height=.5\pgfphysicalheight,%
    center=\pgfpoint{.25\pgfphysicalwidth}{.833\pgfphysicalheight}%
  }%
  \pgfpageslogicalpageoptions{2}
  {%
    border code=\pgfsetlinewidth{2pt}\pgfstroke,%
    border shrink=\pgfpageoptionborder,%
    resized width=.5\pgfphysicalwidth,%
    resized height=.5\pgfphysicalheight,%
    center=\pgfpoint{.75\pgfphysicalwidth}{.833\pgfphysicalheight}%
  }%
  \pgfpageslogicalpageoptions{3}
  {%
    border code=\pgfsetlinewidth{2pt}\pgfstroke,%
    border shrink=\pgfpageoptionborder,%
    resized width=.5\pgfphysicalwidth,%
    resized height=.5\pgfphysicalheight,%
    center=\pgfpoint{.25\pgfphysicalwidth}{.5\pgfphysicalheight}%
  }%
  \pgfpageslogicalpageoptions{4}
  {%
    border code=\pgfsetlinewidth{2pt}\pgfstroke,%
    border shrink=\pgfpageoptionborder,%
    resized width=.5\pgfphysicalwidth,%
    resized height=.5\pgfphysicalheight,%
    center=\pgfpoint{.75\pgfphysicalwidth}{.5\pgfphysicalheight}%
  }%
  \pgfpageslogicalpageoptions{5}
  {%
    border code=\pgfsetlinewidth{2pt}\pgfstroke,%
    border shrink=\pgfpageoptionborder,%
    resized width=.5\pgfphysicalwidth,%
    resized height=.5\pgfphysicalheight,%
    center=\pgfpoint{.25\pgfphysicalwidth}{.167\pgfphysicalheight}%
  }%
  \pgfpageslogicalpageoptions{6}
  {%
    border code=\pgfsetlinewidth{2pt}\pgfstroke,%
    border shrink=\pgfpageoptionborder,%
    resized width=.5\pgfphysicalwidth,%
    resized height=.5\pgfphysicalheight,%
    center=\pgfpoint{.75\pgfphysicalwidth}{.167\pgfphysicalheight}%
  }%
}


  \pgfpagesuselayout{6 on 1 boxed}[letterpaper, border shrink=5mm]
  \nofiles
}

\usepackage{listings}
%\lstset{language=TeX}
\usepackage{multimedia}
\usepackage[normalem]{ulem}
\usepackage{amssymb}

%\usecolortheme[named=Purple]{structure} 
%\usetheme{Copenhagen}
\usetheme{Warsaw} 
\usecolortheme{seahorse}
\useoutertheme{infolines} 
%\usetheme[height=7mm]{Rochester} 
%\setbeamertemplate{items}[ball] 
\setbeamertemplate{blocks}[rounded][shadow=true] 
%\setbeamertemplate{navigation symbols}{} 
\author{Joe Fields}
\title{Introduction to Proof} 
%\subtitle{}
\date{Lecture 2}
\institute[SCSU]{ {\tt fieldsj1@southernct.edu} }

\newcommand{\versionNum}{$3.2$\ }

\newboolean{InTextBook}
\setboolean{InTextBook}{false}
\newboolean{InWorkBook}
\setboolean{InWorkBook}{false}
\newboolean{InHints}
\setboolean{InHints}{false}

%When this boolean is true (beginning in Section 5.1) we will use the convention
% that $0 \in \Naturals$.  If it is false we will continue to count $1$ as the smallest
%natural number (thus making Giuseppe Peano spin in his grave...)
 
\newboolean{ZeroInNaturals}

%This boolean is used to distinguish the version where we use $\sim$ rather than $\lnot$

\newboolean{LNotIsSim}

%The values of the last two booleans are set in ``switches.tex''

\setboolean{ZeroInNaturals}{true}
\setboolean{LNotIsSim}{false}


\let\savedlnot\lnot
\ifthenelse{\boolean{LNotIsSim}}{\renewcommand{\lnot}{\sim} }{}

%This command puts different amounts of space depending on whether we are
% in the text, the workbook or the hints & solutions manual. 
\newcommand{\twsvspace}[3]{%
 \ifthenelse{\boolean{InTextBook} }{\vspace{#1}}{%
  \ifthenelse{\boolean{InWorkBook} }{\vspace{#2}}{%
   \ifthenelse{\boolean{InHints} }{\vspace{#3}}{} %
   }%
  }%
 }


\newcommand{\wbvfill}{\ifthenelse{\boolean{InWorkBook}}{\vfill}{}}
\newcommand{\wbitemsep}{\ifthenelse{\boolean{InWorkBook} }{\rule[-24pt]{0pt}{60pt}}{}}
\newcommand{\textbookpagebreak}{\ifthenelse{\boolean{InTextBook}}{\newpage}{}}
\newcommand{\workbookpagebreak}{\ifthenelse{\boolean{InWorkBook}}{\newpage}{}}
\newcommand{\hintspagebreak}{\ifthenelse{\boolean{InHints}}{\newpage}{}}

\newcommand{\hint}[1]{\ifthenelse{\boolean{InHints}}{ {\par \hspace{12pt} \color[rgb]{0,0,1} #1 } }{}}
\newcommand{\inlinehint}[1]{\ifthenelse{\boolean{InHints}}{ { \color[rgb]{0,0,1} #1 } }{}}

\newlength{\cwidth}
\newcommand{\cents}{\settowidth{\cwidth}{c}%
\divide\cwidth by2
\advance\cwidth by-.1pt
c\kern-\cwidth
\vrule width .1pt depth.2ex height1.2ex
\kern\cwidth}

\newcommand{\sageprompt}{ {\tt sage$>$} }
\newcommand{\tab}{\rule{20pt}{0pt}}
\newcommand{\blnk}{\rule{1.5pt}{0pt}\rule{.4pt}{1.2pt}\rule{9pt}{.4pt}\rule{.4pt}{1.2pt}\rule{1.5pt}{0pt}}
\newcommand{\suchthat}{\; \rule[-3pt]{.5pt}{13pt} \;}
\newcommand{\divides}{\!\mid\!}
\newcommand{\tdiv}{\; \mbox{div} \;}
\newcommand{\restrict}[2]{#1 \,\rule[-4pt]{.25pt}{14pt}_{\,#2}}
\newcommand{\lcm}[2]{\mbox{lcm} (#1, #2)}
\renewcommand{\gcd}[2]{\mbox{gcd} (#1, #2)}
\newcommand{\Naturals}{{\mathbb N}}
\newcommand{\Integers}{{\mathbb Z}}
\newcommand{\Znoneg}{{\mathbb Z}^{\mbox{\tiny noneg}}}
\ifthenelse{\boolean{ZeroInNaturals}}{%
  \newcommand{\Zplus}{{\mathbb Z}^+} }{%
  \newcommand{\Zplus}{{\mathbb N}} }
\newcommand{\Enoneg}{{\mathbb E}^{\mbox{\tiny noneg}}}
\newcommand{\Qnoneg}{{\mathbb Q}^{\mbox{\tiny noneg}}}
\newcommand{\Rnoneg}{{\mathbb R}^{\mbox{\tiny noneg}}}
\newcommand{\Rationals}{{\mathbb Q}}
\newcommand{\Reals}{{\mathbb R}}
\newcommand{\Complexes}{{\mathbb C}}
%\newcommand{\F2}{{\mathbb F}_{2}}
\newcommand{\relQ}{\mbox{\textsf Q}}
\newcommand{\relR}{\mbox{\textsf R}}
\newcommand{\nrelR}{\mbox{\raisebox{1pt}{$\not$}\rule{1pt}{0pt}{\textsf R}}}
\newcommand{\relS}{\mbox{\textsf S}}
\newcommand{\relA}{\mbox{\textsf A}}
\newcommand{\Dom}[1]{\mbox{Dom}(#1)}
\newcommand{\Cod}[1]{\mbox{Cod}(#1)}
\newcommand{\Rng}[1]{\mbox{Rng}(#1)}

\DeclareMathOperator\caret{\raisebox{1ex}{$\scriptstyle\wedge$}}

\newtheorem*{defi}{Definition}
\newtheorem*{exer}{Exercise}
\newtheorem{thm}{Theorem}[section]
\newtheorem*{thm*}{Theorem}
\newtheorem{lem}[thm]{Lemma}
\newtheorem*{lem*}{Lemma}
\newtheorem{cor}{Corollary}
\newtheorem{conj}{Conjecture}

\renewenvironment{proof}%
{\begin{quote} \emph{Proof:} }%
{\rule{0pt}{0pt} \newline \rule{0pt}{15pt} \hfill Q.E.D. \end{quote}}


\newcommand{\vs}{\rule{0pt}{12pt}}

\AtBeginSection[]
{
 \begin{frame}{Table of Contents} 
  \tableofcontents[currentsection]
 \end{frame}
}

%%%% SAVE %%%%
%{ %magic to get a full screen image...
%\setbeamertemplate{navigation symbols}{}  % hide navigation buttons 
%\setbeamertemplate{background canvas}{\centerline{\includegraphics 
%	[height=\paperheight]{Cantor_4.jpeg}}}
%\begin{frame}[plain]
%\rule{0pt}{0pt}
%\end{frame} 
%} %end of magic


\begin{document}

\begin{frame}[plain]
  \titlepage
\end{frame}

\section{last time}

\begin{frame}{a bit of finishing up}
\begin{itemize}
\item An additional resource: Overleaf -- \href{https://www.overleaf.com/}{https://www.overleaf.com/} \pause
\item The geometric character of complex multiplication. \pause
\end{itemize}
\end{frame}

\begin{frame}{polar form and complex multiplication}
\begin{itemize}
\item Arg and Mod of a complex number.\pause
\item When multiplying, the Mods get multiplied and the Args get added.\pause
\item Example: $(1+2i) \cdot (2+i)$\pause
\item A truly weird use of $e$:  \newline
a complex number with Mod $r$ and Arg $\theta$ can be written as \textellipsis \pause
\[ re^{i\theta}. \]
\end{itemize}
\end{frame}


\section{definitions}

\begin{frame}{is 0 even?}
\begin{itemize}
\item Is 0 an even number? \pause
\item Well what should the definition of ``even'' be? \pause
\item A number is even if it is exactly twice some other number. \pause
\item Right, so $2\pi \approx 6.283185$ is even because it's twice $\pi$? \pause
\item What's our universe of discourse? \pause
\item Okay, so ``a natural number is even if it is exactly twice some other number.'' \pause
\item Sorry, that makes $3$ even since it is exactly twice $1.5$.
\end{itemize}
\end{frame}

\begin{frame}{yes, 0 is even}
\begin{itemize}
\item Okay, so both the number and the thing it's the double of must be natural numbers? \pause
\item Actually, it's perfectly reasonable to talk about {\em negative} even numbers. \pause
\item An integer $x$ is even if it is exactly twice some integer $y$. \pause
\item Notice that I also dropped the word ``other''?\pause
\item We're used to the double of some number being different - but there's no reason to insist on that! \pause
\item So what is $0$ the double of?
  
\end{itemize}
\end{frame}

\begin{frame}{precise definitions are critical}
\begin{itemize}
\item Notice how we resolved the question of $0$'s evenness, not by thinking much about $0$, but rather, by refining our understanding of what the word ``even'' should mean. \pause
\item Precise definitions are one of the factors that make mathematics so powerful. \pause
\item Notice that there is an insurmountable problem though! \pause 
\item When we define new concepts we must define them in terms of things we already know. \pause
\item And those things must be defined in terms of more primitive concepts. \pause
\item And so on \textellipsis \pause
\end{itemize}
\end{frame}

\begin{frame}{alternative definitions have to be consistent}
\begin{itemize}
\item Often mathematicians develop multiple definitions for the same concept. \pause
\item Some things may be easier to prove using an alternative definition. \pause
\item If two or more definitions exist for some concept we have to prove that they really define the same thing! \pause
\item TFAE
\end{itemize}
\end{frame}

\section{primes}

\begin{frame}{primes}
\begin{itemize}
\item An integer $p$ is prime if it is bigger than 1 and its only divisors are 1 and itself. \pause
\item An integer $p$ is prime if it is bigger than 1 and if it divides a product $xy$, then it must either divide $x$ or $y$ (or both). \pause
\item Those are equivalent definitions.\pause
\item Doing the TFAE proof is a little beyond our scope right now.\pause 
\item The second definition can be adapted to other more abstract settings - Ring theory.
\end{itemize}
\end{frame}

\begin{frame}{finding primes}
\begin{itemize}
\item The Sieve of Eratosthenes. \pause
\item The idea is to ``sieve out'' (i.e. remove) all of the non-primes in successive stages while simultaneously identifying the primes. \pause
\item Begin with the entire set of positive integers $\Zplus$. \pause
\item The first stage is to just cross off 1. \pause
\item In successive stages you circle the smallest non-sieved entry (thus identifying the next prime) and cross off all of its multiples. \pause
\item See Figure 1.1.
\end{itemize}
\end{frame}

\begin{frame}{Eratosthenes}
\begin{itemize}
\item Head Librarian at the great Library of Alexandria. \pause
\item Nickname: Beta \pause
\item Knew the world was round and determined its circumference about 1700 years before Columbus. \pause
\item Further reading: the MacTutor Histoy of Math Archive -- \href{https://mathshistory.st-andrews.ac.uk/}{https://mathshistory.st-andrews.ac.uk/}.
\end{itemize}
\end{frame}

\begin{frame}{more fun with primes}
\begin{itemize}
\item The book contains two interesting (but false) conjectures about primes. \pause
\item Conjecture 1: Whenever $p$ is a prime, $2^p-1$ is also a prime. \pause
\item Conjecture 2: The values of the polynomial $x^2-31x+257$ (at integer inputs) are always prime.\pause
\item Let's investigate these a bit using sage. 
\end{itemize}
\end{frame}

\end{document}
