%\documentclass[handout,landscape]{beamer}
\documentclass[landscape]{beamer}
%\hypersetup{pdfpagemode=FullScreen}
\mode<handout>
{
  \usepackage{pgf}
  \usepackage{pgfpages}

\pgfpagesdeclarelayout{6 on 1 boxed}
{
  \edef\pgfpageoptionheight{\the\paperheight} 
  \edef\pgfpageoptionwidth{\the\paperwidth}
  \edef\pgfpageoptionborder{0pt}
}
{
  \pgfpagesphysicalpageoptions
  {%
    logical pages=6,%
    physical height=\pgfpageoptionheight,%
    physical width=\pgfpageoptionwidth%
  }
  \pgfpageslogicalpageoptions{1}
  {%
    border code=\pgfsetlinewidth{2pt}\pgfstroke,%
    border shrink=\pgfpageoptionborder,%
    resized width=.5\pgfphysicalwidth,%
    resized height=.5\pgfphysicalheight,%
    center=\pgfpoint{.25\pgfphysicalwidth}{.833\pgfphysicalheight}%
  }%
  \pgfpageslogicalpageoptions{2}
  {%
    border code=\pgfsetlinewidth{2pt}\pgfstroke,%
    border shrink=\pgfpageoptionborder,%
    resized width=.5\pgfphysicalwidth,%
    resized height=.5\pgfphysicalheight,%
    center=\pgfpoint{.75\pgfphysicalwidth}{.833\pgfphysicalheight}%
  }%
  \pgfpageslogicalpageoptions{3}
  {%
    border code=\pgfsetlinewidth{2pt}\pgfstroke,%
    border shrink=\pgfpageoptionborder,%
    resized width=.5\pgfphysicalwidth,%
    resized height=.5\pgfphysicalheight,%
    center=\pgfpoint{.25\pgfphysicalwidth}{.5\pgfphysicalheight}%
  }%
  \pgfpageslogicalpageoptions{4}
  {%
    border code=\pgfsetlinewidth{2pt}\pgfstroke,%
    border shrink=\pgfpageoptionborder,%
    resized width=.5\pgfphysicalwidth,%
    resized height=.5\pgfphysicalheight,%
    center=\pgfpoint{.75\pgfphysicalwidth}{.5\pgfphysicalheight}%
  }%
  \pgfpageslogicalpageoptions{5}
  {%
    border code=\pgfsetlinewidth{2pt}\pgfstroke,%
    border shrink=\pgfpageoptionborder,%
    resized width=.5\pgfphysicalwidth,%
    resized height=.5\pgfphysicalheight,%
    center=\pgfpoint{.25\pgfphysicalwidth}{.167\pgfphysicalheight}%
  }%
  \pgfpageslogicalpageoptions{6}
  {%
    border code=\pgfsetlinewidth{2pt}\pgfstroke,%
    border shrink=\pgfpageoptionborder,%
    resized width=.5\pgfphysicalwidth,%
    resized height=.5\pgfphysicalheight,%
    center=\pgfpoint{.75\pgfphysicalwidth}{.167\pgfphysicalheight}%
  }%
}


  \pgfpagesuselayout{6 on 1 boxed}[letterpaper, border shrink=5mm]
  \nofiles
}

\usepackage{listings}
%\lstset{language=TeX}
\usepackage{multimedia}
\usepackage[normalem]{ulem}
%\usepackage{amssymb}

\usepackage{ifthen}

%\usecolortheme[named=Purple]{structure} 
%\usetheme{Copenhagen}
\usetheme{Warsaw} 
\usecolortheme{seahorse}
\useoutertheme{infolines} 
%\usetheme[height=7mm]{Rochester} 
%\setbeamertemplate{items}[ball] 
\setbeamertemplate{blocks}[rounded][shadow=true] 
%\setbeamertemplate{navigation symbols}{} 
\author{Joe Fields}
\title{Introduction to Proof} 
%\subtitle{}
\date{Lecture 19 (GIAM \S 3.6) \newline proving existential statements}
\institute[SCSU]{ {\tt fieldsj1@southernct.edu} }

\newcommand{\versionNum}{$3.2$\ }

\newboolean{InTextBook}
\setboolean{InTextBook}{false}
\newboolean{InWorkBook}
\setboolean{InWorkBook}{false}
\newboolean{InHints}
\setboolean{InHints}{false}

%When this boolean is true (beginning in Section 5.1) we will use the convention
% that $0 \in \Naturals$.  If it is false we will continue to count $1$ as the smallest
%natural number (thus making Giuseppe Peano spin in his grave...)
 
\newboolean{ZeroInNaturals}

%This boolean is used to distinguish the version where we use $\sim$ rather than $\lnot$

\newboolean{LNotIsSim}

%The values of the last two booleans are set in ``switches.tex''

\setboolean{ZeroInNaturals}{true}
\setboolean{LNotIsSim}{false}


\let\savedlnot\lnot
\ifthenelse{\boolean{LNotIsSim}}{\renewcommand{\lnot}{\sim} }{}

%This command puts different amounts of space depending on whether we are
% in the text, the workbook or the hints & solutions manual. 
\newcommand{\twsvspace}[3]{%
 \ifthenelse{\boolean{InTextBook} }{\vspace{#1}}{%
  \ifthenelse{\boolean{InWorkBook} }{\vspace{#2}}{%
   \ifthenelse{\boolean{InHints} }{\vspace{#3}}{} %
   }%
  }%
 }


\newcommand{\wbvfill}{\ifthenelse{\boolean{InWorkBook}}{\vfill}{}}
\newcommand{\wbitemsep}{\ifthenelse{\boolean{InWorkBook} }{\rule[-24pt]{0pt}{60pt}}{}}
\newcommand{\textbookpagebreak}{\ifthenelse{\boolean{InTextBook}}{\newpage}{}}
\newcommand{\workbookpagebreak}{\ifthenelse{\boolean{InWorkBook}}{\newpage}{}}
\newcommand{\hintspagebreak}{\ifthenelse{\boolean{InHints}}{\newpage}{}}

\newcommand{\hint}[1]{\ifthenelse{\boolean{InHints}}{ {\par \hspace{12pt} \color[rgb]{0,0,1} #1 } }{}}
\newcommand{\inlinehint}[1]{\ifthenelse{\boolean{InHints}}{ { \color[rgb]{0,0,1} #1 } }{}}

\newlength{\cwidth}
\newcommand{\cents}{\settowidth{\cwidth}{c}%
\divide\cwidth by2
\advance\cwidth by-.1pt
c\kern-\cwidth
\vrule width .1pt depth.2ex height1.2ex
\kern\cwidth}

\newcommand{\sageprompt}{ {\tt sage$>$} }
\newcommand{\tab}{\rule{20pt}{0pt}}
\newcommand{\blnk}{\rule{1.5pt}{0pt}\rule{.4pt}{1.2pt}\rule{9pt}{.4pt}\rule{.4pt}{1.2pt}\rule{1.5pt}{0pt}}
\newcommand{\suchthat}{\; \rule[-3pt]{.5pt}{13pt} \;}
\newcommand{\divides}{\!\mid\!}
\newcommand{\tdiv}{\; \mbox{div} \;}
\newcommand{\restrict}[2]{#1 \,\rule[-4pt]{.25pt}{14pt}_{\,#2}}
\newcommand{\lcm}[2]{\mbox{lcm} (#1, #2)}
\renewcommand{\gcd}[2]{\mbox{gcd} (#1, #2)}
\newcommand{\Naturals}{{\mathbb N}}
\newcommand{\Integers}{{\mathbb Z}}
\newcommand{\Znoneg}{{\mathbb Z}^{\mbox{\tiny noneg}}}
\ifthenelse{\boolean{ZeroInNaturals}}{%
  \newcommand{\Zplus}{{\mathbb Z}^+} }{%
  \newcommand{\Zplus}{{\mathbb N}} }
\newcommand{\Enoneg}{{\mathbb E}^{\mbox{\tiny noneg}}}
\newcommand{\Qnoneg}{{\mathbb Q}^{\mbox{\tiny noneg}}}
\newcommand{\Rnoneg}{{\mathbb R}^{\mbox{\tiny noneg}}}
\newcommand{\Rationals}{{\mathbb Q}}
\newcommand{\Reals}{{\mathbb R}}
\newcommand{\Complexes}{{\mathbb C}}
%\newcommand{\F2}{{\mathbb F}_{2}}
\newcommand{\relQ}{\mbox{\textsf Q}}
\newcommand{\relR}{\mbox{\textsf R}}
\newcommand{\nrelR}{\mbox{\raisebox{1pt}{$\not$}\rule{1pt}{0pt}{\textsf R}}}
\newcommand{\relS}{\mbox{\textsf S}}
\newcommand{\relA}{\mbox{\textsf A}}
\newcommand{\Dom}[1]{\mbox{Dom}(#1)}
\newcommand{\Cod}[1]{\mbox{Cod}(#1)}
\newcommand{\Rng}[1]{\mbox{Rng}(#1)}

\DeclareMathOperator\caret{\raisebox{1ex}{$\scriptstyle\wedge$}}

\newtheorem*{defi}{Definition}
\newtheorem*{exer}{Exercise}
\newtheorem{thm}{Theorem}[section]
\newtheorem*{thm*}{Theorem}
\newtheorem{lem}[thm]{Lemma}
\newtheorem*{lem*}{Lemma}
\newtheorem{cor}{Corollary}
\newtheorem{conj}{Conjecture}

\renewenvironment{proof}%
{\begin{quote} \emph{Proof:} }%
{\rule{0pt}{0pt} \newline \rule{0pt}{15pt} \hfill Q.E.D. \end{quote}}


\newcommand{\vs}{\rule{0pt}{12pt}}
\newcommand{\notimplies}{\;\not\!\!\!\implies}

%\def\mycommand{\setlength{\abovedisplayskip}{-12pt}%
%\setlength{\belowdisplayskip}{-12pt}%
%\setlength{\abovedisplayshortskip}{0pt}%
%\setlength{\belowdisplayshortskip}{0pt}}

%\let\oldselectfont\selectfont
%\def\selectfont{\oldselectfont\mycommand}

%\mycommand

\AtBeginSection[]
{
 \begin{frame}{Table of Contents} 
  \tableofcontents[currentsection]
 \end{frame}
}

%%%% SAVE %%%%
%{ %magic to get a full screen image...
%\setbeamertemplate{navigation symbols}{}  % hide navigation buttons 
%\setbeamertemplate{background canvas}{\centerline{\includegraphics 
%	[height=\paperheight]{Cantor_4.jpeg}}}
%\begin{frame}[plain]
%\rule{0pt}{0pt}
%\end{frame} 
%} %end of magic


\begin{document}

\begin{frame}[plain]
  \titlepage
\end{frame}

\section{philosophy}

\begin{frame}{intro}
\begin{itemize}
\item Proving an existential statement is the same thing as disproving a universal statement. \pause
\item Disproving an existential statement is the same thing as proving a universal statement. \pause
\item We could probably just skip this section\textellipsis \pause
\item Sometimes there is more emphasis on the existential. \pause
\item Missouri, the ``show me'' state.
\end{itemize}
\end{frame}

\begin{frame}{some quickies}
\begin{itemize}
\item There is an even prime. \pause
\item There is a Fibonacci number that is a perfect square.\pause
\item There is a magic number.
\end{itemize}
\end{frame}

\section{examples}

\begin{frame}{ivt}
\begin{itemize}
\item The Intermediate Value Theorem tells us that if $f(x)$ is a continuous function whose domain contains an interval $[a,b]$ then for every real number $r$ that lies between $f(a)$ and $f(b)$, there is an $x \in (a,b)$ such that $f(x) = r$. \pause
\item A common use of IVT is to prove the existance of a root (a.k.a. zero) for some function $f(x)$. \pause
\item If you can show that $f(a)$ is positive and that $f(b)$ is negative (or vice versa) then IVT guarantees
that at some point $x \in (a,b)$ we will have $f(x) = 0$. \pause
\item For instance, consider $f(x) = x^3 - 2$. All polynomial functions are continuous so $f(x)$ satisfies that hypothesis.  Since $f(1) = -1$ and $f(2) = 6$ it follows from IVT that there is an $x$ in $(1,2)$ where $f(x)=0$. \pause
\item Congratulations, we've just shown that $\sqrt[3]{2}$ lies somewhere in $(1,2)$.
\end{itemize}
\end{frame}

\begin{frame}{a dilemma approach}
\begin{itemize}
\item Question: do you think that an irrational number raised to an irrational power could possibly be rational? \pause
\item It certainly seems unlikely! \pause
\item Weirdly, it is actually true that $\displaystyle \exists \alpha, \beta \in \Reals, \; \alpha \notin \Rationals \, \land \, \beta \notin \Rationals \, \land \, \alpha^\beta \in \Rationals$. \pause
\item The weirdest thing of all is that we know this to be true, but we have no idea what $\alpha$ and $\beta$ are!
\end{itemize}
\end{frame}

\begin{frame}{my favorite proof}
\begin{itemize}
\item Is $\sqrt{2}^{\sqrt{2}}$ rational? \pause
\item If so, then we are done -- let $\alpha = \beta = \sqrt{2}$. \pause
\item If not, then let $\alpha = \sqrt{2}^{\sqrt{2}}$ and let $\beta = \sqrt{2}$. \pause
\item Note that $\alpha^\beta \, = \, \left( \sqrt{2}^{\sqrt{2}} \right)^{\sqrt{2}}  \, = \, \sqrt{2}^{\left(\sqrt{2}\cdot \sqrt{2}\right)} \, = \, \sqrt{2}^2 \, = \, 2$ which is clearly rational.
\end{itemize}
\end{frame}

\section{well-ordering principle}

\begin{frame}{wop}
\begin{itemize}
\item A set is said to have the wop if every subset of it must have a least element. \pause
\item $\Naturals$ has the wop. \pause $\Integers$ doesn't have the wop. \pause 
\item $\Rationals^+$ also doesn't have wop.  \pause Think about the rational numbers in $(0,1)$. \pause Can we name a smallest member?
\end{itemize}
\end{frame}

\begin{frame}{wop 2}
\begin{itemize}
\item Since the naturals have the well-ordering principle, any subset of the naturals will automagically have a least element. \pause Even if we can't currently say what it is! \pause
\item In 2013, Yitang Zhang proved that there are infinitely many pairs of consecutive primes that differ by $70,\!000,\!000$. \pause
\item This tells us that there is some smallest natural number that occurs infinitely often as the gap between consecutive primes. \pause
\item What Zhang did was to prove the so-called bounded gaps conjecture. \pause
\item The state of the art has improved somewhat, currently we know that a gap of $246$ occurs infinitely often among primes.  \pause
\item But we still don't know what the smallest gap that occurs infinitely often is. \pause
\item If it is 2, that's the twin prime conjecture. 
\end{itemize}
\end{frame}

\begin{frame}{vampires}
\begin{itemize}
\item A \emph{vampire number} 
is a 
$2n$ digit number $v$ that factors as $v=xy$
where $x$ and $y$ are $n$ digit numbers and the digits of $v$ are the 
precisely the digits in $x$ and $y$ (counting repetitions) in some order.  The numbers $x$ and $y$
are known as the ``fangs'' of $v$.  To eliminate trivial
cases, both fangs may not end with zeros. \pause
\item In the book it notes that $125460 = 204 \cdot 615 $ is a 6 digit vampire number \pause \newline
-- so the set of vampire numbers is non-empty \pause \newline
-- so wop guarantees that there is a smallest 6 digit vampire number \pause \newline
-- but what is it? \pause
\item Let's code\textellipsis  
\end{itemize}
\end{frame}

\section{unique existance}

\begin{frame}{basics}
\begin{itemize}
\item An exclamation point after the existential quantifier is used to say ``there exists a unique\textellipsis'' \pause
\item Example: \pause \newline
\[ \forall n, d \in \Naturals, \; \exists ! q \in \Naturals, \; \exists ! r \in \Naturals, \; \mbox{such that} \; n=qd+r \; \land \; 0 \leq r < d \] 
\pause
\item A unique existence proof usually has two parts: \pause
\item Show existence. \pause
\item Show that if two such things exist we get a contradiction. \pause
\item So there's a mini proof by contradiction in there.
\end{itemize}
\end{frame}

\begin{frame}{example}
\begin{itemize}
\item The gcd of two natural numbers is unique. \pause
\item The gcd of $a$ and $b$ is the unique smallest positive number in the $\Integers$-module generated by $a$ and $b$. \pause
\item See Figure 3.4 in GIAM
\end{itemize}
\end{frame}


\end{document}
