%\documentclass[handout,landscape]{beamer}
\documentclass[landscape]{beamer}
%\hypersetup{pdfpagemode=FullScreen}
\mode<handout>
{
  \usepackage{pgf}
  \usepackage{pgfpages}

\pgfpagesdeclarelayout{6 on 1 boxed}
{
  \edef\pgfpageoptionheight{\the\paperheight} 
  \edef\pgfpageoptionwidth{\the\paperwidth}
  \edef\pgfpageoptionborder{0pt}
}
{
  \pgfpagesphysicalpageoptions
  {%
    logical pages=6,%
    physical height=\pgfpageoptionheight,%
    physical width=\pgfpageoptionwidth%
  }
  \pgfpageslogicalpageoptions{1}
  {%
    border code=\pgfsetlinewidth{2pt}\pgfstroke,%
    border shrink=\pgfpageoptionborder,%
    resized width=.5\pgfphysicalwidth,%
    resized height=.5\pgfphysicalheight,%
    center=\pgfpoint{.25\pgfphysicalwidth}{.833\pgfphysicalheight}%
  }%
  \pgfpageslogicalpageoptions{2}
  {%
    border code=\pgfsetlinewidth{2pt}\pgfstroke,%
    border shrink=\pgfpageoptionborder,%
    resized width=.5\pgfphysicalwidth,%
    resized height=.5\pgfphysicalheight,%
    center=\pgfpoint{.75\pgfphysicalwidth}{.833\pgfphysicalheight}%
  }%
  \pgfpageslogicalpageoptions{3}
  {%
    border code=\pgfsetlinewidth{2pt}\pgfstroke,%
    border shrink=\pgfpageoptionborder,%
    resized width=.5\pgfphysicalwidth,%
    resized height=.5\pgfphysicalheight,%
    center=\pgfpoint{.25\pgfphysicalwidth}{.5\pgfphysicalheight}%
  }%
  \pgfpageslogicalpageoptions{4}
  {%
    border code=\pgfsetlinewidth{2pt}\pgfstroke,%
    border shrink=\pgfpageoptionborder,%
    resized width=.5\pgfphysicalwidth,%
    resized height=.5\pgfphysicalheight,%
    center=\pgfpoint{.75\pgfphysicalwidth}{.5\pgfphysicalheight}%
  }%
  \pgfpageslogicalpageoptions{5}
  {%
    border code=\pgfsetlinewidth{2pt}\pgfstroke,%
    border shrink=\pgfpageoptionborder,%
    resized width=.5\pgfphysicalwidth,%
    resized height=.5\pgfphysicalheight,%
    center=\pgfpoint{.25\pgfphysicalwidth}{.167\pgfphysicalheight}%
  }%
  \pgfpageslogicalpageoptions{6}
  {%
    border code=\pgfsetlinewidth{2pt}\pgfstroke,%
    border shrink=\pgfpageoptionborder,%
    resized width=.5\pgfphysicalwidth,%
    resized height=.5\pgfphysicalheight,%
    center=\pgfpoint{.75\pgfphysicalwidth}{.167\pgfphysicalheight}%
  }%
}


  \pgfpagesuselayout{6 on 1 boxed}[letterpaper, border shrink=5mm]
  \nofiles
}

\usepackage{listings}
%\lstset{language=TeX}
\usepackage{multimedia}
\usepackage[normalem]{ulem}
\usepackage{amssymb}

%\usecolortheme[named=Purple]{structure} 
%\usetheme{Copenhagen}
\usetheme{Warsaw} 
\usecolortheme{seahorse}
\useoutertheme{infolines} 
%\usetheme[height=7mm]{Rochester} 
%\setbeamertemplate{items}[ball] 
\setbeamertemplate{blocks}[rounded][shadow=true] 
%\setbeamertemplate{navigation symbols}{} 
\author{Joe Fields}
\title{Introduction to Proof} 
%\subtitle{}
\date{Lecture 10 (GIAM \S 2.3)}
\institute[SCSU]{ {\tt fieldsj1@southernct.edu} }

\newcommand{\versionNum}{$3.2$\ }

\newboolean{InTextBook}
\setboolean{InTextBook}{false}
\newboolean{InWorkBook}
\setboolean{InWorkBook}{false}
\newboolean{InHints}
\setboolean{InHints}{false}

%When this boolean is true (beginning in Section 5.1) we will use the convention
% that $0 \in \Naturals$.  If it is false we will continue to count $1$ as the smallest
%natural number (thus making Giuseppe Peano spin in his grave...)
 
\newboolean{ZeroInNaturals}

%This boolean is used to distinguish the version where we use $\sim$ rather than $\lnot$

\newboolean{LNotIsSim}

%The values of the last two booleans are set in ``switches.tex''

\setboolean{ZeroInNaturals}{true}
\setboolean{LNotIsSim}{false}


\let\savedlnot\lnot
\ifthenelse{\boolean{LNotIsSim}}{\renewcommand{\lnot}{\sim} }{}

%This command puts different amounts of space depending on whether we are
% in the text, the workbook or the hints & solutions manual. 
\newcommand{\twsvspace}[3]{%
 \ifthenelse{\boolean{InTextBook} }{\vspace{#1}}{%
  \ifthenelse{\boolean{InWorkBook} }{\vspace{#2}}{%
   \ifthenelse{\boolean{InHints} }{\vspace{#3}}{} %
   }%
  }%
 }


\newcommand{\wbvfill}{\ifthenelse{\boolean{InWorkBook}}{\vfill}{}}
\newcommand{\wbitemsep}{\ifthenelse{\boolean{InWorkBook} }{\rule[-24pt]{0pt}{60pt}}{}}
\newcommand{\textbookpagebreak}{\ifthenelse{\boolean{InTextBook}}{\newpage}{}}
\newcommand{\workbookpagebreak}{\ifthenelse{\boolean{InWorkBook}}{\newpage}{}}
\newcommand{\hintspagebreak}{\ifthenelse{\boolean{InHints}}{\newpage}{}}

\newcommand{\hint}[1]{\ifthenelse{\boolean{InHints}}{ {\par \hspace{12pt} \color[rgb]{0,0,1} #1 } }{}}
\newcommand{\inlinehint}[1]{\ifthenelse{\boolean{InHints}}{ { \color[rgb]{0,0,1} #1 } }{}}

\newlength{\cwidth}
\newcommand{\cents}{\settowidth{\cwidth}{c}%
\divide\cwidth by2
\advance\cwidth by-.1pt
c\kern-\cwidth
\vrule width .1pt depth.2ex height1.2ex
\kern\cwidth}

\newcommand{\sageprompt}{ {\tt sage$>$} }
\newcommand{\tab}{\rule{20pt}{0pt}}
\newcommand{\blnk}{\rule{1.5pt}{0pt}\rule{.4pt}{1.2pt}\rule{9pt}{.4pt}\rule{.4pt}{1.2pt}\rule{1.5pt}{0pt}}
\newcommand{\suchthat}{\; \rule[-3pt]{.5pt}{13pt} \;}
\newcommand{\divides}{\!\mid\!}
\newcommand{\tdiv}{\; \mbox{div} \;}
\newcommand{\restrict}[2]{#1 \,\rule[-4pt]{.25pt}{14pt}_{\,#2}}
\newcommand{\lcm}[2]{\mbox{lcm} (#1, #2)}
\renewcommand{\gcd}[2]{\mbox{gcd} (#1, #2)}
\newcommand{\Naturals}{{\mathbb N}}
\newcommand{\Integers}{{\mathbb Z}}
\newcommand{\Znoneg}{{\mathbb Z}^{\mbox{\tiny noneg}}}
\ifthenelse{\boolean{ZeroInNaturals}}{%
  \newcommand{\Zplus}{{\mathbb Z}^+} }{%
  \newcommand{\Zplus}{{\mathbb N}} }
\newcommand{\Enoneg}{{\mathbb E}^{\mbox{\tiny noneg}}}
\newcommand{\Qnoneg}{{\mathbb Q}^{\mbox{\tiny noneg}}}
\newcommand{\Rnoneg}{{\mathbb R}^{\mbox{\tiny noneg}}}
\newcommand{\Rationals}{{\mathbb Q}}
\newcommand{\Reals}{{\mathbb R}}
\newcommand{\Complexes}{{\mathbb C}}
%\newcommand{\F2}{{\mathbb F}_{2}}
\newcommand{\relQ}{\mbox{\textsf Q}}
\newcommand{\relR}{\mbox{\textsf R}}
\newcommand{\nrelR}{\mbox{\raisebox{1pt}{$\not$}\rule{1pt}{0pt}{\textsf R}}}
\newcommand{\relS}{\mbox{\textsf S}}
\newcommand{\relA}{\mbox{\textsf A}}
\newcommand{\Dom}[1]{\mbox{Dom}(#1)}
\newcommand{\Cod}[1]{\mbox{Cod}(#1)}
\newcommand{\Rng}[1]{\mbox{Rng}(#1)}

\DeclareMathOperator\caret{\raisebox{1ex}{$\scriptstyle\wedge$}}

\newtheorem*{defi}{Definition}
\newtheorem*{exer}{Exercise}
\newtheorem{thm}{Theorem}[section]
\newtheorem*{thm*}{Theorem}
\newtheorem{lem}[thm]{Lemma}
\newtheorem*{lem*}{Lemma}
\newtheorem{cor}{Corollary}
\newtheorem{conj}{Conjecture}

\renewenvironment{proof}%
{\begin{quote} \emph{Proof:} }%
{\rule{0pt}{0pt} \newline \rule{0pt}{15pt} \hfill Q.E.D. \end{quote}}


\newcommand{\vs}{\rule{0pt}{12pt}}

\AtBeginSection[]
{
 \begin{frame}{Table of Contents} 
  \tableofcontents[currentsection]
 \end{frame}
}

%%%% SAVE %%%%
%{ %magic to get a full screen image...
%\setbeamertemplate{navigation symbols}{}  % hide navigation buttons 
%\setbeamertemplate{background canvas}{\centerline{\includegraphics 
%	[height=\paperheight]{Cantor_4.jpeg}}}
%\begin{frame}[plain]
%\rule{0pt}{0pt}
%\end{frame} 
%} %end of magic


\begin{document}

\begin{frame}[plain]
  \titlepage
\end{frame}


\section{basics and notation}

\begin{frame}{logical equivalence}
\begin{itemize}
\item In the last section we said that a conditional and its contrapositive are the same. \pause
\item What exactly does that mean?\pause
\item Literally that in every row of a truth table they have the same value.\pause
\item We call this {\em logical equivalence}.\pause
\item There are two notations: $\iff$ and $\; \cong \;$. \pause
\end{itemize}
\end{frame}

\begin{frame}{more on equivalence}
\begin{itemize}
\item Don't write $(A \implies B) = (\lnot B \implies \lnot A)$. \pause Since `$=$' already has a job. \pause
\item You can either write $(A \implies B) \iff (\lnot B \implies \lnot A)$, \pause (Actually this is probably the most popular notation.) \pause
\item or, $(A \implies B) \; \cong \; (\lnot B \implies \lnot A)$. \pause (We'll use this for the most part.) 
\end{itemize}
\end{frame}

\begin{frame}{truth tables}
\begin{itemize}
\item It turns out that lots of compound sentences are equivalent. \pause
\item In every possible scenario they have the same truth value. \pause
\item Example: $A\lor B$ and $A \lor (\lnot A \land B)$.\pause
\item Make a truth table! \pause
\item You can stop and say they're inequivalent when you first encounter a place where they don't match. \pause
\item If they match in every row, then you've proved that they are equivalent.
\end{itemize}
\end{frame}

\begin{frame}{or maybe not}
\begin{itemize}
\item Creating truth tables to verify a logical equivalence is an instance of something known as an {\em exhaustive proof}. \pause
\item We'll do quite a few of these in the beginning, but they tend to get tedious.\pause
\item In the next section we're going to develop some ``algebraic moves'' that we can use to transform one side of an equivalence into the other. \pause
\item But first\textellipsis
\end{itemize}
\end{frame}

\begin{frame}{Examples}
\begin{itemize}
\item Show that $A \land (A \lor B)$ is equivalent to $(A \land B) \lor A$. \pause
\item Show that $A \land (B \lor \lnot A)$ is equivalent to $(A \land B)$. \pause
\item Show that $\lnot (A \implies B)$ is equivalent to $(A \land \lnot B)$. 
\end{itemize}
\end{frame}

\section{proving equivalences}

\begin{frame}{basic logical equivalences}
\begin{itemize}
\item Instead of truth tables, we'll look at developing some basic ``rules of algebra'' \pause
\item Then we'll transform the LHS of a proposed equivalence into the RHS, using a sequence of approved steps.\pause
\item The basic logical equivalences are in Table~2.2 on page 83 in GIAM.  \pause
\item There's also a link to a single-page handout version in the video description.  
\end{itemize}
\end{frame}

\begin{frame}{commutative and associative laws}
\begin{itemize}
\item We saw these briefly in section 2.1 \pause
\item Both $\land$ and $\lor$ satisfy the commutative property. \pause (So you can change the physical order ``on the page'') \pause
\item Both $\land$ and $\lor$ satisfy the associative property. \pause (So you can re-parenthesize any way you want.) \pause  
\item Note that commutativity is about spatial order and associativity is about temporal order. \pause
\item Linguistics \pause
\item Logic diagrams
\end{itemize}
\end{frame}

\begin{frame}{distributive laws}
\begin{itemize}
\item In ordinary algebra, when does an operation distribute over another? \pause
\item addition, multiplication and exponents. \pause
\item $a(b+c) = ab + ac$ and $(ab)^n = a^n\cdot b^n$, but $(a+b)^n \neq a^n + b^n$ \pause
\item Calculus teachers have a name for the classic mistake: $(x+y)^2 = x^2 + y^2$ \pause \hspace{4ex} The freshman's fallacy. \pause
\item Operations that are 'repeated' versions distribute over the `unrepeated' version. \pause
\end{itemize}
\end{frame}

\begin{frame}{distributive laws in logic}
\begin{itemize}
\item Conjunction distributes over disjunction.\pause
\item Disjunction distributes over conjunction.\pause
\item Note: There must be two {\em different} operations. \pause
\item There's no distribution here:  
\[ A \land (B \land C) \] 
\end{itemize}
\end{frame}

\begin{frame}{DeMorgan's Laws}
\begin{itemize}
\item Thinking about truth tables. \pause
\item A conjunction is true in just one row of a truth table. \pause
\item So the negation of a conjunction is false in just one row. \pause \newline
(And true in the other 3.) \pause
\item So a conjunction's negation is a disjunction. \pause (and {\em vice versa}) \pause
\item This means that the following reasonable looking thing: \pause \newline
$\displaystyle \lnot (A \land B) \quad = \quad \lnot A \land \lnot B $ \pause \hspace{4ex} IS WRONG!!! \pause
\item The correct thing is known as DeMorgan's laws -- you not only ditribute the negation sign, you also change the operator.\pause
\item Algebraic analog.
\end{itemize}
\end{frame}

\begin{frame}{double negatives}
\begin{itemize}
\item $ \lnot(\lnot A) \; \cong \; A$ \pause
\end{itemize}
\end{frame}

\begin{frame}{t and c}
\begin{itemize}
\item  Some new symbols: write $t$ for a statement that is always true,\pause \newline
and write $c$ for a statement that is always false. \pause
\item tautologies and contradictions \pause
\item Complementarity\pause
\item Identity laws\pause
\item Domination
\end{itemize}
\end{frame}

\begin{frame}{wrapping up}
\begin{itemize}
\item Idempotence \pause
\item There's really no such thing as exponentiation in the algebra of Logic.\pause
\item Absorption \pause
\item A weapon of last resort?
\item One thing not on the list that's often needed is the equivalence of a conditional and the related disjunction. \pause
\[ A \implies B \quad \; \cong \; \quad \lnot A \lor B \]
\end{itemize}
\end{frame}

\section{IKK}

\begin{frame}{Smullyan}
\begin{itemize}
\item Raymond Smullyan \pause
\item Origin story of ``Knights and Knaves'' \pause
\item Knights always speak honestly. \pause (They make true statements.) \pause
\item Knaves always speak falsely. \pause
\item We usually don't know who is a Knight and who is a Knave. \pause 
\item So the puzzle is often merely to figure out what sorts of people we're dealing with.
\end{itemize}
\end{frame}

\begin{frame}{Locke and Demosthenes}
\begin{itemize}
\item Locke says, ``Demosthenes is a knave.'' 
\item Demosthenes says ``Locke and I are knights.'' \pause
\item Often we can resolve things by examining what the consequences are of both possibilities for one of the people.\pause
\item Focus on Locke. \pause
\item Or you could focus on Demosthenes.
\end{itemize}
\end{frame}

\end{document}

\end{document}
