\ifdefined\ishandout
  \documentclass[handout,landscape]{beamer} 
\else
  \documentclass[landscape]{beamer}
\fi

%\hypersetup{pdfpagemode=FullScreen} %Enabling this option will cause the slides to go full-screen on opening

\mode<handout>
{
  \usepackage{pgf}
  \usepackage{pgfpages}

\pgfpagesdeclarelayout{6 on 1 boxed}
{
  \edef\pgfpageoptionheight{\the\paperheight} 
  \edef\pgfpageoptionwidth{\the\paperwidth}
  \edef\pgfpageoptionborder{0pt}
}
{
  \pgfpagesphysicalpageoptions
  {%
    logical pages=6,%
    physical height=\pgfpageoptionheight,%
    physical width=\pgfpageoptionwidth%
  }
  \pgfpageslogicalpageoptions{1}
  {%
    border code=\pgfsetlinewidth{1pt}\pgfstroke,%
    border shrink=\pgfpageoptionborder,%
    resized width=.5\pgfphysicalwidth,%
    resized height=.5\pgfphysicalheight,%
    center=\pgfpoint{.25\pgfphysicalwidth}{.833\pgfphysicalheight}%
  }%
  \pgfpageslogicalpageoptions{2}
  {%
    border code=\pgfsetlinewidth{1pt}\pgfstroke,%
    border shrink=\pgfpageoptionborder,%
    resized width=.5\pgfphysicalwidth,%
    resized height=.5\pgfphysicalheight,%
    center=\pgfpoint{.75\pgfphysicalwidth}{.833\pgfphysicalheight}%
  }%
  \pgfpageslogicalpageoptions{3}
  {%
    border code=\pgfsetlinewidth{1pt}\pgfstroke,%
    border shrink=\pgfpageoptionborder,%
    resized width=.5\pgfphysicalwidth,%
    resized height=.5\pgfphysicalheight,%
    center=\pgfpoint{.25\pgfphysicalwidth}{.5\pgfphysicalheight}%
  }%
  \pgfpageslogicalpageoptions{4}
  {%
    border code=\pgfsetlinewidth{1pt}\pgfstroke,%
    border shrink=\pgfpageoptionborder,%
    resized width=.5\pgfphysicalwidth,%
    resized height=.5\pgfphysicalheight,%
    center=\pgfpoint{.75\pgfphysicalwidth}{.5\pgfphysicalheight}%
  }%
  \pgfpageslogicalpageoptions{5}
  {%
    border code=\pgfsetlinewidth{1pt}\pgfstroke,%
    border shrink=\pgfpageoptionborder,%
    resized width=.5\pgfphysicalwidth,%
    resized height=.5\pgfphysicalheight,%
    center=\pgfpoint{.25\pgfphysicalwidth}{.167\pgfphysicalheight}%
  }%
  \pgfpageslogicalpageoptions{6}
  {%
    border code=\pgfsetlinewidth{1pt}\pgfstroke,%
    border shrink=\pgfpageoptionborder,%
    resized width=.5\pgfphysicalwidth,%
    resized height=.5\pgfphysicalheight,%
    center=\pgfpoint{.75\pgfphysicalwidth}{.167\pgfphysicalheight}%
  }%
}


  \pgfpagesuselayout{6 on 1 boxed}[letterpaper, border shrink=5mm]
  \nofiles
}

\usepackage{listings}
\usepackage{multimedia}
\usepackage[normalem]{ulem}
\usepackage{ifthen}
\usepackage{textcomp}

\usetheme{Warsaw} 
\usecolortheme{seahorse}
\useoutertheme{infolines} 

\setbeamertemplate{blocks}[rounded][shadow=true] 

\author{Joe Fields}
\title{Introduction to Proof} 

\date{Lecture 31 (GIAM \S 6.3) \newline equivalence relations}
\institute[SCSU]{ {\tt fieldsj1@southernct.edu} }

\newcommand{\versionNum}{$3.2$\ }

\newboolean{InTextBook}
\setboolean{InTextBook}{false}
\newboolean{InWorkBook}
\setboolean{InWorkBook}{false}
\newboolean{InHints}
\setboolean{InHints}{false}

%When this boolean is true (beginning in Section 5.1) we will use the convention
% that $0 \in \Naturals$.  If it is false we will continue to count $1$ as the smallest
%natural number (thus making Giuseppe Peano spin in his grave...)
 
\newboolean{ZeroInNaturals}

%This boolean is used to distinguish the version where we use $\sim$ rather than $\lnot$

\newboolean{LNotIsSim}

%The values of the last two booleans are set in ``switches.tex''

\setboolean{ZeroInNaturals}{true}
\setboolean{LNotIsSim}{false}


\let\savedlnot\lnot
\ifthenelse{\boolean{LNotIsSim}}{\renewcommand{\lnot}{\sim} }{}

%This command puts different amounts of space depending on whether we are
% in the text, the workbook or the hints & solutions manual. 
\newcommand{\twsvspace}[3]{%
 \ifthenelse{\boolean{InTextBook} }{\vspace{#1}}{%
  \ifthenelse{\boolean{InWorkBook} }{\vspace{#2}}{%
   \ifthenelse{\boolean{InHints} }{\vspace{#3}}{} %
   }%
  }%
 }


\newcommand{\wbvfill}{\ifthenelse{\boolean{InWorkBook}}{\vfill}{}}
\newcommand{\wbitemsep}{\ifthenelse{\boolean{InWorkBook} }{\rule[-24pt]{0pt}{60pt}}{}}
\newcommand{\textbookpagebreak}{\ifthenelse{\boolean{InTextBook}}{\newpage}{}}
\newcommand{\workbookpagebreak}{\ifthenelse{\boolean{InWorkBook}}{\newpage}{}}
\newcommand{\hintspagebreak}{\ifthenelse{\boolean{InHints}}{\newpage}{}}

\newcommand{\hint}[1]{\ifthenelse{\boolean{InHints}}{ {\par \hspace{12pt} \color[rgb]{0,0,1} #1 } }{}}
\newcommand{\inlinehint}[1]{\ifthenelse{\boolean{InHints}}{ { \color[rgb]{0,0,1} #1 } }{}}

\newlength{\cwidth}
\newcommand{\cents}{\settowidth{\cwidth}{c}%
\divide\cwidth by2
\advance\cwidth by-.1pt
c\kern-\cwidth
\vrule width .1pt depth.2ex height1.2ex
\kern\cwidth}

\newcommand{\sageprompt}{ {\tt sage$>$} }
\newcommand{\tab}{\rule{20pt}{0pt}}
\newcommand{\blnk}{\rule{1.5pt}{0pt}\rule{.4pt}{1.2pt}\rule{9pt}{.4pt}\rule{.4pt}{1.2pt}\rule{1.5pt}{0pt}}
\newcommand{\suchthat}{\; \rule[-3pt]{.5pt}{13pt} \;}
\newcommand{\divides}{\!\mid\!}
\newcommand{\tdiv}{\; \mbox{div} \;}
\newcommand{\restrict}[2]{#1 \,\rule[-4pt]{.25pt}{14pt}_{\,#2}}
\newcommand{\lcm}[2]{\mbox{lcm} (#1, #2)}
\renewcommand{\gcd}[2]{\mbox{gcd} (#1, #2)}
\newcommand{\Naturals}{{\mathbb N}}
\newcommand{\Integers}{{\mathbb Z}}
\newcommand{\Znoneg}{{\mathbb Z}^{\mbox{\tiny noneg}}}
\ifthenelse{\boolean{ZeroInNaturals}}{%
  \newcommand{\Zplus}{{\mathbb Z}^+} }{%
  \newcommand{\Zplus}{{\mathbb N}} }
\newcommand{\Enoneg}{{\mathbb E}^{\mbox{\tiny noneg}}}
\newcommand{\Qnoneg}{{\mathbb Q}^{\mbox{\tiny noneg}}}
\newcommand{\Rnoneg}{{\mathbb R}^{\mbox{\tiny noneg}}}
\newcommand{\Rationals}{{\mathbb Q}}
\newcommand{\Reals}{{\mathbb R}}
\newcommand{\Complexes}{{\mathbb C}}
%\newcommand{\F2}{{\mathbb F}_{2}}
\newcommand{\relQ}{\mbox{\textsf Q}}
\newcommand{\relR}{\mbox{\textsf R}}
\newcommand{\nrelR}{\mbox{\raisebox{1pt}{$\not$}\rule{1pt}{0pt}{\textsf R}}}
\newcommand{\relS}{\mbox{\textsf S}}
\newcommand{\relA}{\mbox{\textsf A}}
\newcommand{\Dom}[1]{\mbox{Dom}(#1)}
\newcommand{\Cod}[1]{\mbox{Cod}(#1)}
\newcommand{\Rng}[1]{\mbox{Rng}(#1)}

\DeclareMathOperator\caret{\raisebox{1ex}{$\scriptstyle\wedge$}}

\newtheorem*{defi}{Definition}
\newtheorem*{exer}{Exercise}
\newtheorem{thm}{Theorem}[section]
\newtheorem*{thm*}{Theorem}
\newtheorem{lem}[thm]{Lemma}
\newtheorem*{lem*}{Lemma}
\newtheorem{cor}{Corollary}
\newtheorem{conj}{Conjecture}

\renewenvironment{proof}%
{\begin{quote} \emph{Proof:} }%
{\rule{0pt}{0pt} \newline \rule{0pt}{15pt} \hfill Q.E.D. \end{quote}}


\newcommand{\vs}{\rule{0pt}{11pt}}
\newcommand{\notimplies}{\;\not\!\!\!\implies}
\newcommand{\dx}{\,\mbox{d}x}

\AtBeginSection[]
{
 \begin{frame}{Table of Contents} 
  \tableofcontents[currentsection]
 \end{frame}
}

%%%% SAVE %%%%
%{ %magic to get a full screen image...
%\setbeamertemplate{navigation symbols}{}  % hide navigation buttons 
%\setbeamertemplate{background canvas}{\centerline{\includegraphics 
%	[height=\paperheight]{Cantor_4.jpeg}}}
%\begin{frame}[plain]
%\rule{0pt}{0pt}
%\end{frame} 
%} %end of magic


\begin{document}

\begin{frame}[plain]
  \titlepage
\end{frame}

\section{intro}

\begin{frame}{what is an equivalence relation?}
\begin{itemize}
\item Equals light\pause
\item Usually some feature is highlighted while others are lost. \pause
\item I have dozens of screwdrivers - none of them are equal (they all have slightly different uses) but if I just need a poking device, they are all equivalent.  \pause
\end{itemize}
\end{frame}

\begin{frame}{the official definition}
\begin{itemize}
\item A relation $\relR$ on a set $S$ is an equivalence relation if it is:\pause
\item Reflexive,\pause
\item Symmetric,\pause
\item and Transitive.
\end{itemize}
\end{frame}

\begin{frame}{an example}
\begin{itemize}
\item Consider the relation $\equiv_3$ defined by

\[ x \equiv_3 y \quad \iff \quad x \mod{3} \; = \; y \mod{3} \] \pause
\item Proving that this is reflexive, symmetric and transitive is easy since it is defined using ordinary $=$ (which is reflexive, symmetric and transitive).  \pause
\item Let's do this proof\textellipsis
\end{itemize}
\end{frame}

\begin{frame}{proof}
\begin{thm*}
The relation $\equiv_3$ is an equivalence relation.
\end{thm*} \pause

\begin{quote} \emph{Proof:}

We must show that $\equiv_3$ is reflexive, symmetric and transitive. Note that the domain of $\equiv_3$ is $\Integers$.
\pause

\begin{itemize}
\item[reflexive] Suppose $x$ is an arbitrary integer.  Since $x \pmod{3} \; = \; x \pmod{3}$ (by the reflexive property of equality) it follows from the definition that $x \equiv_3 x$.  Thus, $\equiv_3$ is reflexive. \pause
\item[symmetric] Suppose that $x$ and $y$ are arbitrary integers.  We must show that if $x \equiv_3 y$, then $y \equiv_3 x$, so let's additionally assume that $x \equiv_3 y$.  By the definition of $\equiv_3$, it follows that $x \pmod{3} \; = \; y \pmod{3}$.  By the symmetric property of equality we know that $y \pmod{3} \; = \; x \pmod{3}$, and from {\em that} it follows that $y \equiv_3 x$.  Thus, $\equiv_3$ is symmetric.
\end{itemize}
\end{quote}
\end{frame}

\begin{frame}{more proof}

\begin{quote}
\begin{itemize}
\item[transitive] Suppose that $x$, $y$ and $z$ are arbitrary integers.  We must show that if $x \equiv_3 y$ and $y \equiv_3 z$ then $x \equiv_3 z$. So we make the further assumptions that $x \equiv_3 y$ and $y \equiv_3 z$.  Since $x \equiv_3 y$, it follows that $x \pmod{3} \; = \; y \pmod{3}$.  Similarly, since $y \equiv_3 z$, it follows that $y \pmod{3} \; = \; z \pmod{3}$.  By the transitive property of equality, it now follows that $x \pmod{3} \; = \; z \pmod{3}$, and from that we deduce that $x \equiv_3 z$. Thus, $\equiv_3$ is transitive. \pause
\end{itemize}

Since we have shown that $\equiv_3$ is reflexive, symmetric and transitive, we have proved that it is an equivalence relation.

\rule{0pt}{0pt} \newline \rule{0pt}{15pt} \hfill Q.E.D. 

\end{quote}
\end{frame}

\section{equivalence classes}

\begin{frame}{informal intro}
\begin{itemize}
\item In the example of my extensive tool collection, my screwdrivers all ended up in the ``poking devices'' equivalence class.  My hammers would all end up in the ``smashing devices'' equivalence class, {\em et cetera}. \pause
\item In the $\equiv_3$ example, there is an equivalence class for each possible remainder mod 3. \pause
\[ \overline{0} \; = \; \{0, 3, 6, 9, \ldots \} \]
\[ \overline{1} \; = \; \{1, 4, 7, 10, \ldots \} \]
\[ \overline{2} \; = \; \{2, 5, 8, 11, \ldots \} \]
\pause

\item Equivalence classes contain all the things that can be treated as one (because they are equivalent under the relation). \pause
\end{itemize}
\end{frame}

\begin{frame}{formal definition}
\begin{itemize}
\item Given an equivalence relation $\relR$ on a set $S$, and an element $x \in S$, the {\em equivalence class of $x$} is 

\[ \{ y \in S \suchthat y \relR x \} \]
\pause

\item Notice that (at a minimum) the equivalence class of $x$ will contain $x$ itself. \pause Why? \pause
\item The notation for equivalence classes isn't completely standardized. \pause
\item A popular choice would be $\overline{x}$, but you will often see $x/\relR$ too. \pause
\item Try to keep in mind that $\overline{x}$ refers to an entire {\em set} of things. \pause
\item $x \in S$ but $\overline{x} \subseteq S$
\end{itemize}
\end{frame}

\begin{frame}{one ring to rule them all}
\begin{itemize}
\item Every element of $S$ defines an equivalence class, \pause
\item but every element of an equivalence class defines that very same equivalence class!\pause
\item To refer to the entire collection of equivalence classes we write $S/\relR$. \pause
\item Example: For $\equiv_3$ on $\Integers$. There are three equivalence classes, but there is some wiggle room concerning what we call them:
\pause

\[ \Integers /\!\equiv_3 \quad = \quad \{ \overline{0}, \overline{1}, \overline{2} \} \]

\pause

but also

\[ \Integers / \equiv_3 \; = \; \{ \overline{-1}, \overline{0}, \overline{1} \}. \]

\pause

\item Notice that $\overline{2} \; = \;\overline{-1}$ since they are the same set - all the things that have remainder 2 when divided by 3. 
\end{itemize}
\end{frame}
  
\section{partitions}

\begin{frame}{a motivating example}
\begin{itemize}
\item Let's take a look at the equivalence classes for $\equiv_5$ \pause
\begin{align*}
 \overline{0} \; &= \; \{ \ldots -10, -5, 0, 5, 10, \ldots \} \\
 \overline{1} \; &= \; \{ \ldots \rule{4pt}{0pt} -9, -4, 1, 6, 11, \ldots \} \\
 \overline{2} \; &= \; \{ \ldots \rule{4pt}{0pt}-8, -3, 2, 7, 12, \ldots \} \\
 \overline{3} \; &= \; \{ \ldots \rule{4pt}{0pt}-7, -2, 3, 8, 13, \ldots \} \\
 \overline{4} \; &= \; \{ \ldots \rule{4pt}{0pt}-6, -1, 4, 9, 14, \ldots \} \\
 \end{align*}
 \pause
\item Notice two things: 
\begin{enumerate}
  \item Every integer is in one of these sets.
  \item There is no overlap between them.
\end{enumerate}
\end{itemize}
\end{frame}

\begin{frame}{what is a partition?}
\begin{itemize}
\item A {\em partition of a set $S$} is a set $P$ (whose elements are subsets of $S$) such that \pause
\bigskip

\begin{enumerate}
  \item $ \displaystyle S \; = \; \bigcup_{X \in P} X$ \pause
  \item \rule{0pt}{24pt} $ \displaystyle \forall X, Y \in P, X \neq Y \implies X\cap Y = \emptyset $
\end{enumerate}

\end{itemize}
\end{frame}

\begin{frame}{an example}
\begin{itemize}
\item Consider the relation $\relS$ on $\Reals$ which we use to say that two numbers have the same sign. \pause
\bigskip
\item The equivalence classes for this relation are $\Reals^+$, $\Reals^-$, and $\{0\}$. \pause
\item The partition of the real numbers into positive, negative and zero is the basic idea of the trichotomy property.
\end{itemize}
\end{frame}

\begin{frame}{another example}
\begin{itemize}
\item Graph isomorphism. \pause
\bigskip
\item Graphs (this is a new usage of an old word -- we don't mean $y=x^2$) \pause \newline
are a mathematical model for things that have connections between them. \pause
\item A graph consists of two sets $V$ and $E$ -- known respectively as vertices and edges. \pause
\item There is also a relation from $E$ to $V$ that connects an edge to its two endpoints. \pause
\item Suppose $V = \{a, b\}$, $E = \{e \}$ and the edge-endpoint relation is $\{ (e,a), (e,b) \}$ \pause
\item That was a very complicated way to describe a pretty simple object.


\end{itemize}
\end{frame}

\begin{frame}{continued}
\begin{itemize}
\item Two graph are isomorphic when there is a mapping that takes vertices to vertices, edges to edges and preserves the edge-endpoint relation. \pause
\bigskip
\item This is much easier done than said! \pause
\item Etymology -- isomorphic means ``equal shape'' \pause
\item Having ``equal shapes'' determine a set of equivalence classes (although in this instance they might better be called isomorphism classes). \pause
\item Examples of isomorphism classes: $K_n$, $C_n$, $P_n$, $W_n$ (draw pictures).

\end{itemize}
\end{frame}

\end{document}
