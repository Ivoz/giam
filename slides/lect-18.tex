%\documentclass[handout,landscape]{beamer}
\documentclass[landscape]{beamer}
%\hypersetup{pdfpagemode=FullScreen}
\mode<handout>
{
  \usepackage{pgf}
  \usepackage{pgfpages}

\pgfpagesdeclarelayout{6 on 1 boxed}
{
  \edef\pgfpageoptionheight{\the\paperheight} 
  \edef\pgfpageoptionwidth{\the\paperwidth}
  \edef\pgfpageoptionborder{0pt}
}
{
  \pgfpagesphysicalpageoptions
  {%
    logical pages=6,%
    physical height=\pgfpageoptionheight,%
    physical width=\pgfpageoptionwidth%
  }
  \pgfpageslogicalpageoptions{1}
  {%
    border code=\pgfsetlinewidth{2pt}\pgfstroke,%
    border shrink=\pgfpageoptionborder,%
    resized width=.5\pgfphysicalwidth,%
    resized height=.5\pgfphysicalheight,%
    center=\pgfpoint{.25\pgfphysicalwidth}{.833\pgfphysicalheight}%
  }%
  \pgfpageslogicalpageoptions{2}
  {%
    border code=\pgfsetlinewidth{2pt}\pgfstroke,%
    border shrink=\pgfpageoptionborder,%
    resized width=.5\pgfphysicalwidth,%
    resized height=.5\pgfphysicalheight,%
    center=\pgfpoint{.75\pgfphysicalwidth}{.833\pgfphysicalheight}%
  }%
  \pgfpageslogicalpageoptions{3}
  {%
    border code=\pgfsetlinewidth{2pt}\pgfstroke,%
    border shrink=\pgfpageoptionborder,%
    resized width=.5\pgfphysicalwidth,%
    resized height=.5\pgfphysicalheight,%
    center=\pgfpoint{.25\pgfphysicalwidth}{.5\pgfphysicalheight}%
  }%
  \pgfpageslogicalpageoptions{4}
  {%
    border code=\pgfsetlinewidth{2pt}\pgfstroke,%
    border shrink=\pgfpageoptionborder,%
    resized width=.5\pgfphysicalwidth,%
    resized height=.5\pgfphysicalheight,%
    center=\pgfpoint{.75\pgfphysicalwidth}{.5\pgfphysicalheight}%
  }%
  \pgfpageslogicalpageoptions{5}
  {%
    border code=\pgfsetlinewidth{2pt}\pgfstroke,%
    border shrink=\pgfpageoptionborder,%
    resized width=.5\pgfphysicalwidth,%
    resized height=.5\pgfphysicalheight,%
    center=\pgfpoint{.25\pgfphysicalwidth}{.167\pgfphysicalheight}%
  }%
  \pgfpageslogicalpageoptions{6}
  {%
    border code=\pgfsetlinewidth{2pt}\pgfstroke,%
    border shrink=\pgfpageoptionborder,%
    resized width=.5\pgfphysicalwidth,%
    resized height=.5\pgfphysicalheight,%
    center=\pgfpoint{.75\pgfphysicalwidth}{.167\pgfphysicalheight}%
  }%
}


  \pgfpagesuselayout{6 on 1 boxed}[letterpaper, border shrink=5mm]
  \nofiles
}

\usepackage{listings}
%\lstset{language=TeX}
\usepackage{multimedia}
\usepackage[normalem]{ulem}
%\usepackage{amssymb}

\usepackage{ifthen}

%\usecolortheme[named=Purple]{structure} 
%\usetheme{Copenhagen}
\usetheme{Warsaw} 
\usecolortheme{seahorse}
\useoutertheme{infolines} 
%\usetheme[height=7mm]{Rochester} 
%\setbeamertemplate{items}[ball] 
\setbeamertemplate{blocks}[rounded][shadow=true] 
%\setbeamertemplate{navigation symbols}{} 
\author{Joe Fields}
\title{Introduction to Proof} 
%\subtitle{}
\date{Lecture 18 (GIAM \S 3.5)}
\institute[SCSU]{ {\tt fieldsj1@southernct.edu} }

\newcommand{\versionNum}{$3.2$\ }

\newboolean{InTextBook}
\setboolean{InTextBook}{false}
\newboolean{InWorkBook}
\setboolean{InWorkBook}{false}
\newboolean{InHints}
\setboolean{InHints}{false}

%When this boolean is true (beginning in Section 5.1) we will use the convention
% that $0 \in \Naturals$.  If it is false we will continue to count $1$ as the smallest
%natural number (thus making Giuseppe Peano spin in his grave...)
 
\newboolean{ZeroInNaturals}

%This boolean is used to distinguish the version where we use $\sim$ rather than $\lnot$

\newboolean{LNotIsSim}

%The values of the last two booleans are set in ``switches.tex''

\setboolean{ZeroInNaturals}{true}
\setboolean{LNotIsSim}{false}


\let\savedlnot\lnot
\ifthenelse{\boolean{LNotIsSim}}{\renewcommand{\lnot}{\sim} }{}

%This command puts different amounts of space depending on whether we are
% in the text, the workbook or the hints & solutions manual. 
\newcommand{\twsvspace}[3]{%
 \ifthenelse{\boolean{InTextBook} }{\vspace{#1}}{%
  \ifthenelse{\boolean{InWorkBook} }{\vspace{#2}}{%
   \ifthenelse{\boolean{InHints} }{\vspace{#3}}{} %
   }%
  }%
 }


\newcommand{\wbvfill}{\ifthenelse{\boolean{InWorkBook}}{\vfill}{}}
\newcommand{\wbitemsep}{\ifthenelse{\boolean{InWorkBook} }{\rule[-24pt]{0pt}{60pt}}{}}
\newcommand{\textbookpagebreak}{\ifthenelse{\boolean{InTextBook}}{\newpage}{}}
\newcommand{\workbookpagebreak}{\ifthenelse{\boolean{InWorkBook}}{\newpage}{}}
\newcommand{\hintspagebreak}{\ifthenelse{\boolean{InHints}}{\newpage}{}}

\newcommand{\hint}[1]{\ifthenelse{\boolean{InHints}}{ {\par \hspace{12pt} \color[rgb]{0,0,1} #1 } }{}}
\newcommand{\inlinehint}[1]{\ifthenelse{\boolean{InHints}}{ { \color[rgb]{0,0,1} #1 } }{}}

\newlength{\cwidth}
\newcommand{\cents}{\settowidth{\cwidth}{c}%
\divide\cwidth by2
\advance\cwidth by-.1pt
c\kern-\cwidth
\vrule width .1pt depth.2ex height1.2ex
\kern\cwidth}

\newcommand{\sageprompt}{ {\tt sage$>$} }
\newcommand{\tab}{\rule{20pt}{0pt}}
\newcommand{\blnk}{\rule{1.5pt}{0pt}\rule{.4pt}{1.2pt}\rule{9pt}{.4pt}\rule{.4pt}{1.2pt}\rule{1.5pt}{0pt}}
\newcommand{\suchthat}{\; \rule[-3pt]{.5pt}{13pt} \;}
\newcommand{\divides}{\!\mid\!}
\newcommand{\tdiv}{\; \mbox{div} \;}
\newcommand{\restrict}[2]{#1 \,\rule[-4pt]{.25pt}{14pt}_{\,#2}}
\newcommand{\lcm}[2]{\mbox{lcm} (#1, #2)}
\renewcommand{\gcd}[2]{\mbox{gcd} (#1, #2)}
\newcommand{\Naturals}{{\mathbb N}}
\newcommand{\Integers}{{\mathbb Z}}
\newcommand{\Znoneg}{{\mathbb Z}^{\mbox{\tiny noneg}}}
\ifthenelse{\boolean{ZeroInNaturals}}{%
  \newcommand{\Zplus}{{\mathbb Z}^+} }{%
  \newcommand{\Zplus}{{\mathbb N}} }
\newcommand{\Enoneg}{{\mathbb E}^{\mbox{\tiny noneg}}}
\newcommand{\Qnoneg}{{\mathbb Q}^{\mbox{\tiny noneg}}}
\newcommand{\Rnoneg}{{\mathbb R}^{\mbox{\tiny noneg}}}
\newcommand{\Rationals}{{\mathbb Q}}
\newcommand{\Reals}{{\mathbb R}}
\newcommand{\Complexes}{{\mathbb C}}
%\newcommand{\F2}{{\mathbb F}_{2}}
\newcommand{\relQ}{\mbox{\textsf Q}}
\newcommand{\relR}{\mbox{\textsf R}}
\newcommand{\nrelR}{\mbox{\raisebox{1pt}{$\not$}\rule{1pt}{0pt}{\textsf R}}}
\newcommand{\relS}{\mbox{\textsf S}}
\newcommand{\relA}{\mbox{\textsf A}}
\newcommand{\Dom}[1]{\mbox{Dom}(#1)}
\newcommand{\Cod}[1]{\mbox{Cod}(#1)}
\newcommand{\Rng}[1]{\mbox{Rng}(#1)}

\DeclareMathOperator\caret{\raisebox{1ex}{$\scriptstyle\wedge$}}

\newtheorem*{defi}{Definition}
\newtheorem*{exer}{Exercise}
\newtheorem{thm}{Theorem}[section]
\newtheorem*{thm*}{Theorem}
\newtheorem{lem}[thm]{Lemma}
\newtheorem*{lem*}{Lemma}
\newtheorem{cor}{Corollary}
\newtheorem{conj}{Conjecture}

\renewenvironment{proof}%
{\begin{quote} \emph{Proof:} }%
{\rule{0pt}{0pt} \newline \rule{0pt}{15pt} \hfill Q.E.D. \end{quote}}


\newcommand{\vs}{\rule{0pt}{12pt}}
\newcommand{\notimplies}{\;\not\!\!\!\implies}

%\def\mycommand{\setlength{\abovedisplayskip}{-12pt}%
%\setlength{\belowdisplayskip}{-12pt}%
%\setlength{\abovedisplayshortskip}{0pt}%
%\setlength{\belowdisplayshortskip}{0pt}}

%\let\oldselectfont\selectfont
%\def\selectfont{\oldselectfont\mycommand}

%\mycommand

\AtBeginSection[]
{
 \begin{frame}{Table of Contents} 
  \tableofcontents[currentsection]
 \end{frame}
}

%%%% SAVE %%%%
%{ %magic to get a full screen image...
%\setbeamertemplate{navigation symbols}{}  % hide navigation buttons 
%\setbeamertemplate{background canvas}{\centerline{\includegraphics 
%	[height=\paperheight]{Cantor_4.jpeg}}}
%\begin{frame}[plain]
%\rule{0pt}{0pt}
%\end{frame} 
%} %end of magic


\begin{document}

\begin{frame}[plain]
  \titlepage
\end{frame}

\section{intro}

\begin{frame}{intro}
\begin{itemize}
\item Exhaustion vs. Cases \pause
\item Exhaustion means literally check that the statement is true for {\em every} element of the universe of discourse. \pause
\item This is impossible if the universe is infinite. \pause
\item Cases means divide the universe up into a finite number of sets (which may be infinite themselves) and essentially do a seperate proof for each.\pause
\item Tons of things can be proved by using the partition of $\Integers$ into even and odd.
\end{itemize}
\end{frame}

\begin{frame}{introducing tautologies}
\begin{itemize}
\item Let's examine a truth table for the statement $A \implies t$. \pause

\begin{center}
\begin{tabular}{c|c|c}
$A$  & $t$ & $A \implies t$ \\ \hline
T & T & T \\
$\phi$ & T & T \\
\end{tabular}
\end{center}
\pause

\item This means that {\em any} statement implies a tautology. \pause
\item {\em That} means that you are always free to create a new line in a proof if it is tautological! 
\end{itemize}
\end{frame}

\begin{frame}{dilemma}
\begin{itemize}
\item Recall the rule of inference known as `constructive dilemma.' \pause

\begin{center}
\begin{tabular}{cl}
 & $A \implies B$ \\
 & $C \implies D$ \\ 
 & $A \lor C$ \\ \hline
$\therefore$ & $B \lor D$ \\
\end{tabular}
\end{center}
\pause

\item A special case of this occurs when $C = \lnot A$, and $B$ and $D$ are equal. \pause

\begin{center}
\begin{tabular}{cl}
 & $A \implies B$ \\
 & $\lnot A \implies B$ \\ 
 & $A \lor \lnot A$ \\ \hline
$\therefore$ & $B$ \\
\end{tabular}
\end{center}
\pause

\item The line that says $A \lor \lnot A$ is a tautology. \pause You can add such a thing at any point in a proof.
\end{itemize}
\end{frame}

\section{some dilemmas}

\begin{frame}{parity}
\begin{itemize}
\item $\displaystyle \forall x \in \Integers, \; x^2+x \; \mbox{is even}$. \pause
\item Let's unveil a proof step by step:\pause
\begin{itemize}
\item Suppose $x$ is a particular but arbitrarily chosen integer. \pause
\item Note that $x$ is either even or odd. \pause \only<5>{\tiny (that was me introducing a tautology)} \pause \pause
\item When $x$ is even, $x^2 + x$ is the sum of two even numbers, \newline \rule{12pt}{0pt} hence, it is even. \pause
\item When $x$ is odd, $x^2 + x$ is the sum of two odd numbers, \newline \rule{12pt}{0pt} hence it is even.\pause
\item Therefore, $x^2+x$ is even.\pause
\item \hfill Q.E.D. 
\end{itemize}
\end{itemize}
\end{frame}

\begin{frame}{trichotomy}
\begin{itemize}
\item Not too long ago we needed the fact that $\displaystyle \forall x \in \Reals, \; x^2 \geq 0$. \pause
\item Here we go again: \pause
\begin{itemize}
\item Suppose $x$ is a particular but arbitrarily chosen real number. \pause
\item Note that either $x\geq 0$ or $x<0$. \pause
\item When $x \geq 0$ then $x^2 \geq 0$. \pause
\item When $x < 0$ then $x^2 > 0$. \pause \newline \rule{12pt}{0pt} (Since the product of two negative numbers is positive.) \pause
\item Therefore, $x^2 \geq 0$. \hfill Q.E.D. \pause
\end{itemize}
\item This proof could have used the partition of $\Reals$ into three parts: positive, negative and zero. \pause
\item That is known as the `trichotomy property.' \pause
\item The $x^2 > 0$ vs. $x^2 \geq 0$ issue.
\end{itemize}
\end{frame}

\section{cases}

\begin{frame}{squares mod 4}
\begin{itemize}
\item The value of a square, reduced mod 4, is never $2$ or $3$. \pause
\item $\displaystyle \forall n \in \Integers, \; n^2 \mod 4 \, = \, 0 \quad \mbox{or} \quad n^2 \mod 4 \, = \, 1$ \pause
\item There are two equivalent ways to talk about the cases: \pause
\begin{itemize} 
\item The value of $n \mod 4$ is in $\{0,1,2,3\}$. \pause
\item $n$ is of the form $4q+k$ where $k$ is in $\{0,1,2,3\}$.\pause
\end{itemize}
\end{itemize}
\begin{thm*}
$\displaystyle \forall n \in \Integers, \exists q \in \Integers, \; n^2 = 4q \; \lor \; n^2 = 4q+1$
\end{thm*}

\end{frame}

\begin{frame}{4 cases}

{\em Proof:} Consider the 4 cases we obtain from the quotient remainder theorem. \pause
\begin{enumerate}
\item ($n=4k$) When $n=4k$ we have $n^2 = 16k^2 = 4(4k^2)$. Since $4k^2$ is an integer, we have $n^2=4q$ with $q = 4k^2$. \pause
\item ($n=4k+1$) When $n=4k+1$ we have $n^2 = 16k^2 + 8k + 1 = 4(4k^2+2k) +1 $.  Since $4k^2+2k$ is an integer, we have $n^2=4q+1$ with $q = 4k^2+2k$. \pause
\item ($n=4k+2$) When $n=4k+2$ we have $n^2 = 16k^2 + 16k + 4 = 4(4k^2+4k+1) $.  Since $4k^2+4k+1$ is an integer, we have $n^2=4q$ with $q = 4k^2+4k+1$. \pause
\item ($n=4k+3$) When $n=4k+3$ we have $n^2 = 16k^2 + 24k + 9 = 4(4k^2+6k+2)+1 $.  Since $4k^2+6k+2$ is an integer, we have $n^2=4q+1$ with $q = 4k^2+6k+2$. \pause
\end{enumerate}
Since these cases are exhaustive, and the desired conclusion holds in each of them, the proof is complete.
\hfill Q.E.D

\end{frame}

\begin{frame}{more succinctly}
\begin{thm*}
$\displaystyle \forall n \in \Integers, \exists q \in \Integers, \; n^2 = 4q \; \lor \; n^2 = 4q+1$
\end{thm*}
\pause
\begin{proof} 
Consider the 2 cases: $n$ is even, or, $n$ is odd.\pause
\begin{enumerate}
\item ($n$ is even) When $n$ is even there is an integer $k$ such that $n=2k$. Then $n^2=(2k)^2=4k^2$ so $n^2$ is of the form $4q$ where $q=k^2$. \pause
\item ($n$ is odd) When $n$ is odd there is an integer $k$ such that $n=2k+1$. Then $n^2=(2k+1)^2=4k^2+4k+1$ so $n^2$ is of the form $4q$ where $q=k^2+k$. \pause
\end{enumerate}
Since these cases are exhaustive, and the desired conclusion holds in each of them, the proof is complete.
\end{proof}
\end{frame}

\begin{frame}{notes}
\begin{itemize}
\item It's okay to re-use a variable in the cases. \pause
\item Each case should stand as a seperate argument. \pause
\item Definitely try to find an approach that uses the least possible number of cases. \pause
\item ``Everything should be made as simple as possible, but no simpler.'' \pause
\item ``The other cases are similar.''
\end{itemize}
\end{frame}


\section{some exhaustion}

\begin{frame}{intro}
\begin{itemize}
\item Exhaustion can be natural or artificial. \pause
\item When the universe of discourse is actually finite, I call that natural. \pause
\item By `artificial' I mean situtations where the universe of discourse is not finite, but some statement has been 
verified up to a certain point. \pause
\end{itemize}
\end{frame}

\begin{frame}{four 4's}
\begin{itemize}
\item This is a classic puzzle that dates to the 19th century. \pause
\item Write four 4's in a row and place mathematical operations between them and attempt to produce as many natural numbers as you can. \pause
\item It turns out that if you allow parentheses and $+$, $-$, $\times$ and $\div$ you can get everything from $0$ 
to $9$. \pause
\item $10$ and $11$ require us to allow $\sqrt{\phantom{x}}$ and factorials, respectively.
\item There's a nice Numberphile video:  \href{https://youtu.be/Noo4lN-vSvw}{https://youtu.be/Noo4lN-vSvw}
\end{itemize}
\end{frame}

\begin{frame}{proof}
\begin{itemize}
\item So here's an exhaustive proof that using just parentheses and $+$, $-$, $\times$ and $\div$, the four 4's puzzle is solvable from $0$ to $9$: \pause

\begin{align*}
 0  & =  4 - 4 + 4 - 4 &  \\
 1  & =  4 \div 4 + 4 - 4 & \\
 2  & =  4 - (4 + 4) \div 4  & \\
 3  & =  (4 \times 4 - 4) \div 4 & \\
 4  & =  4 \times (4 - 4) + 4 & \\
 5  & =  (4 + 4 \times 4) \div 4 & \\
 6  & =  (4 + 4) \div 4 + 4  & \\
 7  & =  4 + 4 - 4 \div 4  & \\
 8  & =  4 \div 4 \times (4 + 4) &  \\
 9  & =  4 \div 4 + 4 + 4  & \\
\end{align*}

\end{itemize}
\end{frame}

\begin{frame}{comments}
\begin{itemize}
\item If we add $\sqrt{\phantom{x}}$ and $\log$ to the allowable operations you can get {\em every} natural number! \pause
\item See that Numberphile video. \pause
\item So I guess that was the `artificial' sort of proof by exhaustion. \pause
\end{itemize}
\end{frame}

\begin{frame}{Goldbach's conjecture}
\begin{itemize}
\item Actually, Euler's strengthening of Goldbach's conjecture. \pause
\item Every even number greater than 2 may be written as the sum of two primes. \pause
\item The statement has been verified for every even number from $4$ to $4 \times 10^{18}$. \pause
\item Goldbach's comet. \href{https://en.wikipedia.org/wiki/Goldbach's_comet}{wikipedia.org}
\end{itemize}
\end{frame}

\begin{frame}{not too exhausting}
\begin{itemize}
\item So, we can verify Goldbach's conjecture for $4 \leq 2n \leq 20$ exhaustively. \pause
\item This is one of those `artificial' exhaustive proofs; we're just arbitrarily choosing an upper limit of $20$. \pause

\begin{align*}
 4  & =  2+2 &  \\
 6  & =  3+3 & \\
 8  & =  3+5 & \\
 10  & = 3+7 & \\
 12 & =  5+7 & \\
 14  & = 3+11 & \\
 16  & = 3+13 & \\
 18  & = 5+13 & \\
 20 & =  3+17 &  \\
\end{align*}

\end{itemize}
\end{frame}

\begin{frame}{graph pebbling}
\begin{itemize}
\item Graph pebbling questions give an example of where the universe of discourse is a legit finite set. \pause
\item Proving a graph pebbling number can be done by exhaustion. \pause
\end{itemize}
\end{frame}

\end{document}
