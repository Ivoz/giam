%\documentclass[handout,landscape]{beamer}
\documentclass[landscape]{beamer}
\hypersetup{pdfpagemode=FullScreen}
\mode<handout>
{
  \usepackage{pgf}
  \usepackage{pgfpages}

\pgfpagesdeclarelayout{6 on 1 boxed}
{
  \edef\pgfpageoptionheight{\the\paperheight} 
  \edef\pgfpageoptionwidth{\the\paperwidth}
  \edef\pgfpageoptionborder{0pt}
}
{
  \pgfpagesphysicalpageoptions
  {%
    logical pages=6,%
    physical height=\pgfpageoptionheight,%
    physical width=\pgfpageoptionwidth%
  }
  \pgfpageslogicalpageoptions{1}
  {%
    border code=\pgfsetlinewidth{2pt}\pgfstroke,%
    border shrink=\pgfpageoptionborder,%
    resized width=.5\pgfphysicalwidth,%
    resized height=.5\pgfphysicalheight,%
    center=\pgfpoint{.25\pgfphysicalwidth}{.833\pgfphysicalheight}%
  }%
  \pgfpageslogicalpageoptions{2}
  {%
    border code=\pgfsetlinewidth{2pt}\pgfstroke,%
    border shrink=\pgfpageoptionborder,%
    resized width=.5\pgfphysicalwidth,%
    resized height=.5\pgfphysicalheight,%
    center=\pgfpoint{.75\pgfphysicalwidth}{.833\pgfphysicalheight}%
  }%
  \pgfpageslogicalpageoptions{3}
  {%
    border code=\pgfsetlinewidth{2pt}\pgfstroke,%
    border shrink=\pgfpageoptionborder,%
    resized width=.5\pgfphysicalwidth,%
    resized height=.5\pgfphysicalheight,%
    center=\pgfpoint{.25\pgfphysicalwidth}{.5\pgfphysicalheight}%
  }%
  \pgfpageslogicalpageoptions{4}
  {%
    border code=\pgfsetlinewidth{2pt}\pgfstroke,%
    border shrink=\pgfpageoptionborder,%
    resized width=.5\pgfphysicalwidth,%
    resized height=.5\pgfphysicalheight,%
    center=\pgfpoint{.75\pgfphysicalwidth}{.5\pgfphysicalheight}%
  }%
  \pgfpageslogicalpageoptions{5}
  {%
    border code=\pgfsetlinewidth{2pt}\pgfstroke,%
    border shrink=\pgfpageoptionborder,%
    resized width=.5\pgfphysicalwidth,%
    resized height=.5\pgfphysicalheight,%
    center=\pgfpoint{.25\pgfphysicalwidth}{.167\pgfphysicalheight}%
  }%
  \pgfpageslogicalpageoptions{6}
  {%
    border code=\pgfsetlinewidth{2pt}\pgfstroke,%
    border shrink=\pgfpageoptionborder,%
    resized width=.5\pgfphysicalwidth,%
    resized height=.5\pgfphysicalheight,%
    center=\pgfpoint{.75\pgfphysicalwidth}{.167\pgfphysicalheight}%
  }%
}


  \pgfpagesuselayout{6 on 1 boxed}[letterpaper, border shrink=5mm]
  \nofiles
}

\usepackage{listings}
%\lstset{language=TeX}
\usepackage{multimedia}
\usepackage[normalem]{ulem}
\usepackage{amssymb}

%\usecolortheme[named=Purple]{structure} 
%\usetheme{Copenhagen}
\usetheme{Warsaw} 
\usecolortheme{seahorse}
\useoutertheme{infolines} 
%\usetheme[height=7mm]{Rochester} 
%\setbeamertemplate{items}[ball] 
\setbeamertemplate{blocks}[rounded][shadow=true] 
%\setbeamertemplate{navigation symbols}{} 
\author{Joe Fields}
\title{Introduction to Proof} 
%\subtitle{}
\date{Lecture 4}
\institute[SCSU]{ {\tt fieldsj1@southernct.edu} }

\newcommand{\versionNum}{$3.2$\ }

\newboolean{InTextBook}
\setboolean{InTextBook}{false}
\newboolean{InWorkBook}
\setboolean{InWorkBook}{false}
\newboolean{InHints}
\setboolean{InHints}{false}

%When this boolean is true (beginning in Section 5.1) we will use the convention
% that $0 \in \Naturals$.  If it is false we will continue to count $1$ as the smallest
%natural number (thus making Giuseppe Peano spin in his grave...)
 
\newboolean{ZeroInNaturals}

%This boolean is used to distinguish the version where we use $\sim$ rather than $\lnot$

\newboolean{LNotIsSim}

%The values of the last two booleans are set in ``switches.tex''

\setboolean{ZeroInNaturals}{true}
\setboolean{LNotIsSim}{false}


\let\savedlnot\lnot
\ifthenelse{\boolean{LNotIsSim}}{\renewcommand{\lnot}{\sim} }{}

%This command puts different amounts of space depending on whether we are
% in the text, the workbook or the hints & solutions manual. 
\newcommand{\twsvspace}[3]{%
 \ifthenelse{\boolean{InTextBook} }{\vspace{#1}}{%
  \ifthenelse{\boolean{InWorkBook} }{\vspace{#2}}{%
   \ifthenelse{\boolean{InHints} }{\vspace{#3}}{} %
   }%
  }%
 }


\newcommand{\wbvfill}{\ifthenelse{\boolean{InWorkBook}}{\vfill}{}}
\newcommand{\wbitemsep}{\ifthenelse{\boolean{InWorkBook} }{\rule[-24pt]{0pt}{60pt}}{}}
\newcommand{\textbookpagebreak}{\ifthenelse{\boolean{InTextBook}}{\newpage}{}}
\newcommand{\workbookpagebreak}{\ifthenelse{\boolean{InWorkBook}}{\newpage}{}}
\newcommand{\hintspagebreak}{\ifthenelse{\boolean{InHints}}{\newpage}{}}

\newcommand{\hint}[1]{\ifthenelse{\boolean{InHints}}{ {\par \hspace{12pt} \color[rgb]{0,0,1} #1 } }{}}
\newcommand{\inlinehint}[1]{\ifthenelse{\boolean{InHints}}{ { \color[rgb]{0,0,1} #1 } }{}}

\newlength{\cwidth}
\newcommand{\cents}{\settowidth{\cwidth}{c}%
\divide\cwidth by2
\advance\cwidth by-.1pt
c\kern-\cwidth
\vrule width .1pt depth.2ex height1.2ex
\kern\cwidth}

\newcommand{\sageprompt}{ {\tt sage$>$} }
\newcommand{\tab}{\rule{20pt}{0pt}}
\newcommand{\blnk}{\rule{1.5pt}{0pt}\rule{.4pt}{1.2pt}\rule{9pt}{.4pt}\rule{.4pt}{1.2pt}\rule{1.5pt}{0pt}}
\newcommand{\suchthat}{\; \rule[-3pt]{.5pt}{13pt} \;}
\newcommand{\divides}{\!\mid\!}
\newcommand{\tdiv}{\; \mbox{div} \;}
\newcommand{\restrict}[2]{#1 \,\rule[-4pt]{.25pt}{14pt}_{\,#2}}
\newcommand{\lcm}[2]{\mbox{lcm} (#1, #2)}
\renewcommand{\gcd}[2]{\mbox{gcd} (#1, #2)}
\newcommand{\Naturals}{{\mathbb N}}
\newcommand{\Integers}{{\mathbb Z}}
\newcommand{\Znoneg}{{\mathbb Z}^{\mbox{\tiny noneg}}}
\ifthenelse{\boolean{ZeroInNaturals}}{%
  \newcommand{\Zplus}{{\mathbb Z}^+} }{%
  \newcommand{\Zplus}{{\mathbb N}} }
\newcommand{\Enoneg}{{\mathbb E}^{\mbox{\tiny noneg}}}
\newcommand{\Qnoneg}{{\mathbb Q}^{\mbox{\tiny noneg}}}
\newcommand{\Rnoneg}{{\mathbb R}^{\mbox{\tiny noneg}}}
\newcommand{\Rationals}{{\mathbb Q}}
\newcommand{\Reals}{{\mathbb R}}
\newcommand{\Complexes}{{\mathbb C}}
%\newcommand{\F2}{{\mathbb F}_{2}}
\newcommand{\relQ}{\mbox{\textsf Q}}
\newcommand{\relR}{\mbox{\textsf R}}
\newcommand{\nrelR}{\mbox{\raisebox{1pt}{$\not$}\rule{1pt}{0pt}{\textsf R}}}
\newcommand{\relS}{\mbox{\textsf S}}
\newcommand{\relA}{\mbox{\textsf A}}
\newcommand{\Dom}[1]{\mbox{Dom}(#1)}
\newcommand{\Cod}[1]{\mbox{Cod}(#1)}
\newcommand{\Rng}[1]{\mbox{Rng}(#1)}

\DeclareMathOperator\caret{\raisebox{1ex}{$\scriptstyle\wedge$}}

\newtheorem*{defi}{Definition}
\newtheorem*{exer}{Exercise}
\newtheorem{thm}{Theorem}[section]
\newtheorem*{thm*}{Theorem}
\newtheorem{lem}[thm]{Lemma}
\newtheorem*{lem*}{Lemma}
\newtheorem{cor}{Corollary}
\newtheorem{conj}{Conjecture}

\renewenvironment{proof}%
{\begin{quote} \emph{Proof:} }%
{\rule{0pt}{0pt} \newline \rule{0pt}{15pt} \hfill Q.E.D. \end{quote}}


\newcommand{\vs}{\rule{0pt}{12pt}}

\AtBeginSection[]
{
 \begin{frame}{Table of Contents} 
  \tableofcontents[currentsection]
 \end{frame}
}

%%%% SAVE %%%%
%{ %magic to get a full screen image...
%\setbeamertemplate{navigation symbols}{}  % hide navigation buttons 
%\setbeamertemplate{background canvas}{\centerline{\includegraphics 
%	[height=\paperheight]{Cantor_4.jpeg}}}
%\begin{frame}[plain]
%\rule{0pt}{0pt}
%\end{frame} 
%} %end of magic


\begin{document}

\begin{frame}[plain]
  \titlepage
\end{frame}


\section{even and odd}

\begin{frame}{even}
\begin{itemize}
\item We've played with the concept of ``even'' a few times so far, but now we're ready to give a precise definition using standard mathematical notation. \pause
\item \rule{0pt}{18pt} In words: \newline
An integer $n$ is even exactly when there is an integer $m$ such that $n = 2m$. \pause

\item \rule{0pt}{18pt} In notation: \newline
An integer $n$ is even \hspace{8pt} $\iff$ \hspace{8pt}  $\exists m \in \Integers, \; n = 2m$. \pause

\item \rule{0pt}{18pt} Notice that ``exactly when'' has become a two-headed arrow, ``there is'' has turned into the existential quantifier $\exists$, and ``such that'' was totally eliminated.
\end{itemize}
\end{frame}

\begin{frame}{odd}
\begin{itemize}
\item \rule{0pt}{18pt} In words: \newline
An integer $n$ is odd exactly when there is an integer $m$ such that $n = 2m+1$. \pause

\item \rule{0pt}{18pt} In notation: \newline
An integer $n$ is odd \hspace{8pt} $\iff$ \hspace{8pt}  $\exists m \in \Integers, \; n = 2m+1$. \pause

\end{itemize}
\end{frame}

\section{decimal and base-n notation}

\begin{frame}{base 10}
\begin{itemize}
\item We need to highlight something that you're so very familiar with that you probably have ceased noticing it.  (Like the ``Are fish aware of water'' conundrum.) \pause
\item When you write down a number, like (just for example) 147, you are using place notation. \pause
\item Place notation allows us to write arbitrarily large numbers using just 10 symbols. \pause
\item The number 147 really means $1\cdot 100 + 4\cdot 10 + 7$. \pause
\item We can also rewrite the position values using exponents: $1\cdot 10^2 + 4\cdot 10^1 + 7\cdot 10^0$.\pause
\end{itemize}
\end{frame}

\begin{frame}{overcoming prejudice}
\begin{itemize}
\item The powers of 10 in the forgoing are a result of human beings having 10 fingers. \pause
\item People might reasonably have used 5 or even 20 as the base of our numbering scheme. \pause
\item In ancient Babylon they actually used a base of 60 for their numbers!  There remain remnants of this to the current day.  (60 minutes in an hour, 360 degrees in a circle\textellipsis) \pause
\end{itemize}
\end{frame}

\begin{frame}{}
\begin{itemize}
\item Let's try counting in base-5. \pause
\item Little twelve-toes. \pause
\item Numbers in an arbitrary base $b$:

\[ x_n \ldots x_2 x_1 x_0 \quad \mbox{means} \quad x_n\cdot b^n + \ldots + x_2\cdot b^2 + x_1 \cdot b^1 + x_0 \cdot b^0 \]
\pause

\item Use a subscript to indicate the base if you're using something besdides 10: 

\[ 12345_7 \quad = \quad 1\cdot 7^4 + 2\cdot 7^3 + 3\cdot 7^2 + 4\cdot 7 + 5 \quad = \quad 3267 \]

\pause

\item In base $b$ the digits run from 0 to $b-1$ (Because $b$ itself will be written like 10: $10_b$ \pause  

\end{itemize}
\end{frame}

\begin{frame}{CS bases}
\begin{itemize}
\item binary \pause

digits are just 0 and 1 \pause

\item octal \pause

digits are in $\{0, 1, 2, 3, 4, 5, 6, 7 \}$ \pause

\item hexa-decimal \pause

digits are 0--9 and then A--F.

\end{itemize}
\end{frame}

\begin{frame}{conversions}
\begin{itemize}
\item To go from some unusual base to decimal, just use the meaning of place value. \pause
\item Example: $123_5$ means $1\cdot 5^2 + 2 \cdot 5 + 3$ \hspace{12pt} that last thing can just be calculated as an ordinary base 10 number (it's 38.) \pause
\item To go from decimal to another base you can use the repeated division algorithm. Be careful, you'll find the digits of your answer in reverse order. \pause 
\end{itemize}
\end{frame}

\begin{frame}{CS conversions}
\begin{itemize}
\item For the bases common in CS you can use binary as a {\em lingua franca} \pause
\item See the table on page 33.
\end{itemize}
\end{frame}

\section{divisibility}

\begin{frame}{gazinta}
\begin{itemize}
\item There is a very succinct notation for the situation where one number divides evenly into another.\pause

\item $7 \divides 42$ means that there is no remainder if you try to divide $7$ into $42$. \pause
\item Notice how the order of the numbers is flipped around (as well as $\mid$ vs. $/$) when you compare the actual division to the divisibility question.\newline

$7\divides 42$ is TRUE \hspace{.3in} $42/7$ is SIX \pause

\item Similarly, \newline

$5 \divides 37$ is FALSE, \hspace{.1in}while\hspace{.1in} $37/5$ is $7.4$
 \end{itemize}
\end{frame}

\begin{frame}{definition}
\begin{itemize}
\item The formal definition of the divisibility symbol actually avoids doing division at all! \pause
\item \rule{0pt}{18pt}  $d \divides n$ \hspace{8pt} $\iff$ \hspace{8pt}  $\exists k \in \Integers, \; n = kd$. \pause
\item Pop Quiz!  What number $k$ must exist to show that $7\divides 91$ is true? 
\end{itemize}
\end{frame}


\section{floor and ceiling}

\begin{frame}{rounding up}
\begin{itemize}
\item The {\em ceiling} notation looks rather like absolute value bars.  Note the additional marks at the top of the bars. \pause
\item $\lceil \pi \rceil \; = \; 4$ \pause
\item The basic idea is to use ceiling notation where rounding up is appropriate. \pause

\item The formal definition is:

$ \displaystyle \forall x \in \Reals, y = \lceil x \rceil$ \hspace{8pt} $\iff$ \hspace{8pt} $y \in \Integers \; \mbox{and} \; y-1 < x \leq y$ \pause

\item (Translaton) 
\end{itemize}
\end{frame}

\begin{frame}{rounding down}
\begin{itemize}
\item The {\em floor} notation is similar. \pause
\item $\lfloor \pi \rfloor \; = \; 3$ \pause
\item The formal definition is:

$ \displaystyle \forall x \in \Reals, y = \lfloor x \rfloor$ \hspace{8pt} $\iff$ \hspace{8pt} $y \in \Integers \; \mbox{and} \; y \leq x < y+1$ \pause
\end{itemize}
\end{frame}


\section{div and mod}

\begin{frame}{}
\begin{itemize}
\item When we're thinking about dividing a number $n$ by a divisor $d$, there are standard ways to refer to the quotient and remainder we get. \pause
\item Suppose dividing $n$ by $d$ results in a quotient $q$ and a remainder $r$. \pause
\item We write $n \mod d$ for the remainder and $n \,\mbox{div}\, d$ for the quotient. \pause
\item Examples: $17 \mod 5 = 2$ and $37 \mod 6 = 1$. \pause
\item Note that $n \mod d = 0$ is equivalent to $d\divides n$.
\end{itemize}
\end{frame}

\section{binomial coefficients}

\begin{frame}{Pascal's triangle}

\centerline{
\begin{tabular}{ccccccccccccc}
 &   &   &   &   &   & 1 &   &   &   &   &   & \\
 &   &   &   &   & 1 &   & 1 &   &   &   &   & \\
 &   &   &   & 1 &   & 2 &   & 1 &   &   &   & \\ 
 &   &   & 1 &   & 3 &   & 3 &   & 1 &   &   & \\
 &   & 1 &   & 4 &   & 6 &   & 4 &   & 1 &   & \\
 & 1 &   & 5 &   &10 &   &10 &   & 5 &   & 1 & \\
1&   & 6 &   &15 &   &20 &   &15 &   & 6 &   & 1
\end{tabular}
}
\end{frame}


\begin{frame}{choose}
\begin{itemize}
\item Both the ``columns'' and the rows of Pascal's triangle are numbered starting at 0. \pause
\item The columns are actually slanted at a fairly large angle.  We could square up the triangle but -- tradition\textellipsis \pause
\item The entry in row $n$ and column $k$ is written $\displaystyle \binom{n}{k}$. \pause
\item It looks like someone left out a fraction bar, but don't be tempted to add one, this is something different! \pause
\item The symbol  $\displaystyle \binom{n}{k}$ is pronounced ``n choose k.''  It counts the number of ways we could choose $k$ things from a set of size $n$.
\end{itemize}
\end{frame}

\begin{frame}{a formula}
\begin{itemize}
\item Fortunately there is a formula, but unfortunately it's written in terms of factorials. \pause
\item The factorial of a number $n$ is written $n!$. The value of $n!$ is the product of all the integers between $1$ and $n$. \pause
\item For example $7! \; = \; 7\cdot 6 \cdot 5 \cdot 4 \cdot 3 \cdot 2 \cdot 1 \; = \; 5040$. \pause 
\item Finally, the formula for the ``choose'' numbers:

\[ \binom{n}{k} \quad = \quad \frac{n!}{k! (n-k)!}. \]
\pause 
\item Or you can use your calculator. 
\end{itemize}
\end{frame}

\end{document}
