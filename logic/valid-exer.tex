\begin{enumerate}
\item Determine the logical form of the following arguments.  Use symbols
to express that form and determine whether the form is valid or invalid.
If the form is invalid, determine the type of error made.  Comment on the 
soundness of the argument as well, in particular, determine whether any of
the premises are questionable.
\begin{enumerate}
\item All who are guilty are in prison. \newline
  George is not in prison.  \newline
  Therefore, George is not guilty.
 
 \wbvfill
 
  \hint{ 
  This looks like modus tollens. Let $G$ refer to ``guilt'' and $P$ refer to ``in prison''
  
\begin{center}
\begin{tabular}{cl}
 & $\forall x, G(x) \implies P(x)$ \\
 & ${\lnot}P(\mbox{George}) $ \\ \hline
$\therefore$ & ${\lnot}G(\mbox{George})$ \\
\end{tabular}
\end{center}

You should note that while the form is valid, there is something terribly wrong with this argument. Is it really true that everyone who is guilty of a crime is in prison?
}

\item If one eats oranges one will have high levels of vitamin C. \newline
  You do have high levels of vitamin C. \newline
  Therefore, you must eat oranges.
  
  \wbvfill

\workbookpagebreak

\item All fish live in water. \newline
  The mackerel is a fish. \newline
  Therefore, the mackerel lives in water. 
  
  \wbvfill

\item If you're lazy, don't take math courses.\newline
  Everyone is lazy. \newline
  Therefore, no one should take math courses.
  
  \wbvfill

\item All fish live in water. \newline
  The octopus lives in water. \newline
  Therefore, the octopus is a fish.
  
  \wbvfill

\item If a person goes into politics, they are a scoundrel.\newline
  Harold has gone into politics. \newline
  Therefore, Harold is a scoundrel. 
\end{enumerate}

\wbvfill

\workbookpagebreak

\item Below is a rule of inference that we call extended elimination.

\begin{tabular}{cl}
 & $(A \lor B) \lor C$ \\
 & ${\lnot}A$ \\
 & ${\lnot}B$ \\ \hline
$\therefore$ & $C$ \\
\end{tabular}

Use a truth table to verify that this rule is valid.

\hint{

\vfill

In the following truth table the predicate variables occupy the first 3 columns, the argument's 
premises are in the next three columns and the conclusion is in the right-most column.  The
truth values have already been filled-in.  You only need to identify the critical rows and 
verify that the conclusion is true in those rows.

\vfill

 \newpage
 
\begin{tabular}{|c|c|c||c|c|c||c|} \hline
\rule[-8pt]{0pt}{30pt}$A$ & $B$ & $C$ & $(A \lor B) \lor C$ & \rule{20pt}{0pt} ${\lnot}A$ \rule{20pt}{0pt} & \rule{20pt}{0pt} ${\lnot}B$ \rule{20pt}{0pt} & \rule{20pt}{0pt} $C$ \rule{20pt}{0pt} \\ \hline
\rule[-8pt]{0pt}{30pt}$T$ & $T$ & $T$ & $T$ & $\phi$ & $\phi$ & $T$  \\ \hline
\rule[-8pt]{0pt}{30pt}$T$ & $T$ & $\phi$  & $T$ & $\phi$ & $\phi$ & $\phi$   \\ \hline
\rule[-8pt]{0pt}{30pt}$T$ & $\phi$  & $T$ & $T$ & $\phi$ & $T$  & $T$  \\ \hline
\rule[-8pt]{0pt}{30pt}$T$ & $\phi$  & $\phi$  & $T$ & $\phi$ & $T$ & $\phi$   \\  \hline
\rule[-8pt]{0pt}{30pt}$\phi$  & $T$ & $T$ & $T$ & $T$ & $\phi$ &  $T$ \\ \hline
\rule[-8pt]{0pt}{30pt}$\phi$  & $T$ & $\phi$  & $T$ & $T$ & $\phi$ & $\phi$  \\ \hline
\rule[-8pt]{0pt}{30pt}$\phi$  & $\phi$  & $T$ & $T$ & $T$ & $T$ & $T$  \\ \hline
\rule[-8pt]{0pt}{30pt}$\phi$  & $\phi$  & $\phi$  & $\phi$ & $T$ & $T$ & $\phi$  \\  \hline
\end{tabular}

\vfill
}

\workbookpagebreak

\item If we allow quantifiers and open sentences in an argument form we
get a couple of new argument forms.  Arguments involving existentially quantified 
premises are rare -- the new forms we are speaking of are called ``universal modus 
ponens'' and ``universal modus tollens.''   The minor premises may also be quantified
or they may involve particular elements of the universe of discourse -- this leads
us to distinguish argument subtypes that are termed ``universal'' and ``particular.''

For example  \begin{tabular}{cl}
 & $\forall x, A(x) \implies B(x)$ \\
 & $A(p)$ \\ \hline
$\therefore$ & $B(p)$ \\
\end{tabular}  is the particular form of universal modus ponens (here, $p$
is not a variable -- it stands for some particular element of the universe of
discourse)
and \begin{tabular}{cl}
 & $\forall x, A(x) \implies B(x)$ \\
 & $\forall x, {\lnot}B(x)$ \\ \hline
$\therefore$ & $\forall x, {\lnot}A(x)$ \\
\end{tabular} is the universal form of (universal) modus tollens.

Reexamine the arguments from problem (1), determine their forms
(including quantifiers) and whether they are universal or particular.

\hint{
Hint: All of them except for one are the particular form -- number 4 is the exception.

Here's an analysis of number 5:

All fish live in water. \newline
The octopus lives in water.  \newline
Therefore, the octopus is a fish. \newline

Let $F(x)$ be the open sentence ``x is a fish'' and let $W(x)$ be the open sentence ``x lives in water.''

Our argument has the form

 \begin{center}
\begin{tabular}{cl}
 & $\forall x, F(x) \implies W(x)$ \\
 & $W(\mbox{the octopus}) $ \\ \hline
$\therefore$ & $F(\mbox{the octopus})$ \\
\end{tabular}
\end{center}

Clearly something is wrong -- a converse error has been made -- if everything that lived in water was necessarily a fish the argument would be OK (in fact it would then be the particular form of universal modus ponens).  But that is the converse of the major premise given.    
}

\workbookpagebreak

\rule{0pt}{0pt}

\workbookpagebreak

\item Identify the rule of inference being used.

\begin{enumerate}
\item The Buley Library is very tall.\\
Therefore, either the Buley Library is very tall or it has many
levels underground.

\hint{disjunctive addition}
\wbvfill

\item The grass is green.\\
The sky is blue.\\
Therefore, the grass is green and the sky is blue.

\hint{conjunctive addition}
\wbvfill

\item $g$ has order 3 or it has order 4.\\
If $g$ has order 3, then $g$ has an inverse.\\
If $g$ has order 4, then $g$ has an inverse.\\
Therefore, $g$ has an inverse.

\hint{constructive dilemma}
\wbvfill

\item $x$ is greater than 5 and $x$ is less than 53.\\
Therefore, $x$ is less than 53.

\hint{conjunctive simplification}
\wbvfill

\item If $a|b$, then $a$ is a perfect square.\\
If $a|b$, then $b$ is a perfect square.\\
Therefore, if $a|b$, then $a$ is a perfect square and $b$ is
a perfect square.

\hint{Note that the conclusion could be re-expressed as the conjunction of the two conditionals that
are found in the premises.  This is conjunctive addition with a bit of ``window dressing.''}
\wbvfill

\end{enumerate}

\workbookpagebreak

\item Read the following proof that the sum of two odd numbers is even.
Discuss the rules of inference used.\\
\begin{proof}
Let $x$ and $y$ be odd numbers. Then $x=2k+1$
and $y=2j+1$ for some integers $j$ and $k$. By algebra,
\[
x+y = 2k+1 + 2j+1 = 2(k+j+1).
\]

Note that $k+j+1$ is an integer because $k$ and $j$ are integers.
Hence $x+y$ is even. 
\end{proof}

\hint{The definition for ``odd'' only involves the oddness of a single integer, but the first line of our
proof is a conjunction claiming that $x$ and $y$ are both odd.  It seems that two conjunctive simplifications, followed by applications of the definition, followed by a conjunctive addition have been used in order to
go from the first sentence to the second.}
 
 \wbvfill
 
 \rule{0pt}{0pt}
 
 \wbvfill
 
\item Sometimes in constructing a proof we find it necessary to ``weaken'' an inequality.  For example,
we might have already deduced that $x < y$ but what we need in our argument is that $x \leq y$.  It is
okay to deduce $x \leq y$ from $x < y$ because the former is just shorthand for $x<y \lor x=y$.  What
rule of inference are we using in order to deduce that $x \leq y$ is true in this situation?

\hint{disjunctive addition}

\wbvfill

\end{enumerate}
