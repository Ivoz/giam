\begin{enumerate}
\item There is a common variant of the existential quantifier,
$\exists !$, if you write $\exists ! \, x, \, P(x)$ you are asserting 
that there is a \index{unique existence}\emph{unique} element 
in the universe that makes $P(x)$ true.
Determine how to negate the sentence $\exists ! \, x, \, P(x)$.

\hint{
Unique existence is essentially saying that there is exactly 1 element of the universe of discourse that makes P(x) true. The negation of "there is exactly 1" is "there's either none, or at least 2".

Is that enough of a hint?
}

\item The order in which quantifiers appear is important.  Let $L(x,y)$
be the open sentence ``$x$ is in love with $y$.''  Discuss the meanings of the
following quantified statements and find their negations.

\begin{enumerate}
\item $\forall x \, \exists y \; L(x,y)$.
\item $\exists x \, \forall y \; L(x, y)$.
\item $\forall x \, \forall y \; L(x, y)$.
\item $\exists x \, \exists y \; L(x, y)$.
\end{enumerate}

\hint{

\begin{enumerate}
\item $\forall x \, \exists y \; L(x,y)$.

This is a fairly optimistic statement  ``For everyone out there, there's somebody that they are in love with.''

\item $\exists x \, \forall y \; L(x, y)$.

This one, on the other hand, says something fairly strange: ``There's someone who has fallen in love with every other human being.'' I don't know, maybe the Dalai Lama? Mother Theresa?...
Anyway, do the last two for yourself.

\item $\forall x \, \forall y \; L(x, y)$.
\item $\exists x \, \exists y \; L(x, y)$.

\vspace{.5in}

Here's a couple of bonus questions. Two of the statements above have different meanings if you just interchange the order that the quantifiers appear in. What do the following mean (in contrast to the ones above)?

\item $\exists y \, \forall x \; L(x, y)$.
\item $\forall y \, \exists x \; L(x,y)$.
\end{enumerate}

}

\item Determine a useful denial of: 

$\displaystyle \forall \epsilon>0 \, \exists 
\delta>0 \, \forall x \, (|x-c| < \delta) \implies (|f(x)-l| < \epsilon) $.

The denial above gives a criterion for saying $\lim_{x\rightarrow c}f(x) \neq l.$

\hint{
This is asking you to put a couple of things together. The first thing is that in negating a quantified statement, we get a new statement with all the quantified variables occurring in the same order but with $\forall$'s and $\exists$'s interchanged. The second issue is that the logical statement that appears after all the quantifiers needs to be negated. Since, in this statement we have a conditional, you must remember to negate that properly (its negation is a conjunction).

$\displaystyle \exists \epsilon>0 \, \forall 
\delta>0 \, \exists x \, (|x-c| < \delta)  \land  (|f(x)-l| \geq \epsilon) $.

}

\item A \index{Sophie Germain prime} \emph{Sophie Germain prime} is a prime number $p$
such that the corresponding odd number $2p+1$ is also a prime.  For example 11 is a 
Sophie Germain prime since $23 = 2\cdot 11 + 1$ is also prime.  Almost all Sophie Germain
primes are congruent to $5 \pmod{6}$, nevertheless, there are exceptions -- so the
statement ``There are Sophie Germain primes that are not 5 mod 6.'' is true.  Verify this.

\hint{The exceptions are very small prime numbers. You should be able to find them easily.}

\item  Alvin, Betty, and Charlie enter a cafeteria which offers three different
entrees, turkey sandwich, veggie burger, and pizza; four different
beverages, soda, water, coffee, and milk; and two types of desserts,
pie and pudding. Alvin takes a turkey sandwich, a soda, and a pie.
Betty takes a veggie burger, a soda, and a pie. Charlie takes a pizza
and a soda. Based on this information, determine whether the following
statements are true or false.

\begin{enumerate}
\item \label{negated}$\forall$ people $p$, $\exists$ dessert $d$ such that $ p$
took $d$. \hint{false}
\item \label{compare}$\exists$ person $p$ such that $\forall$ desserts
$d$, $p$ did not take $d$. \hint{true}
\item $\forall$ entrees $e$, $\exists$ person $p$ such that $ p$ took
$e$. \hint{true}
\item \label{entree}$\exists$ entree $e$ such that  $\forall$ people
$p,\ p$ took $e$. \hint{false}
\item $\forall$ people $p$, $p$ took a dessert $\iff p$ did not take
a pizza. \hint{true}
\item Change one word of statement \ref{entree} so that it becomes true. \hint{entree $\longrightarrow$ beverage}
\item Write down the negation of \ref{negated} and compare it to statement
\ref{compare}. Hopefully you will see that they are the same! Does
this make you want to modify one or both of your answers to \ref{negated}
and \ref{compare}? \hint{$\exists$ person $p$ such that $\forall$ desserts
$d$, $p$ did not take $d$. Yes I do.  No, I got them right in the first place!}
\end{enumerate}

\end{enumerate}
