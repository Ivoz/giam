\chapter*{Preface: for Instructors}
\chaptermark{For Instructors}

At many universities and colleges in the United States a course which
provides a transition from lower-level mathematics courses to those in
the major has been adopted.  Some may find it hard to believe that
a course like Calculus II is considered ``lower-level'' so let's drop
the pejoratives and say what's really going on.   Courses for 
Math majors, and especially those one takes in the Junior and Senior 
years, focus on proofs --- students are expected to learn {\em why} a
given statement is true, and be able to come up with their own convincing 
arguments concerning such ``why''s.   Mathematics courses that precede 
these typically focus on ``how.''  How does one find the minimum value a
continuous function takes on an interval?  How does one determine the 
arclength along some curve.  Et cetera.  The essential {\em raison d'etre}
of this text and others like it is to ease this transition from ``how'' 
courses to ``why'' courses.  In other words, our purpose is to help 
students develop a certain facility with mathematical proof.

It should be noted that helping people to become good proof writers --
the primary focus of this text -- is, very nearly, an impossible task. 
Indeed, it can be argued that the best way to learn to write proofs
is by writing a lot of proofs.  Devising many different proofs, and
doing so in various settings, definitely develops the facility we hope
to engender in a so-called ``transitions'' course.  Perhaps the
pedagogical pendulum will swing back to the previous tradition of
essentially throwing students to the wolves.  That is, students might
be expected to learn the art of proof writing while actually writing
proofs in courses like algebra and analysis\footnote{At the University
of Maryland, Baltimore County, where I did my undergraduate work, 
these courses were actually known as the ``proofs''
courses.}.  Judging from the feedback I receive from students who have 
completed our transitions course at Southern Connecticut State
University, I think such a return to the methods of the past is
unlikely.  The benefits of these transitions courses are enormous, and
even though the curriculum for undergraduate Mathematics majors is
an extremely full one, the place of a transition course is, I think, assured.  

What precisely are the benefits of these transitions courses?  One of
my pet theories is that the process one goes through in learning to
write and understand proofs represents a fundamental reorganization of
the brain.  The only evidence for this stance, albeit rather indirect,
are the almost universal reports of ``weird dreams'' from students in
these courses.  Our minds evolved in a setting where inductive
reasoning is not only acceptable, but advisable in coping with the
world.  Imagine some Cro Magnon child touching a burning branch and
being burned by it.  S/He quite reasonably draws the conclusion that
s/he should not touch {\em any} burning branches, or indeed anything
that is on fire.  A Mathematician has to train him or herself to think
strictly by the rules of deductive reasoning -- the above experience 
would only provide the lesson that at that particular instant of time,
that particular burning branch caused a sensation of pain.  Ideally,
no further conclusions would be drawn -- obviously this is an
untenable method of reasoning for an animal driven by the desire to
survive to adulthood, but it is the {\em only} way to think in the
artificial world of Mathematics. 

While a gentle introduction to the art of reading and writing proofs
is the primary focus of this text, there are other subsidiary goals
for a transitions course that we hope to address.  Principal among
these is the need for an introduction to the ``culture'' of
Mathematics.  There is a shared mythos and language common to all
Mathematicians -- although there are certainly some distinct dialects!
Another goal that is of extraordinary importance is impressing the
budding young Mathematics student with the importance of play.  My
thesis adviser\footnote{Dr.\ Vera Pless, to whom I am indebted in more
ways than I can express.} used to be famous for saying ``Well, I don't
know! Why don't you monkey around with it a little \ldots ''  In the course
of monkeying around -- doing small examples by hand, trying bigger
examples with the aid of a computer, changing some element of the
problem to see how it affected the answer, and various other
activities that can best be described as ``play,'' eventually patterns
emerged, conjectures made themselves apparent, and possible proof
techniques suggested themselves.  In this text there are a great many
open-ended problems, some with associated hints as to how to proceed
(which the wise student will avoid until hair-thinning becomes
evident), whose point is to introduce students to this process of
mathematical discovery.

To recap, the goals of this text are: an introduction to reading and
writing mathematical proofs, an introduction to mathematical culture,
and an introduction to the process of discovery in Mathematics.  Two
pedagogical principles have been of foremost importance in determining
{\em how} this material is organized and presented.  One is the
so-called ``rule of three'' which is probably familiar to most
educators.  Propounded by (among others) Hughes, Hallett, et al. in their
reform Calculus it states that, when possible, information should be
delivered via three distinct mechanisms -- symbolically, graphically and
numerically.  The other is also a ``rule of three'' of sorts, it is captured by
the old speechwriter's maxim -- ``Tell 'em what you're gonna tell 'em.
Tell 'em.  Then tell 'em what you told 'em.''  Important and/or difficult
topics are revisited at least three times in this book.  In marked contrast to
the norm in Mathematics, the first treatment of a topic is {\em not}
rigorous, precise definitions are often withheld.  The intent is to
provide a bit of intuition regarding the subject material.  Another
reason for providing a crude introduction to a topic before giving
rigorous detail revolves around the way human memory works.  Unlike
computer memory, which (excluding the effects of the occasional
cosmic ray) is essentially perfect, animal memory is usually imperfect
and mechanisms have evolved to ensure that data that are important to
the individual are not lost.  Repetition and rote learning are often
derided these days, but the importance of multiple exposures to a
concept in ``anchoring'' it in the mind should not be underestimated. 

A theme that has recurred over and over in my own thinking about
the transitions course is that the ``transition'' is that from inductive
to deductive mental processes.  Yet, often, we the instructors of these courses
are ourselves so thoroughly ingrained with the deductive approach
that the mode of instruction presupposes the very transition we 
hope to facilitate!  In this book I have, to a certain extent, 
taken the approach of
teaching deductive methods using inductive ones.  The first 
time a concept is encountered should only be viewed as providing
evidence that lends credence to some mathematical truth.  Most 
concepts that are introduced in this intuitive fashion are eventually
exposited in a rigorous manner -- there are exceptions though, ideas
whose scope is beyond that of the present work which are nonetheless
presented here with very little concern for precision.  It should
not be forgotten that a good transition ought to blend seamlessly 
into whatever follows.  The courses that follow this material 
should be proof-intensive courses in geometry, number theory, 
analysis and/or algebra.
The introduction of some material from these courses without the
usual rigor is intentional. 

Please resist the temptation to fill in the missing ``proper'' 
definitions and terminology when some concept is introduced and
is missing those, uhmm, missing things.  Give your students the
chance to ruminate, to ``chew''\footnote{Why is it that most %
of the metaphorical ways to refer to ``thinking'' actually seem % 
to refer to ``eating''?} on these new concepts for a while
{\em on their own!}  Later we'll make sure they get the same 
standard definitions that we all know and cherish.  As a practical
matter, if you spend more than 3 weeks in Chapter~1, you are 
probably filling in too much of that missing detail -- so stop it.
It really won't hurt them to think in an imprecise way (at first)
about something so long as we get them to be rigorous by the 
end of the day.   

Finally, it will probably be necessary to point out to your
students that they should actually {\em read} the text.  I
don't mean to be as snide as that probably sounds\ldots  Their
experiences with math texts up to this point have probably impressed
them with the futility of reading --- just see what kind of problems
are assigned and skim 'til you find an example that shows you ``how
to do one like that.''  Clearly such an approach is far less fruitful
in advanced study than it is in courses which emphasize learning
calculational techniques.  I find that giving expressed reading 
assignments and quizzing them on the material that they are supposed
to have read helps.  There are ``exercises'' given within most 
sections (as opposed to the ``Exercises'' that appear at the end
of the sections) these make good fodder for quizzes and/or probing 
questions from the professor.  The book is written in an expansive,
friendly style with whimsical touches here and there.  Some students
have reported that they actually enjoyed reading it!\footnote{Although %
it should be added that they were making that report to someone %
from whom they wanted a good grade.}
  
Earlier versions of this text included a systematic error regarding how
the natural numbers are defined.  In the first chapter they began with $1$
and somewhat later the convention shifted so that the smallest natural 
was $0$.  I received quite a bit of more or less good natured ribbing 
about this ``continuity error.''  In the current version I have tried to make it explicit -- the 
change happens in the last paragraph of Section~\ref{sec:induct}.  

%% Emacs customization
%% 
%% Local Variables: ***
%% TeX-master: "GIAM.tex" ***
%% comment-column:0 ***
%% comment-start: "%% "  ***
%% comment-end:"***" ***
%% End: ***
 
